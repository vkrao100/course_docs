\documentclass[11pt]{article}
\usepackage{amsmath,amssymb,amsthm}
\usepackage{graphicx}
\usepackage[margin=1in]{geometry}
\usepackage{fancyhdr}
\usepackage{tkz-berge}
\usetikzlibrary{positioning,chains,fit,shapes,calc}
\setlength{\parindent}{0pt}
\setlength{\parskip}{5pt plus 1pt}
\setlength{\headheight}{13.6pt}
\newcommand\question[2]{\vspace{.25in}\hrule\textbf{#1: #2}\vspace{.5em}\hrule\vspace{.10in}}
\renewcommand\part[1]{\vspace{.10in}\textbf{(#1)}}
\newcommand\algorithm{\vspace{.10in}\textbf{Algorithm: }}
\newcommand\correctness{\vspace{.10in}\textbf{Correctness: }}
\newcommand\runtime{\vspace{.10in}\textbf{Running time: }}
\pagestyle{fancyplain}
\lhead{\textbf{\NAME\ (\UID)}}
\chead{\textbf{HW\HWNUM}}
\rhead{ECE 6810, \today}
\begin{document}\raggedright
%Section A==============Change the values below to match your information==================
\newcommand\NAME{Vikas Kumar Rao}  % your name
\newcommand\UID{U1072596}     % your utah UID
\newcommand\HWNUM{1}              % the homework number
%Section B==============Put your answers to the questions below here=======================
% no need to restate the problem --- the graders know which problem is which,
% but replacing "The First Problem" with a short phrase will help you remember
% which problem this is when you read over your homeworks to study.
\question{1}{summarizing performance numbers}
\part{a}
Let us compute the execution time of all the systems. We use arithmetic mean for computation as it is good for time and latencies.

$average time = \frac{1}{n}\sum\limits_{i=1}^n x_i$

BASE  - $\frac{3+2.5+1+12}{4} = 4.625s$\\
NEW1  - $\frac{7+3+5+1}{4} = 4s$\\
NEW2  - $\frac{2+1+3+8}{4} = 3.5s$\\
NEW3  - $\frac{1+3+2+13}{4} = 4.75s$\\

We can see that system `NEW2' has the least average execution time.

\part{b}
Let us compute the average energy consumption of all the systems. We use arithmetic mean for computation as it is good for time and latencies.

$average time = \frac{1}{n}\sum\limits_{i=1}^n x_i$

BASE  - $\frac{20+40+50+15}{4} = 31.25J$\\
NEW1  - $\frac{10+30+15+30}{4} = 21.25J$\\
NEW2  - $\frac{30+60+20+20}{4} = 32.5J$\\
NEW3  - $\frac{70+35+30+10}{4} = 36.25J$\\

We can see that system `NEW1' has the least energy consumption.

\part{c}
we know that 

$energy = Power \times time$\\
$Power = \frac{Energy}{time}$

\begin{table}[h]
\centering
\label{powerTable}
\caption{Power across machines}
\begin{tabular}{|c|c|c|c|c|}
 \hline
 \multicolumn{5}{|c|}{Individual Power}\\
 \hline
 BASE&$\frac{20}{3}=6.67$&$\frac{40}{2.5}=26$&$\frac{50}{1}=50$&$\frac{15}{12}=1.25$\\
 \hline
 NEW1&$\frac{10}{7}=1.42$&$\frac{30}{3}=10$&$\frac{15}{5}=3$&$\frac{30}{1}=30$\\
 \hline
 NEW2&$\frac{30}{2}=15$&$\frac{60}{1}=60$&$\frac{20}{3}=6.67$&$\frac{20}{8}=2.5$\\
 \hline
 NEW3&$\frac{70}{1}=70$&$\frac{35}{3}=11.67$&$\frac{30}{2}=15$&$\frac{10}{13}=0.76$\\
 \hline
\end{tabular}
\end{table}

calculating the power across machines for all applications in table ~\ref{powerTable}

Average power of the systems can be calculated using arithmetic mean

BASE  - $\frac{6.67+26+50+1.25}{4} = 18.48W$\\
NEW1  - $\frac{1.42+10+3+30}{4} = 11.1W$\\
NEW2  - $\frac{15+60+6.67+2.5}{4} = 21.05W$\\
NEW3  - $\frac{70+11.67+15+0.76}{4} = 24.35W$\\

From the results, it is clear that system `NEW1' consumes less power.

\question{2}{Optimizing CPU time}

\part{a}

Old processor-\\
We know that average Cycles per instruction(CPI) is given as \\
$CPI_{avg} = \sum frequency\* cycles$\\
$CPI_{avg} = 0.1*2+0.05*1+0.05*2+0.3*1+0.5*4 = 2.65$

average PCI is given as\\
$PCI_{avg}=\frac{1}{CPI_{avg}} = \frac{1}{2.65} = 0.377$ for old processor

New processor-\\
Assumptions - For old processor, Let us assume that the total instructions are 100.  
Let us calculate the total instructions/ frequencies for the new processor.\\ 
Out of 100, number of MULT instruction = 50. 60\% of them are MULT followed by ADD, thus \\
0.6*50 = 30

calculating the instruction frequencies in new processor-\\
number of MULT followed by ADD - FMAD = 30\\
number of MULT remaining = 50 - 30 = 20\\
number of ADD  remaining = 30 - 30 = 0 (since 30 of these instructions are executed in FMAD)\\

To normalize the instruction freqeuncy, we need to find the total instructions in new processor. 

total instructions  = 20 + 30 + 0 + 10 + 5 + 5 = 70 

number of Load remaining = $\frac{10}{70}*100 = 14.28\%$ 
number of Store remaining = $\frac{5}{70}*100 = 7.14\%$ 
number of Branch remaining = $\frac{5}{70}*100 = 7.14\%$ 
number of ADD remaining = $\frac{0}{70}*100 = 0\% $
number of MULT remaining = $\frac{20}{70}*100 = 28.59\% $
number of FMAD remaining = $\frac{30}{70}*100 = 42.85\% $

Now using the same formula for $CPI_{avg} = 2*0.142+1*0.071+2*0.071+1*0+4*0.285+4*0.428 = 3.35$

$PCI_{avg} = \frac{1}{CPI_{avg}} = 0.29$ for new processor

\part{b}
Old Processor - \\
CPI = 2.65\\
InstructionCount (IC) = 100\\
CPU time = CPI*IC*CT = 100*2.65*CT = 265*CT

New Processor - \\
CPI = 3.35\\
InstructionCount (IC) = 70\\
CPU time = CPI*IC*CT = 70*3.35*CT = 235*CT

efficiency = $\frac{Old CPU time}{New CPU Time} = \frac{265*CT}{235*CT}$\\
Assuming the same cycle time(CT), we have efficiency = 1.13, since the value is more than 1, the new processor is faster(speedup) than the older processor.

\question{3}{Amdahl's Law}

From Amdahl's law of diminishing returns, we know that \\

$speedup_{overall} = \frac{ExecutionTime_{old}}{ExecutionTime_{new}}=\frac{1}{(1-Fraction_{enhanced})+\frac{Fraction_{enhanced}}{Speedup_{enhanced}}}$

Using the above equation in following sections for analyzing the energy optimizations\\
\part{a}{reducing wireless interface energy by 10\%}\\
$Fraction_{enhanced} = 50\% = 0.5$\\
$Speedup_{enhanced} = \frac{x}{(1-\% reduced)*x} = \frac{1}{0.9}=1.11$\\

substituting the values in Amdahl's equation:\\

$Speedup_{overall}=\frac{1}{(1-0.5)+\frac{0.5}{1.11}}=1.053$\\

\part{b}{reducing CPU energy by 60\%}\\
$Fraction_{enhanced} = 10\% = 0.1$\\
$Speedup_{enhanced} = \frac{x}{(1-\% reduced)*x} = \frac{1}{0.4}=2.5$\\

substituting the values in Amdahl's equation:\\

$Speedup_{overall}=\frac{1}{(1-0.1)+\frac{0.1}{2.5}}=1.064$\\

\part{c}{reducing display energy by 50\%}\\
$Fraction_{enhanced} = 20\% = 0.2$\\
$Speedup_{enhanced} = \frac{x}{(1-\% reduced)*x} = \frac{1}{0.5}=2$\\

substituting the values in Amdahl's equation:\\

$Speedup_{overall}=\frac{1}{(1-0.2)+\frac{0.2}{2}}=1.11$\\

From the above experiments, it is clear that reducing display energy by 50\% gives the best energy savings.

\question{4}{Power and energy}
We know that\\
$power = voltage \times Current(P=VI)$\\
$Energy = Power \times Time(E=PT)$\\
$Energy = (Power_{static} + Power_{dynamic})\times Time(E=PT)$\\
$power_{static} = voltage \times Current_{static}$\\
$power_{dynamic} = activity \times capacitance \times voltage^2 \times frequency$\\

given:\\
frequency ($f$) = 2 GHz\\
Application time ($t$) = 15 s\\
dynamic power $(P_{dynamic})$= 70 W\\
static power $(P_{static})$= 30 W\\

\part{a}
Using the third equation -

Energy = (70+30)*15 = 1500J

\part{b}
frequency scaled down by 30\%. Change in frequency affects the execution time and dynamic power, but the static power remains the same.\\

new frequency - $f_{new} = (1-0.3)*2GHz = 1.4GHz$\\
new time  - $t_{new} = \frac{f}{f_{new}}*time = \frac{2}{1.4}*15 = 21.43s$\\
dynamic power is directly proportinal to f, thus:\\
new dynamic power - $P_{dynamic} = \frac{70*0.7f}{f} = 49 W $\\
energy = (30+49)*21.43 = 1693J

\part{c}
both voltage and frequency scaled down by 30\%. Hence time, static power and dynamic power change.\\
new frequency - $f_{new} = (1-0.3)*2GHz = 1.4GHz$\\
new time  - $t_{new} = \frac{f}{f_{new}}*time = \frac{2}{1.4}*15 = 21.43s$\\
new static power - $P_{static} = 30*0.7 = 21 W$(from 4th equation)\\
new dynamic power - $P_{dynamic} = 70 * 0.7 * 0.7^2 = 24.01 W$(from 5th equation)\\ 

energy = (21+24)*21.43 = 964.35J

\question{5}{Instruction Set Architecture}

\begin{enumerate}
\item LOAD R5, 6000(R0)\\
R5 = Mem[6000+R0]\\
R5 = Mem[6000+1000]\\
R5 = Mem[7000]\\
R5 = 1\\
\item ADD R4,(R4)\\
R4 = R4 + Mem[R4]\\
R4 = 6000 + Mem[6000]\\
R4 = 6000 + 12\\
R4 = 6012\\
\item SUB R2,R1\\
R2 = R2 - R1\\
R2 = 99 - 25\\
R2 = 74\\
\item LOAD R6, @(R0)\\
R6 = Mem[Mem[R0]]\\
R6 = Mem[Mem[1000]]\\
R6 = Mem[3000]\\
R6 = 33\\
\item ADD R6, R4\\
R6 = R6 + R4\\
R6 = 33 + 6012\\
R6 = 6045\\
\item SUB R5, R6\\
R5 = R5 - R6\\
R5 = 1 - 6045\\
R5 = -6044\\
\item ADD R2, R5\\
R2 = R2 + R5\\
R2 = 74 + (-6044)\\
R2 = -5970\\
\item ADD R2, (R3+R0)\\
R2 = R2 + Mem[R3 + R0]\\
R2 = -5970 + Mem[4000 + 1000]\\
R2 = -5970 + Mem[5000]\\
R2 = -5970 + 71\\
R2 = -5899\\

\end{enumerate}


\end{document}