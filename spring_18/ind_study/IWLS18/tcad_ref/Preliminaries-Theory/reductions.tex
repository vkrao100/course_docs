\documentclass{article}
\usepackage{amsmath}
\usepackage{graphicx}
\bibliographystyle{plain}

\begin{document}
Reduction of a polynomial $F$ $w.r.t.$ another polynomial $G$ is defined as follows:
\begin{equation}
F - \frac{lt(F)}{lt(G)} \cdot G = F + \frac{lt(F)}{lt(G)} \cdot G
\end{equation}
where $lt(F)$ and $lt(G)$ denote the leading terms of polynomials, $F$ and $G$, respectively. Note that - can be replaced with + as we are performing modulo 2 sum.\\
Of course, this equation holds only if $lt(G)$ divides $lt(F)$. As we are working on polynomials $modulo$ 2, the coefficients in the polynomials are either 1 or 0. Therefore, the above expression can be written as,
\begin{equation}
F - \frac{lm(F)}{lm(G)} \cdot G = F + \frac{lm(F)}{lm(G)} \cdot G
\end{equation}
where $lm(F)$ and $lm(G)$ denote the leading monomials of polynomials, $F$ and $G$, respectively. \\
Consider, as an example, the polynomials $F$ and $G$ are,
\begin{equation}
F = f\cdot d + f + c
\end{equation}
\begin{equation}
G = f + b + a
\end{equation}
with monomial ordering $f > d > c > b > a$

\par We want to reduce $F$ $w.r.t.$ $G$. Redcution using equation 2 will require two steps, one for the term $f\cdot d$ and other for the term $f$ in $F$. This reduction can be completed in one step if we know all the terms in $F$ that have $f$ in them. Note that the leading monomial of $G$ will always be a single variable, as $G$ models a gate. Now consider the ZBDDs of $F$ and $G$ in Fig. 1 and 2 respectively. The ZBDDs represent the polynomials as a set of monomials (\{$f\cdot d, f,c$\} for $F$ and \{$f,b,a$ for $G$\}) appearing in them. The CUDD manager creates the ZBDDs with the defined monomial order, and therefore, the topmost node in both diagrams is $f$. Checking if $lm(G)$ divides $lm(F)$ becomes trivial as we just need to compare the indices of top-most nodes of $F$ and $G$, which in this case are equal. 
\par We want to perform reduction of $F$ $w.r.t.$ $G$ in one step. If we check the THEN branch of node $f$ in $F$, we will find that it represents the polynomial, $d + 1$. Therefore, the THEN branch of the top-most node of $F$ gives us all the terms that appear with $f$. So the reduction can be performed by multiplying $d + 1$ with $G$ and adding this product to $F$ $modulo$ 2,
\begin{align*}
& (f\cdot d + f + c) + (d + 1)\cdot(f + b + a) \\
&= 2\cdot(f\cdot d + f) + c + (d+1)\cdot(b + a) \\
&= c + (d+1)\cdot(b + a) 
\end{align*}
Consider the follwing terminologies,
\begin{align*}
& \text{$head(F)$ = THEN branch of top-most node of $F$} = d + 1\\
& \text{$tail(F)$ = ELSE branch of top-most node of $F$} = c\\
& \text{$tail(G)$ = ELSE branch of top-most node of $G$} = b + a
\end{align*}
The last step of reduction process can be written as,
\begin{align*}
&= c + (d+1)\cdot(b + a) \\
& tail(F) + head(f)\cdot tail(G)
\end{align*}
The data structure for a ZBDD node has two pointers for the THEN child and ELSE child, respectively. Therefore, $head(F)$, $tail(F)$, and $tail(G)$ can be acquired by just accessing the respective pointers. So the reduction process effectively involves two operations, a modulo 2 sum (SUM) and a product (PROD). The CUDD package provides a function for computing PROD. The SUM operation of two polynomials, $F$ and $G$, can be performed as follows,
\begin{align*}
SUM(F,G) = (F \cup G) - (F \cap G) 
\end{align*} 
where $\cup$, $\cap$, and $-$ represents set union, set intersection, and set difference respectively. The functions  for performing these three operations are present in the CUDD package.


\end{document} 