\section{Introduction}

Automated formal verification and equivalence checking of arithmetic
datapath circuits is challenging. Conventional verification
techniques, such as those based on binary  decision diagrams (BDDs)
\cite{BRYA86}, And-Invert-Graph (AIG) based reductions with SAT or
SMT-solvers \cite{alanmi:cec:iccad2006}, etc., are infeasible in
verifying complex datapath designs. Such designs often implement
algebraic computations over bit-vector operands, therefore finite
integer rings \cite{wienand:cav08} \cite{wedler:date11} or finite
fields \cite{lv:tcad2013} \cite{pruss:tcad}
are considered appropriate models to devise decision
procedures for verification. For this reason, the verification
community has explored the use of algebraic geometry and symbolic
algebra algorithms for verification. 


In such a setting, the logic gates of the circuit are modeled by way
of a set of multivariate polynomials $F=\{f_1,\dots,f_s\}$ in rings
$R[x_1,\dots,x_n]$. Usually, the coefficients $R = \Z, \Zkk,$ or $\Fkk$,
depending on whether the integer, finite integer ring
$\pmod{2^k}$, or respectively the finite field (of $2^k$ elements)
model is employed for verification. This set of polynomials $F$
generates an ideal, and for verification it is required to compute a
{\it \Grobner basis (GB)} \cite{gb_book} of this ideal. Reducing the
primary \textcolor{red}{output polynomials} of the circuit modulo this
GB  results in a unique canonical polynomial expression, and it can be
used for equivalence checking.  


The GB problem exhibits high computational complexity. 
Indeed, computing a GB (using Buchberger's \cite{buchberger_thesis} or
the $F_4$ algorithm \cite{f4}) for large circuits is practically
infeasible. Managing this complexity ought to be a major goal of any
approach. 

\subsubsection{State-of-the-art \& Limitations} Recent
approaches \cite{wienand:cav08} \cite{lv:tcad2013} have discovered that
particularly for circuit verification problems, the
expensive GB computation can be avoided altogether. For arbitrary
combinational \cite{wienand:cav08} \cite{lv:tcad2013} and sequential
circuits \cite{xiaojun:hldvt2016}, a specialized term order 
$>$ can be derived by analyzing the topology of the given 
circuit. This term order is derived by performing a reverse
topological traversal of the circuit, and in this manuscript we refer 
to it as the {\it Reverse Topological Term Order} (RTTO). 
Imposition of RTTO $>$ on the polynomial ring
{\it renders the set of polynomials of the circuit itself a
GB.} Subsequently, the verification problems can be solved solely by
way of GB-reduction (using multi-variate polynomial division), without
any need to explicitly compute a GB. It has now become standard
practice to make use of RTTO-style term orders to solve various formal 
verification problems on digital circuits 
(see for example
\cite{wienand:cav08,lv:tcad2013,wedler:date11,ciesielski:dac2015,farimah:micro2015,rolf:date16}), 
where the early techniques of \cite{wienand:cav08} \cite{lv:tcad2013}
have been extended and improved to verify integer and floating point
arithmetic circuits \cite{ciesielski:dac2015,rolf:date16,rolf:fmcad17}. 


%% For instance,
%% \cite{ciesielski:dac2015} and \cite{rolf:date16} get more insights
%% from the circuit structure that  dictate specific rules on the order of
%% polynomials chosen in GB-reduction -- by accounting for topological
%% levels, reconvergent fanouts, AND-XOR gates with common inputs,
%% etc. 

A common theme among all these relevant works is that {\it by virtue
of RTTO, they move the complexity of verification from one of
computing a GB to that of GB-reduction by way of multivariate
polynomial division.} Moreover, since the \Grobner basis is derived
from the logic gates of the circuit, {\it it comprises Boolean
polynomials.} \textcolor{red}{Boolean polynomials (formally defined in
  Section \ref{sec:prelim})  consist of terms that have coefficients
  from $\Ftwo = \{0,1\}$, and monomials that are a product of
  variables where the degree of each variable is also restricted to
  the set $\{0,1\}$.} For example, in the verification of finite field 
circuits, the overall problem is modeled over the ring
$\Fkk[x_1,\dots,x_n]$; and %, where $\Fkk$ is the field of $2^k$ elements. 
since $\Fkk \supset \Ftwo$, the approaches incorporate computations
over many Boolean polynomials. Similarly, the techniques of
\cite{wedler:date11} use a pseudo-Boolean polynomial model which also
encompasses computations $\pmod{x^2=x}$. \textcolor{red}{Pseudo-Boolean
  polynomials %(defined later    in Section \ref{sec:prelim}) 
are those that have coefficients in integers, and monomials are a
product of variables, where the degree of each variable is still
restricted to the set $\{0,1\}$.}  The aforementioned approaches will
benefit greatly by a {\it dedicated,  
domain-specific implementation of GB-reduction w.r.t. Boolean polynomials,
carried out on the given circuit under RTTO $>$}. So far, the above
techniques  
\cite{wienand:cav08,pruss:tcad,lv:tcad2013,rolf:date16,ciesielski:dac2015,cunxi:aspdac17}
use a general-purpose polynomial division approach, together with an
explicit representation, for this GB-reduction. 
%While some of these
%approaches %\cite{ciesielski:dac2015,rolf:date16} 
%do perform the reduction in some specific ways -- e.g., mimicking
%GB-reduction under RTTO $>$ by substitution \cite{rolf:date16}, or using
%TEDs to perform input-output signature comparisons \cite{ciesielski:dac2015},
%or the use of $F_4$-style GB-reduction on a coefficient matrix
%\cite{lv:tcad2013} -- 
Moreover, the overall concept of polynomial division is
still utilized in its rudimentary form, involving iterative
cancellation of monomials ``1-step at a time'' on explicit
data-structures. Despite recent efforts, such GB-reductions can lead
to {\it a worst-case size explosion problem}, which needs to be
addressed. 


\subsubsection{Objective \& Rationale} This paper addresses the
problem of deriving canonical representations for datapath circuits
as Boolean polynomials. These canonical Boolean polynomials are
used for equivalence checking of datapath designs. The canonical
representation requires a \Grobner basis reduction modulo a set of
Boolean polynomials. 
%This reduction can result in space and time
%explosion and make verification infeasible. 
To make this GB-reduction
on circuits more efficient, this paper describes new techniques,
algorithms and implementations, specifically targeted for 
circuit verification under RTTO $>$.  We make use of
the {\it implicit} characteristic set representation for storing and
manipulating Boolean polynomials using {\it Zero-Suppressed BDDs
(ZBDDs)} \cite{zbdd}. By analyzing the {structure of ZBDDs} for 
polynomial representation under RTTO $>$, we show how this 
GB-reduction can be efficiently implemented using algorithms that
specifically manipulate the ZBDD graph, by interpreting Boolean
polynomial manipulation as the {\it algebra of unate cube sets.}

The algebraic objects
used to model the polynomial ideals derived from digital circuits are
rings of Boolean polynomials.
%generally of two types: (i) multi-variate polynomial
%rings with coefficients from the finite fields ($\Fkk$) of $2^k$
%elements, i.e. $\Fkk[x_1, \dots, x_n]$; and (ii)  quotient rings of the
%type $\Zkk[x_1, \dots, x_n] \pmod{ \langle x_i^2 - x_i\rangle}$
%corresponding to the pseudo-Boolean function models; here 
%$\Zkk = \Z \pmod{ 2^k}$. Over circuits, both models incorporate
%computations over {\it Boolean   polynomials}: (i) $\Ftwo (\equiv \B)
%\subset \Fkk$; whereas (ii) also encompasses (pseudo-) Boolean
%computations $\pmod{ x^2 = x}$.
%; \cite{wienand:cav08} refer to it as ``arithmetic bit-level''. 
When Boolean functions are represented in $\F_2$ (AND/XOR
expressions), and that too as a canonical \Grobner basis, the
representation tends to explode. 
%(depicted later in Section \ref{sec:motiv}).  
Polynomial representations employed in computer
algebra tools, such as the {\it dense-distributive 
data-structure} of the {\sc Singular} computer algebra tool
\cite{DGPS}, are inefficient for this purpose. Since addition and
multiplication $\pmod{2}$ are equivalent to XOR and AND operations,
respectively, GB-reduction can be viewed as a polynomial analog of a
specialized {\it AND/XOR Boolean decomposition} problem. Moreover,
the monomials of a Boolean polynomial are a product of literals in
positive polarity, which can be viewed as {\it unate cubes} in logic
synthesis. Clearly, implicit Boolean set representations such as
decision diagrams could be employed for \Grobner basis reductions over
Boolean polynomials. 
%The decision diagram of choice here is the ZBDD,
%\cite{zbdd}, 
%because of its power to represent and manipulate sparse
%combinatorial problems -- particularly ``sets of combinations'' --
%using the unate cube set algebra framework \cite{zbdd_unate}.  
%Monomials of a Boolean polynomial can be
%interpreted as {\it unate cubes} -- i.e. products of literals in
%positive polarity. Each cube represents one combination, and each
%literal represents an object chosen in the combination. Thus,
%GB-reduction on circuits resembles a classical logic synthesis
%problem, justifying the use of ZBDDs.


\subsubsection{Technical Contributions} %First and foremost, 
%% We address the problem of \Grobner basis reductions over Boolean
%% polynomials to derive canonical representations for the functions
%% implemented by datapath circuits. Our approach makes use of 


%% For this we present new algorithms and

We first describe when and how the GB-reduction encounters a
term-explosion (exponential blow-up) under RTTO $>$, which cannot be
easily overcome by explicit representations. We show that ZBDDs can
avoid this exponential blow-up -- thereby justifying their use. We
describe how Boolean polynomials can be represented using ZBDDs under
the special term ordering constraint imposed due to RTTO $>$. The
implementation of classical polynomial division algorithms that
iteratively cancel one monomial in every step is described on ZBDDs.
Subsequently, {\it we show that RTTO $>$ imposes 
a special structure on ZBDDs that allows to implement reduction
techniques that implicitly cancel multiple   monomials in every step
of polynomial division}. Moreover, due to RTTO $>$,  the
subexpressions that are required for polynomial division are also
readily available as subgraphs in the ZBDDs. Our algorithm exploits
this special structure, thus improving GB-reduction in both space and
time. 
%Moreover, we also implement the specialized
%GB-reduction approach of \cite{rolf:date16} using our ZBDD
%framework. 

Using an implementation integrated with the CUDD \cite{cudd} package,
we perform extensive experiments on datapath circuits for deriving the
canonical representation of the functions implemented by them. The
benchmark designs include various cryptography primitives, such as
finite field multipliers, elliptic curve point addition circuits, and
also  sequential finite field circuits. Experiments conducted on these
benchmarks show {\it orders of magnitude improvement} using our
implementation of GB-reduction, as compared against contemporary
methods. In fact, for these benchmarks, our bit-level (Boolean)
approach is much faster than the word-level approaches
(e.g. \cite{pruss:tcad}). 

We also describe the limitations of our approach when it is
applied to integer arithmetic circuits. We analyze these limitations
and show that for integer arithmetic circuits, a word-level symbolic
reasoning engine is needed to control the monomial explosion
problem. This is not a limitation of our 
algorithms, but rather of the bit-level model in verifying integer
arithmetic circuits. Finally, we are not concerned with equivalence
checking of random logic circuits; AIG-based reductions for
SAT-based verification techniques \cite{alanmi:cec:iccad2006} are more
suitable and efficient for such applications. 



\subsubsection{Paper Organization} The following section reviews
relevant previous work on \Grobner basis based verification of
datapath circuits, and the literature on Boolean \Grobner basis and
applications. Section \ref{sec:prelim} describes the mathematical
background on \Grobner basis reductions, the RTTO based term order
$>$, and how these concepts are applied to datapath
verification. Section \ref{sec:unate} motivates how and why the
Boolean GB-reduction can be viewed as the unate cube set
algebra. Section \ref{sec:theory} describes the theory, algorithms and
implementations for GB-reduction for Boolean polynomials using
ZBDDs. Section \ref{sec:exp} describes the experiments conducted for
verification using our approach, and Sec \ref{sec:conc} concludes the
paper. 


\section{Related Previous Work}
\label{sec:prev}

In the past decade, computer algebra and algebraic geometry based
datapath verification has received a lot of attention, where the
verification problems are formulated in terms of ideal membership,
canonical GB-reductions, or projections of
varieties. The work of \cite{iccad05} used a GB-reduction
approach to derive canonical representations of (word-level) RTL
datapath descriptions over finite rings $\Zkk$. Using the same finite
ring model, \cite{wienand:cav08} addressed data correctness properties
of arithmetic bit-level implementations using a \Grobner basis
formulation. This paper showed how an efficient term order $>$
can be derived from the circuit to simplify the computation. 


In
\cite{lv:date2012} and \cite{lv:tcad2013}, the authors addressed formal
verification of finite field arithmetic circuits using the Strong
Nullstellensatz formulation over $\Fkk$. Using a set ($F$) of
polynomials to describe the logic circuit, along with a set of
vanishing polynomials ($F_0$) over the field $\Fkk$, the verification
problem was formulated as a (radical) ideal membership test, requiring 
a \Grobner basis. Drawing inspirations from \cite{wienand:cav08},
the authors in \cite{lv:date2012} also exploited the same concept of deriving a
specialized term order $>$ to simplify the ideal membership test. In
particular, it was shown that $>$ could be derived by performing a
{\it reverse topological traversal} on the circuit. Imposition of this
term order $>$ rendered the set of polynomials $F\cup F_0$ itself a
\Grobner basis, and verification was then performed simply by a
GB-reduction. This GB-reduction was 
subsequently formulated as Gaussian elimination on a coefficient
matrix \cite{lv:tcad2013,pruss:tcad} in the style of the $F_4$
algorithm; called $F_4$-style reduction in the sequel. 

Formulations of a similar flavor (GB-reduction under the
specialized term order $>$ derived from the circuit) were used
and integrated with SMT-solvers \cite{wedler:date11} for verification
using a pseudo-Boolean model (akin to $\Zkk[X] \pmod{X^2-X}$). More
recently, these concepts have been applied to verify integer
arithmetic \cite{ciesielski:dac2015,rolf:date16}, and also floating
point circuits \cite{rolf:fmcad17}. The authors
in~\cite{cunxi:aspdac17} show that the reduction process can be
{\it parallelized} by performing reduction for each output bit
independently.   

%The above approaches exploit the
%structure of the circuit to derive a term order $>$ that simplifies
%verification to a (set of) \Grobner basis reductions, of which
Polynomial division algorithms form the core computation in the above
techniques. Almost all of the 
aforementioned techniques use a classical polynomial division
approach. While some of these approaches
%\cite{ciesielski:dac2015,rolf:date16}  
do perform the reduction in 
some specific ways -- e.g., mimicking GB-reduction under RTTO $>$ by
substitution \cite{rolf:date16}, or using TEDs to perform input-output
signature comparisons \cite{ciesielski:flow}, or the use of
$F_4$-style GB-reduction on a coefficient matrix
\cite{lv:tcad2013,pruss:tcad} --  the overall concept of polynomial
division is still utilized in its rudimentary form, involving
iterative cancellation of monomials ``1-step at a time'' on explicit
data-structures. 
%In contrast, this paper improves the \Grobner basis
%reduction under RTTO $>$ using an implicit data-structure and
%algorithms that cancel multiple monomials 


\subsection{Boolean \Grobner Basis} 
The symbolic algebra community 
has studied properties of Boolean \Grobner bases 
\cite{michon:bool-ring2006} \cite{polybori:2009}. Boolean GB
formulations have also been used for SAT solving \cite{CEI:stoc-96},
e.g. to derive proof refutation, and for  model checking
\cite{Avrunin:CAV} \cite{vardi-iasted07}. From among these, the work
of PolyBori \cite{polybori:2009} comes closest to ours, and is a
source of inspiration for this work. PolyBori proposed the use of
ZBDDs to compute Gr\"obner bases for Boolean polynomials. PolyBori is
a {\it generic} Boolean GB computational engine that caters to many
permissible term orders. Its division algorithm is also generally
based on the conventional concept of canceling one monomial in every
step of reduction. In contrast, our algorithms are tailored for
GB-reduction under the RTTO $>$. The efficiency of our approach stems
from the observation that the RTTO $>$ imposes a special structure on
the ZBDDs, which allows for multiple monomials to be canceled in one
division-step, along with simplifying the search for
divisors. Experiments show that our approach is an order of magnitude
faster than PolyBori. 


%% \underline{\it Paper Organization:} The following section describes
%% our notation and preliminary concepts. Section \ref{sec:limit}
%% describes the limitations of contemporary approaches and shows how
%% ZBDDs may overcome these. Section \ref{sec:alg} describes our
%% algorithms and implementations of GB-reduction on ZBDDs w.r.t the RTTO
%% $>$. Section \ref{sec:integer} describes how the recent methodologies
%% for integer multiplier circuits can be implemented and improved
%% computationally using ZBDDs. Experiments are described in Section
%% \ref{sec:exp}. Section \ref{sec:concl} concludes the paper. 
