%%%%%%%%%%%%%%%%%%%%%%%%%%%%%%  IEEEsample.tex
%%%%%%%%%%%%%%%%%%%%%%%%%%%%%%%%%%%%%%%%%
%%%%%%%%%%%%%%%%%%%%%%%    More information: see the header of IEEEtran.sty
%%%%%%%%%%%%%%%%%%%%%%%
%%%%%%%%%%%%%%%%%%%%%%%%%%%%%%%%%%%%%%%%%%%%%%%%%%%%%%%%%%%%%%%%%%%%%%%%%%%%%%%%
%%%%
%%%%\documentclass[prodmode,acmtecs]{acmsmall}

\documentclass[conference, onecolumn]{IEEEtran}

%\usepackage[ruled]{./algorithm2e}
%%for algorithm2e package, label has to be following caption in the same line!!!
%% \renewcommand{\algorithmcfname}{ALGORITHM}
%% \SetAlFnt{\small}
%% \SetAlCapFnt{\small}
%% \SetAlCapNameFnt{\small}
%% \SetAlCapHSkip{0pt}
%% \IncMargin{-\parindent}

% Metadata Information
%\acmVolume{X}
%\acmNumber{X}
%\acmArticle{39}
%\acmYear{2010}
%\acmMonth{3}


\usepackage{helvet}
\usepackage{enumerate}
\usepackage{amsmath}
\usepackage{amsfonts}
\usepackage{graphicx}
\usepackage{multirow}
\usepackage{subfig}
\usepackage{comment}
\usepackage{mathtools}
%\usepackage{algorithm}
%%indent in algorithm


%\setcounter{page}{0}
\pagestyle{empty}
%\thispagestyle{empty}


% New command for the table notes.
\def\tabnote#1{{\small{#1}}}

% New command for the line spacing.
\newcommand{\ls}[1]
    {\dimen0=\fontdimen6\the\font
     \lineskip=#1\dimen0
     \advance\lineskip.5\fontdimen5\the\font
     \advance\lineskip-\dimen0
     \lineskiplimit=.9\lineskip
     \baselineskip=\lineskip
     \advance\baselineskip\dimen0
     \normallineskip\lineskip
     \normallineskiplimit\lineskiplimit
     \normalbaselineskip\baselineskip
     \ignorespaces
    }




%the following is for space before and after align or other equation environment.

%%
%\newtheorem{Algorithm}{Algorithm}[section]

\newtheorem{Algorithm}{Algorithm}[section]
\newtheorem{Definition}{Definition}[section]
\newtheorem{Example}{Example}[section]
\newtheorem{Proposition}{Proposition}[section]
\newtheorem{Lemma}{Lemma}[section]
\newtheorem{Theorem}{Theorem}[section]
\newtheorem{Corollary}{Corollary}[section]


\newcommand{\Fq}{{\mathbb{F}}_{q}}
\newcommand{\Fkk}{{\mathbb{F}}_{2^k}}
\newcommand{\Fkkx}[1][x]{\ensuremath{\mathbb{F}}_{2^k}[#1]\xspace}
\newcommand{\Grobner}{Gr\"{o}bner\xspace}
%%%

\newcommand{\debug}[1]{\textcolor{gray}{[ #1 ]}}



%%set spacing between table columns
\setlength{\tabcolsep}{3pt}

%% for larger matrices
\setcounter{MaxMatrixCols}{15}
\begin{document}


% Page heads
%\markboth{J. Lv et al.}{Formal Verification of Finite Field Arithmetic Circuits using Computer Algebra Techniques}


%% \title{\large{\textsc{Formal Verification of Galois Field Arithmetic
%%       Circuits using Computer Algebra Techniques }}}  

\title{\Large \sc 
Boolean Gr\"obner Basis Reductions on Datapath Circuits using the Unate Cube Set Algebra
}

\author{

\IEEEauthorblockN{Utkarsh Gupta,
Priyank Kalla,
Vikas K. Rao}

\ \\
%\IEEEauthorblockA{
\IEEEauthorblockA{Electrical and Computer Engineering\\
University of Utah, Salt Lake City, UT 84112\\
utkarsh.gupta@utah.edu, kalla@ece.utah.edu, vikas.k.rao@utah.edu\\
 }
}

\maketitle
%%%%%%%%%%%%%%%%%%%% abstract %%%%%%%%%%%%%%%%%%%%%
\begin{center}
{\bf Designated Contact Author: Priyank Kalla}\\
Email: kalla@ece.utah.edu\\
\end{center}

\ \\
\ \\
{\bf Prior Publication of this work:} This paper has not been
published in any archival conference proceedings, not in any
journals. The work is being submitted only to IEEE TCAD. 




%%%%%%%%%%%%%%%%%%%% Include your files here %%%%%%%%%%%%%%%%%%%%%


%%%%%%%%%%%%%%%%%%%% The bibliography %%%%%%%%%%%%%%%%%%%%%%%%%%%%
\bibliographystyle{IEEEtran}
\bibliography{logic}

\end{document}

%%%%%%%%%%%%%%%%%%%%%%%%%%%  End of IEEEsample.tex  %%%%%%%%%%%%%%%%%%%%%%%%%%%
