\begin{abstract}

  Resolving an unknown component is a fundamental problem encountered
  in logic synthesis for engineering change orders, post-verification
  debugging and automatic correction of digital circuits. Contemporary
  techniques rely on the iterative/incremental application of SAT solving
  and Craig interpolation to realize the functionality of (or resolve)
  the unknown components. While these techniques have achieved some
  success for control-dominated applications (random logic circuits),
  they are infeasible in resolving the unknown components in
  arithmetic circuits. This paper describes an algebraic approach to
  resolve the functionality of an unknown component in an arithmetic
  circuit so that the circuit implementation matches a given
  specification. Our approach is formulated as a polynomial ideal
  membership test. We go on to pose the problem as a synthesis
  challenge and explore the solution space of the unknown component
  using concepts from the quotient of ideals. We propose a \Grobner basis 
  based algorithm for a systematic, goal driven search for
  implementable solutions. The paper presents results on some
  experiments performed over various finite field arithmetic circuits
  to compare the efficiency of our approach against recent methods.   


% Automatic bug correction is a tedious and resource intensive process. 
%% Automatic correction of unknown components in a given circuit is a
%% resource intensive process. Recent developments in realizing the
%% functionality implemented by these unknown gates rely on incremental
%% SAT solving. Despite using state-of-the-art SAT solvers, these
%% approaches fail to verify multipliers beyond 12-bits and hence are
%% infeasible in a practical setting. The current formal datapath
%% verification methods which utilize symbolic computer algebra concepts,
%% rely heavily on textbook structure of the circuits to realize an
%% unknown component, and hence are not scalable. These approaches model
%% circuit as a set of polynomials over integer rings, and use function
%% extraction, simulation, and term rewriting using coefficient
%% computation to arrive at a solution. The approach is not complete in
%% the sense that the procedure cannot be extended to random logic
%% circuits and finite field circuits due to ambiguities in coefficient
%% computation. The approach also fails to verify circuits when redundant
%% gates are introduced in the design. To overcome all these limitations,
%% this paper describes a formal approach using finite field theory to
%% automatically realize the function implemented by an unknown
%% component, and verify the same. The paper introduces theory on
%% resolving a single unknown component using ideal membership testing
%% and \Grobner basis based reduction. We go onto pose the problem as a
%% synthesis challenge and extend the solution space of the unknown
%% component using concepts from quotient of ideals. Since the solution
%% space is not unique, we will also discuss a systematic, goal driven
%% search for simple implementable solutions. The paper presents results
%% on some preliminary experiments performed over various arithmetic
%% circuits to compare efficiency of our approach against recent methods.  
\end{abstract}
