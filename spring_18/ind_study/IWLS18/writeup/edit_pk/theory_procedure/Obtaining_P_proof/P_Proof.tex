% \documentclass{article}
\documentclass[twocolumn]{IEEEtran}
% \documentclass{llncs}

% Proof already defined in llncs so I am 
% relaxing it for amsthm
% \let\proof\relax
% \let\endproof\relax

\usepackage{hyperref}

\usepackage{algorithm}
\usepackage[noend]{algpseudocode}

\RequirePackage{amssymb, mathptm}
\usepackage{amsbsy}
\usepackage{amsthm}
\usepackage{graphicx}
\usepackage{helvet}
\usepackage{enumerate}
\usepackage{amsmath}
\usepackage{amsfonts}
\usepackage{graphicx}
\usepackage{multirow}
\usepackage{subfig}
\usepackage{comment}
\usepackage{cases}
\usepackage{xcolor}
\usepackage{epstopdf}
\usepackage[normalem]{ulem}
\usepackage{diagbox}
\usepackage{bm}
\usepackage{hhline}
\usepackage{caption}


% \bibliographystyle{plain}
\newcommand{\B}{{\mathbb{B}}}
\newcommand{\Z}{{\mathbb{Z}}}
\newcommand{\R}{{\mathbb{R}}}
\newcommand{\Q}{{\mathbb{Q}}}
\newcommand{\N}{{\mathbb{N}}}
\newcommand{\C}{{\mathbb{C}}}
\newcommand{\Zn}{{\mathbb{Z}}_{n}}
\newcommand{\Zp}{{\mathbb{Z}}_{p}}
\newcommand{\F}{{\mathbb{F}}}
\newcommand{\FF}{{\mathcal{F}}}
\newcommand{\Fbar}{{\overline{\mathbb{F}}}}
\newcommand{\Fq}{{\mathbb{F}}_{q}}
\newcommand{\Fqbar}{{\overline{{\mathbb{F}}_q}}}
\newcommand{\Fkk}{{\mathbb{F}}_{2^k}}
\newcommand{\Zkk}{{\mathbb{Z}}_{2^k}}
\newcommand{\Fkkx}[1][x]{\ensuremath{\mathbb{F}}_{2^k}[#1]\xspace}
\newcommand{\Grobner}{Gr\"{o}bner\xspace}
\newcommand{\bi}{\begin{itemize}}
\newcommand{\ei}{\end{itemize}}

\newcommand{\idealj}{{J = \langle f_1, \dots, f_s\rangle}}
\newcommand{\idealg}{{J = \langle g_1, \dots, g_t\rangle}}
\newcommand{\vfqj}{{V_{\Fq}(J)}}
\newcommand{\vfkkj}{{V_{\Fkk}(J)}}

%%% Added by Utkarsh %%%
\newcommand{\Va}{{V_A}}
\newcommand{\Vb}{{V_B}}
\newcommand{\Vc}{{V_C}}
\newcommand{\Vbc}{{V_{B,C}}}
\newcommand{\Vabc}{{V_{A,B,C}}}
\newcommand{\Vac}{{V_{A,C}}}
\newcommand{\w}{\wedge}

%%%%%%%%%%%%%%%%%%%%%%%%

\newcommand{\blu}{\color{blue}}
% \newcommand{\Grobner}{Gr\"{o}bner\xspace}
\newcommand{\fqring}{\Fq[x_1,\dots,x_n]}


% \newtheoremstyle{mystyle}
%   {\topsep}   % ABOVESPACE
%   {\topsep}   % BELOWSPACE
%   {\itshape}  % BODYFONT
%   {0pt}       % INDENT (empty value is the same as 0pt)
%   {\bf} % HEADFONT
%   {.}         % HEADPUNCT
%   {5pt plus 1pt minus 1pt} % HEADSPACE
%   {}          % CUSTOM-HEAD-SPEC

% \theoremstyle{plain}
\theoremstyle{definition}
% \theoremstyle{remark}
% \theoremstyle{mystyle}

\newtheorem{Algorithm}{Algorithm}[section]

\newtheorem{Definition}{Definition}[section]
% \numberwithin{Definition}{section}

\newtheorem{Example}{Example}[section]
% \numberwithin{Example}{section}

\newtheorem{Proposition}{Proposition}[section]

\newtheorem{Lemma}{Lemma}[section]
% \numberwithin{Lemma}{section}

\newtheorem{Theorem}{Theorem}[section]
% \numberwithin{Theorem}{section}

\newtheorem{Corollary}{Corollary}[section]
\newtheorem{Conjecture}{Conjecture}[section]
\newtheorem{Problem}{Problem}[section]
\newtheorem{Notation}{Notation}[section]
\newtheorem{Setup}{Problem Setup}[section]

\begin{document}

\title{Proof} 

\author{}

% \institute{}
% \maketitle

\thispagestyle{plain}
\pagestyle{plain}

Suppose the unknown component is the $i^{th}$ gate in the circuit with 
the leading term $x_i$. Let the RTTO order for the circuit be $x_1>x_2>\dots>x_n$.
Also assuming that the specification polynomial is $f_{spec}$, we can write
\begin{align}
\label{eqn:spec_red_fi_1}
f_{spec} &= h_1f_1 + h_2f_2 + \dots + h_i(x_i + P) + h_{i+1}f_{i+1} +\dots+ h_nf_n
\end{align}
Since we know $f_{spec}$ polynomial and $f_1$ polynomial, we can obtain the polynomial $h_1$
as the quotient of the reduction $f_{spec} \xrightarrow{f_1}_+$. The remainder of this
reduction can be reduced by $f_2$ to obtain $h_2$. Performing quotient computations in 
a similar fashion, we can obtain the polynomial $h_i$. Notice that we cannot obtain $h_{i+1}$ as the
tail or $P$ part of the polynomial $f_i$ is unknown. We can rearrange the terms in Eqn. \ref{eqn:spec_red_fi_1}
to obtain the following equation.
\begin{align}
\label{eqn:spec_red_fi_2}
f_{spec}-h_1f_1-h_2f_2-\dots-h_ix_i&= h_iP+h_{i+1}f_{i+1} +\dots+ h_nf_n
\end{align}
In the Eqn. \ref{eqn:spec_red_fi_2}, the L.H.S. is known. We can compute the polynomial
$P$ using the $lift$ operation in SINGULAR. The $lift$ operation returns a list of
polynomial coefficients $T_1\dots T_s$ for a list of polynomials $p_1,\dots,p_s$ such that 
the linear combination $T_1p_1+\cdots+T_sp_s$ is equal to some given polynomial $f$.
Using the $lift$ operation with inputs $f_{spec}-h_1f_1-h_2f_2-\dots-h_ix_i$ as $f$
and $h_i,f_{i+1},\dots,f_n$ as $p_1,\dots,p_s$, we can obtain the polynomial coefficients
$T_1\dots T_s$ such that $T_1h_i+\cdots+T_sf_n = f_{spec}-h_1f_1-h_2f_2-\dots-h_ix_i$.
Then the polynomial $T_1$ can be construed as the polynomial $P$ in our formulation.

\par The computed polynomial $P$, can contain any variable $i.e.$ it can contain 
the immediate input variables of the unknown component or it may even contain variables
that are not the immediate inputs. In other words, there is no guarantee about the 
variables that $P$ consists of. But we know from the circuit topology, that there
must be at least one polynomial $P$ only in the immediate input variables of gate $i$, that can 
be used in the polynomial $f_i = x_i+P$ to solve the problem. Let us denote this polynomial by $P'$.

\par Using the Eqn. \ref{eqn:spec_red_fi_2}, we can write,
\begin{align}
\label{eqn:spec_red_fi_3}
f_{spec}-h_1f_1-h_2f_2-\dots-h_ix_i&= h_iP'+h'_{i+1}f_{i+1} +\dots+ h'_nf_n
\end{align}
The polynomials $h'_{i+1}, \dots, h'_n$ in Eqn. \ref{eqn:spec_red_fi_3} need not be 
same as $h_{i+1}, \dots, h_n$ in Eqn. \ref{eqn:spec_red_fi_2} as their computation is dependent on
two different polynomials $P$ and $P'$ respectively. Subtracting the Eqn. \ref{eqn:spec_red_fi_2}
and Eqn. \ref{eqn:spec_red_fi_3}, we get,
\begin{align*}
h_i(P'-P) &= (h_{i+1}-h'_{i+1})f_{i+1} +\dots+ (h_n-h'_n)f_n \\
h_i(P'-P) &\in \langle f_{i+1},\dots,f_n \rangle \\
P'-P &\in \langle f_{i+1},\dots,f_n \rangle:h_i
\end{align*}
We denote the quotient (or colon) ideal $\langle f_{i+1},\dots,f_n \rangle:h_i$
by $C$. Therefore,
\begin{align}
\label{eqn:p'-p}
P'-P \in C
\end{align}
From remainder arithmetic and Eqn. \ref{eqn:p'-p}, we can say that the remainder of the reduction
$P\xrightarrow{GB(C)}_+$ and the remainder of the reduction $P'\xrightarrow{GB(C)}_+$ will be equal.
Let this remainder be $r$. 
\par Now consider a term order similar to RTTO where the variables $x_i$ (the 
output of the unknown component) and the immediate inputs of gate $i$ are placed at the end
in the lexicographic ordering. The $GB(C)$ with this order will contain the first few polynomials only in 
the variables $x_i$ and immediate inputs of gate $i$ (as this term order is an elimination term order).
So the reduction $P'\xrightarrow{GB(C)}_+r$ will produce $r$ consisting of only the immediate inputs
of gate $i$. The reason is that $P'$ is assumed to contain only the immediate inputs of gate $i$, and therefore,
it will only be reduced by the first few polynomials of GB(C) that contain the same variables as $P'$.
After computing $P$ (using the $lift$ operation) and $GB(C)$ with the new term order, we can 
perform the reduction $P\xrightarrow{C}_+$ to obtain $r$ which will only contain the immediate 
inputs of gate $i$ as variables. Therefore, $x_i+r$ can used a replacement for the unknown component.

 \par We can also pick a polynomial $p\in GB(C)$ such that $p$ contains only the immediate 
 inputs of gate $i$ and use $x_i +(r+p)$ as another replacement for the unknown component. This
 is due to the fact that $r+p \xrightarrow{GB(C)}_+r$.

\bibliographystyle{IEEEtran}

%\bibliography{cf}
% \bibliography{utkarsh,xiaojun,tim,logic,oldlogic,cnfsat,condrat_ms}
% \begin{thebibliography}{1}
% \bibliography{fmcad17}
% \end{thebibliography}

% \bibliography{utkarsh}
\end{document}
