\section{Conclusion and future work}
This paper has presented an approach based on \Grobner basis reductions and ideal
membership test to compute a function implemented by an unknown component in a circuit which models
a given specification. 
% We presented a procedure to systematically use \Grobner basis based reduction and ideal membership testing to arrive at a 
% solution, such that the resulting logic function of the circuit conforms to the reference specification. 
The paper also utilizes the concept of quotient of ideals to derive multiple solutions
for the unknown component. The experimental results demonstrate the efficiency of our approach 
for finite field arithmetic circuits where we achieve several orders of 
magnitude improvement as compared to recent SAT-based approach. 
We also present the theory for exploring the solution for the unknown 
component in terms of its immediate inputs.  
% The most desired solution which is in terms of immediate support 
% variables of the unknown component relies on expensive \Grobner basis re-computation 
% with a different term order. 
% As part of our future work, we are exploring heuristics 
% in order to arrive at a guided, simple implementable solutions set for $P$.
% The current set-up deals with one unknown 
% component or sub-circuit, 
As part of our future work, we are working on exploring the approach mentioned in this paper  
to integer arithmetic circuits which is currently restrained by the complicated division process for integer rings. We are also looking into
extending the current approach towards
multiple independent/dependent bugs in the design along
with identifying the potential locations where the circuit can be rectified.
% Also, identifying the bug location, 
% which is the primary concern in the overall scope of automated debugging 
% needs to be addressed as well. 