\section{Preliminaries: Notation and Background}
\label{sec:prelim}
This section reviews some basic concepts from symbolic computer
algebra that we will utilize in our theory. 
\par Let $\F_q$ denote the finite field of $q$ elements, where $q=p^k$ is some 
power ($k$ is a positive integer) of a prime number $p$. We
denote by $R = \F_q[x_1, \dots, x_n]$
the  polynomial ring over variables $x_1, \dots, x_n$ with
coefficients in $\F_q$. A polynomial $f \in R$ is written as a finite sum of terms 
$f = c_1 X_1 +  c_2 X_2 + \dots + c_t X_t$.  Here $c_1, \dots, c_t$
are coefficients and $X_1, \dots, X_t$ are monomials, i.e. power
products of the type $x_1^{e_{1}}\cdot x_2^{e_{2}}\cdots x_n^{e_{n}}$, 
$e_j \in \Z_{\geq  0}$. To systematically manipulate the
polynomials, a monomial order $>$ (also called a term order $e.g.$ lexicographical 
order) is
imposed on the polynomial ring such that  $X_1 >X_2 > \dots >  X_t$.  Subject to
$>$, $lt(f) = c_1 X_1, ~lm(f) = X_1, ~lc(f) = c_1$, are the {\it
leading   term}, {\it   leading monomial} and {\it   leading
coefficient} of $f$, respectively. Also, for a polynomial $f$,
$tail(f) = f - lt(f)$.

\par Gates of a circuit can be modeled with polynomials in $\F_2[x_1,\dots,x_n]$, 
where every Boolean logic gate operator is mapped to a polynomial
function over ${\mathbb{F}}_2$: 

% {\small
\begin{equation}
\label{bool2poly}
\begin{split}
z ~ =  ~ \neg a ~ \rightarrow ~ z+a+1 & \pmod 2  \\
z ~ =  ~ a \wedge b ~ \rightarrow ~ z+a \cdot b & \pmod 2\\
z ~ =  ~ a \vee b ~ \rightarrow ~ z+a+b+a \cdot b & \pmod 2 \\
z ~ =  ~ a \oplus b ~ \rightarrow ~ z+a+b & \pmod 2 
\end{split}
\end{equation}
% }

\par {\bf Polynomial Reduction via division:} Let $f, g$ be polynomials. If $lm(f)$ is divisible by
$lm(g)$, then we say that $f$ {\it is reducible to} $r$ modulo $g$,
denoted $f \stackrel{g}{\textstyle\longrightarrow} r$, where $r = f - {\frac{lt(f)}{lt(g)}} \cdot g$. Similarly, $f$ can be {\it reduced 
w.r.t. a set of polynomials}  $F = \{f_1, \dots, f_s\}$ to obtain a
remainder $r$. This reduction is denoted as $f \stackrel{F} {\textstyle
  \longrightarrow}_+ r$, and the remainder $r$ has the property that
no term in $r$ is divisible by the leading term of any polynomial
$f_j$ in $F$. Algorithm~\ref{algo:mv_reduce} (\cite{gb_book}) shows the 
step-by-step procedure to perform this reduction.  

\begin{algorithm}[H]
 \caption{Multivariate Reduction of $f$ by $F=\{f_1,\dots,f_s\}$}
 \label{algo:mv_reduce}
 \begin{algorithmic}[1]
 % \Procedure{$multi\_variate\_division$}{$f, f_1, \dots, f_s \in \F[x_1, \dots, x_n], f_i\neq 0$}
 \Procedure{$multi\_var\_division$}{$f,\{f_1,\dots,f_s\},f_j\neq0$}
 % \ENSURE $u_1,\dots, u_s, r$ s.t. $f = \sum f_i u_i+r$ where $r$ is
 % reduced w.r.t. $F = \{f_1,\dots, f_s\}$ and max($lp(u_1)lp(f_1), \dots, lp(u_s)lp(f_s), lp(r)$) = $lp(f)$
 \State $u_j \gets 0; ~r \gets 0, ~h \gets f $ 
 \While {  $h \neq 0$ }
 \If{ $\exists j$ s.t. $lm(f_j) ~|~ lm(h)$}
 \State choose $j$ least s.t. $lm(f_j) ~|~ lm(h)$
 \State $u_j = u_j + \frac{lt(h)}{lt(f_j)}$
 \State $h = h - \frac{lt(h)}{lt(f_j)} f_j$
 \Else
 \State $r = r+ lt(h)$
 \State $h = h - lt(h)$
 \EndIf
 \EndWhile
 \State \Return $(\{u_1,\dots,u_s\} , r)$
 \EndProcedure
 \end{algorithmic}
 \end{algorithm}

The algorithm initializes $h$ with the polynomial $f$ and cancels its leading term by some 
polynomial $f_j$. If the leading term $lt(h)$ cannot be canceled by any $lt(f_j)$, then it is added to the 
final remainder $r$ and process is repeated until all the terms in $h$ are analyzed. 
The algorithm also returns the set of quotients $\{u_1,\dots,u_s\}$ of division of $f$ by 
$\{f_1,\dots,f_s\}$, respectively. 

{\bf Polynomial Ideals:} Given a set of polynomials $F = \{f_1, 
\dots, f_s\}\in \mathbb{F}_q[x_1,\dots, x_n]$, denote the ideal
$J$ generated by $F$ as $J = \langle F \rangle = \langle f_1, \dots, f_s \rangle =
\{\sum_{j=1}^{s} h_j\cdot f_j: ~h_j \in R\}.$ The ideal $J$ may have
many different generators, i.e. it is possible to have 
$J = \langle f_1, \dots, f_s\rangle = \langle g_1, \dots, g_t \rangle
= \dots = \langle h_1,\dots, h_r\rangle$. A \Grobner basis (GB) $G$ of ideal
$J$ is one such set of polynomials $G = GB(J) = \{g_1, \dots, g_t\}$
that is a canonical representation of the ideal. 

\begin{Definition}
\label{def:gb}
$\bf{\left[Gr\ddot{o}bner\ Basis\right]}$~\cite{gb_book}: 
For a monomial ordering $>$, a set  of non-zero polynomials $G =
\{g_1,g_2,\cdots,g_t\}$ contained in an ideal $J$, is called a
\Grobner basis of $J$ iff 
$\forall f \in J$, $f\neq 0$, there exists $i \in \{1,\cdots, t\}$ such
that $lm(g_j)$ divides $lm(f)$.
% i.e., $G = GB(J) \Leftrightarrow\  \forall f \in J : f \neq 0 \ \exists g_i \in G :
%lm(g_i)\mid lm(f)$. 
\end{Definition}

The \Grobner basis for an ideal $J$ can be calculated using the 
Buchberger's algorithm (see Alg. 1.7.1 
in~\cite{gb_book}) which takes as input a set of polynomials $\{f_1, 
\dots, f_s\}$ and computes the GB $G = \{g_1,g_2,\cdots,g_t\}$.
% Buchberger's algorithm has a very high complexity and it is not practical
% to compute for large polynomial ideals.  
A GB can be {\it reduced} to eliminate
redundant polynomials from the basis. A reduced GB is a canonical
representation of the ideal. 

\begin{Lemma}
\label{lem:imt}
{\it Ideal Membership Testing(From Section 2.1 in~\cite{gb_book})}
Given $F = \{f_1,\dots,f_s\}$, let $G = \{g_1,g_2,\cdots,g_t\}$ be the GB for
$J = \langle f_1,\dots,f_s\rangle$ with respect to a fixed term ordering.
It is possible to determine if a polynomial $f \in \F_q[x_1,\dots$ $,x_n]$ is
in the ideal $J$, defined as Ideal Membership Testing by checking: 
$f \in J \iff f \xrightarrow{G}_+ 0$

By applying multivariate division, we can write:

$f = u_1g_1 + u_2g_2+ \dots+ u_tg_t;u_i\rightarrow$ quotients of division

While computing $G$, we can keep track of the quotients during reduction and express $G$ in terms of $F$.

\begin{small}
$\begin{bmatrix} g_1 \\ g_2 \\ \vdots \\ g_t \end{bmatrix} = M \cdot
    \begin{bmatrix} f_1 \\ f_2 \\ \vdots \\ f_s \end{bmatrix}$
\end{small} 
:$M$ is a $t\times s$ polynomial matrix.\\
Ideal membership test is implemented as procedure {\it lift} in SINGULAR~\cite{DGPS_410}. 
It takes a polynomial $f$ and ideal $J$ as input parameters and returns a set of polynomials $\{u_1,\dots, u_s\}$ such that $f = u_1f_1 + \cdots + u_sf_s$. 
\end{Lemma}

{\bf Ideal and Variety operations:}\\ 
Given two ideals $J_1 = \langle f_1,\dots,f_s\rangle, J_2=\langle
e_1,\dots,e_r\rangle$, their sum is given as $J_1 + J_2 = \langle
f_1,\dots,f_s,e_1\dots,e_r\rangle$, and their product is given as $J_1\cdot J_2 =
\langle f_j\cdot e_k: 1\leq j\leq s, 1\leq k\leq r\rangle$. 
\par For an ideal $J = \langle f_1, \dots, f_s \rangle$, the {\it variety} of
$J$ over $\F_q$ is denoted by $\vfqj$ and defined as, 
\begin{align*}
\vfqj = \{\mathbf{a} \in \F_q^n: \forall f \in J,f(\mathbf{a})=0\}
\end{align*}
Therefore, $\vfqj$ is the set of all the points in $\F_q^n$ that are
solutions to $f_1 = f_2 = \dots = f_s = 0$. Varieties can 
be different when restricted to the given field $\Fq$
or considered over its algebraic closure $\Fqbar$.
Ideals and varieties are dual concepts: $V(J_1 + J_2) = V(J_1) \cap V(J_2)$, and
$V(J_1\cdot J_2) = V(J_1) \cup V(J_2)$. Moreover, if $J_1 \subseteq
J_2$ then $V(J_1)\supseteq V(J_2)$.


\par For all elements $\alpha \in \Fq, \alpha^q = \alpha$. Therefore, the
polynomial $x^q-x$ vanishes (the polynomial evaluation is zero)
everywhere in $\Fq$, and is called the
vanishing polynomial of the field. We denote by $J_0 = \langle
x_1^q-x_1,\dots,x_n^q-x_n\rangle$ the ideal of all vanishing
polynomials in the ring $R$. Then $V_{\Fq}(J_0) = V_{\Fqbar}(J_0) =
\Fq^n$. Therefore, given any ideal $J$, $V_{\Fq}(J) = V_{\Fqbar}(J)
\cap\Fq^n = V_{\Fqbar}(J) \cap V_{\Fqbar}(J_0) = V_{\Fqbar}(J+J_0) =
V_{\Fq}(J+J_0)$ (\cite{gao:gf-gb-ms}).

\begin{Definition}
\label{def:rtto}
\par {\it Reverse Topological Term Order~\cite{lv:tcad2013}:}
The computational complexity of Buchberger's algorithm is exponential
in the number of variables $n$. As our work is focused on the circuits,
we will describe a term order that renders the set of polynomials for 
the gates of the circuit, a \Grobner basis itself. This term order 
is called Reverse Topological Term Order (RTTO).

\par Let $C$ be an arbitrary combinational
circuit. Let $\{x_1, \dots$ $, x_n\}$ denote the set of all variables
(signals) in $C$. Starting from the primary outputs, perform
a {\it reverse topological traversal} of the circuit and order the
variables such that $x_k > x_j$ if $x_k$ appears earlier in the
reverse topological order. Impose a lex term order $>$ to represent each
gate as a polynomial $f_j$, s.t. $f_j = x_k + tail(f_j)$. Then 
set of polynomials $\{f_1,\dots,f_s\}$ corresponding to the gates of the circuits 
is a \Grobner basis when RTTO is used for ordering.
\end{Definition}
% \par {\bf Weak Nullstellensatz and Elimination Theory:} 
\begin{Theorem}[{\it The Weak Nullstellensatz over finite fields (from
Theorem 3.3 in~\cite{gao:gf-gb-ms})}]
\label{thm:weak-ns-ff}
{\it For a finite field $\Fq$ and the ring $R = \Fq[x_1, \dots, x_n]$, let
$J = \langle f_1, \dots, f_s\rangle \subseteq R$, and let $J_0 = \langle
x_1^q-x_1, \dots, x_n^q -  x_n\rangle$ be the ideal of vanishing
polynomials. Then $V_{\Fq}(J) = \emptyset \iff 1 \in J + J_0 \iff G =
GB(J+J_0) = \{1\}$. }

\par To find whether a set of polynomials $f_1,\dots,f_s$ have no common
zeros in $\Fq$, we can compute the GB $G$ of
$\{f_1,\dots,f_s,x_1^q-x_1,\dots,x_n^q-x_n\}$ and see if $G = \{1\}$. 
\end{Theorem}

\begin{Definition}[{\it Elimination Ideal~\cite{ideals:book}}]
\label{def:elimideal}
Given an ideal $J = \langle f_1, \dots, f_s\rangle \subset \Fq[x_1,\dots,x_n]$, the $l$-th elimination
ideal $J_l$ is defined as $J_l = J \cap \Fq[x_{l+1},\dots,x_n]$.
\end{Definition}

The ideal $J_l$ is called an elimination ideal because the variables $x_1,\dots,x_{l-1}$
have been eliminated.
The next theorem shows how we can obtain the generators of the $l$-th
elimination ideal using \Grobner bases.

\begin{Theorem}[{\it Elimination Theorem~\cite{ideals:book}}]
\label{def:elim}
Given an ideal $J \subset R$ and its GB $G$ $w.r.t.$ the
lexicographical (lex) order on the variables 
where $x_1 > x_2 > \cdots > x_n$, then for every $0 \leq l \leq n$ we
denote by $G_l$ the GB of $l$-th elimination ideal of $J$ and compute it as:
\begin{center}
$G_l = G \cap \Fq[x_{l+1},\dots,x_n]$
\end{center}
\end{Theorem}

We also need to employ notion of difference of varieties in our theoretical
section. The equivalent ideal operation is called the quotient of ideals.

\begin{Definition}
\label{def:quo}
({\it Quotient of Ideals}) If $J_1$ and $J_2$ are ideals in a ring $R$,
then $J_1:J_2$ is the set 
%  \begin{equation}
  $\{f \in R \ |\ f\cdot g \in J_1, \forall g \in J_2\}$ %\nonumber
%  \end{equation}
and is called the {\bf ideal quotient} of $J_1$ by $J_2$.
\end{Definition}

In terms of varieties, $V_{\Fq}(J_1:J_2) = V_{\Fq}(J_1) \setminus V_{\Fq}(J_2)$.
The computation of elimination ideal is based on the intersection of ideals. 
For details on the computation of elimination ideal, we refer the reader to~\cite{ideals:book}. 

\par The computer algebra tools like SINGULAR~\cite{DGPS_410} contain implementations for 
computing elimination ideals and quotient of ideals.  
  