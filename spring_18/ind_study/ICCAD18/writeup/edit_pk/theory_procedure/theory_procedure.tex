\section{Theory and Procedure}
\label{sec:theory}
In this section we describe the procedure and algorithm used to arrive at a function implmentable for the unknown component and also show how we can explore the solution space. We will apply the algorithm on an example to substantiate our theory. 
\subsection{Reference specification is polynomial f}
Consider a specification polynomial $f$ and its circuit implementation $C$, modeled as polynomials $F = \{f_1,\dots,f_s\}\in \mathbb{F}_q[x_1,\dots, x_n]$. The generator of polynomials is given as $J=\langle F \rangle$, with $J_0=\langle x_l^q-x_l\rangle$ being the set of all vanishing polynomials. Let us consider RTTO (\autoref{def:rtto}) as the variable order for the circuit. We will assume $f_i:1\le i \le s$ to be the unknown component which is of the special form:
\begin{gather} 
\label{fiform}
f_i = x_i + P(X_u)
\end{gather}

where $x_i$ is the leading monomial, and $P$ is the tail representing desired solution in variables$:X_u \subset \{x_1,\dots,x_n\} \text{ s.t. } x_j \in X_u \text{ and } x_j<x_i$ in the order. 

As described in ~\autoref{lem:imt} for a correct implementation, specification $f$ should vanish on the variety of ideal generated by the circuit polynomials i.e., $f$ will be in the ideal generated by the circuit:

\begin{equation}
\label{member}
f \in J + J_0; 
f \in \langle f_1,..,f_s\rangle + \langle x_l^q-x_l\rangle;1\le l \le n
\end{equation}
% $f = h_1f_1 + h_2f_2 +\dots+h_if_i+\dots+h_sf_s+H\langle x_i^q-x_i\rangle$
Using Ideal membership testing, we can rewrite $f$ in terms of its generators as:

$f = h_1f_1 + h_2f_2 + \dots+h_if_i+\dots+h_sf_s +H\langle x_l^q-x_l\rangle$

where $H, h_m:1\le m \le s$ are arbitrary elements from $\Fq$.\\
From~\eqref{fiform}:

% $f = h_1f_1 + h_2f_2 +\dots+h_ix_i+h_iP+\dots+h_sf_s+H\langle x_i^q-x_i\rangle$
{\small$f = h_1f_1 + h_2f_2 +\dots+h_ix_i+h_iP(X_u)+\dots+h_sf_s+H\langle x_l^q-x_l\rangle$}

% Given the RTTO $>$, we know the polynomials from $f_s,\dots,f_{i+1}$ and the leading term of $f_i$. Using algorithm:(\ref{algo:mv_reduce}) to reduce 
% $f - h_1f_1 - h_2f_2 -\dots-h_ix_i = h_iP+\dots+h_sf_s+H\langle x_i^q-x_i\rangle$
$f - h_1f_1 -\dots-h_ix_i = h_iP(X_u)+\dots+h_sf_s+H\langle x_l^q-x_l\rangle$

% $f - h_1f_1 - h_2f_2 -\dots-h_ix_i \in \langle h_i,f_{i+1},\dots,f_s, x_i^q-x_i\rangle$
$f - h_1f_1 -\dots-h_ix_i \in \langle h_i,f_{i+1},\dots,f_s, x_l^q-x_l\rangle$\\
We shall call the intermediate remainder computed on the left hand side as $g$.
\begin{equation}
g \in \langle h_i,f_{i+1},\dots,f_s, x_l^q-x_l\rangle
\end{equation}
Given polynomials $h_i, g, f_{i+1},\dots,f_s$, we compute $P$ such that:

 $g = P*h_i+h_{i+1}f_{i+1}+\dots+h_sf_s+H\langle x_l^q-x_l\rangle$

The computed $P$ is a solution to the function implemented by the unknown gate. This linear combination computation is done using $lift$ (\autoref{lem:imt}) implementation in SINGULAR~\cite{DGPS_410}.

%add this as a lemma
We will also have cases, when $h_i$ ends up being a constant, in which case $lift$ returns $g$ itself as a solution $h_i^{'}$. To arrive at a implementable solution, we divide $h_i^{'}$ by the constant $h_i$(multiply the inverse of $h_i$) and reduce the result by rest of the input polynomials\{$f_{i-1},\dots,f_1$\}. 

\begin{align}
h_i^{'}*h_i^{-1}\xrightarrow[]{f_{i+1}}\xrightarrow[]{f_{i+2}}\dots\xrightarrow[]{f_s}P
\end{align}
Despite being a correct solution, the above approach doesn't guarantee the solution to be in the immediate support variables of $f_i$ due to RTTO$>$. To determine a solution in immediate support variable set $x_j$ of $f_i$, we use an elimination term order (\autoref{def:elim}) for the variables $x_k$ followed by $x_j$. We can then compute a $GB$ using this elimination term order with the intermediate solution $P$ added as tail of $f_i$. This $GB$ will have one and only one polynomial which is of the form $x_k + \mathcal{F}(x_j)$, where $\mathcal{F}$ is the function implemented by the gate, and is the most desired solution. 

Since we found two solutions, it is given that $P$ is not unique. We can explore more such solutions which might satisfy the unknown component functionality. Given $P$ as one of the solutions, under RTTO$>$ we have:

$g = P*h_i+h_{i+1}f_{i+1}+\dots+h_sf_s+H^{'}\langle x_l^q-x_l\rangle;$\\
Since $g$ can be computed as any linear combination of polynomials, we can rewrite the equation as:\\
$P*h_i+h_{i+1}f_{i+1}+\dots+h_sf_s+H^{'}\langle x_l^q-x_l\rangle = P^{1}*h_i+h_{i+1}^{'}f_{i+1}+\dots+h_s^{'}f_s+H^{'}\langle x_l^q-x_l\rangle$;\\
Rearranging the terms:

$(P-P^{1})h_i = (h_{i+1}-h_{i+1}^{'})f_{i+1}+\dots+(h_{s}-h_{s}^{'})f_s+(H-H^{'})x_l^q-x_l;$

$(P-P^{1})h_i \in \langle f_{i+1},\dots,f_s,x_l^q-x_l\rangle;$\\
By definition of Quotient of Ideals (\autoref{def:quo}):
\vspace{0.1in}
\begin{equation}
\label{quotcomp}
P-P^{1} \in \langle f_{i+1},\dots,f_s,x_l^q-x_l\rangle:h_i;
\end{equation}

There can be many $P^{*}$'s which might satisfy the above membership test. We can pick any polynomial from the quotient operation, add the previous solution $P$ and compute a new $P$'s. All such $P$'s computed are valid solutions and will satisfy the membership test with specification polynomial $f$. The following example for a redundant boolean logic implementing the majority function demonstrates the above procedure.

\begin{Example}
Consider the 2-bit Mastrovito multiplier given in fig.~\ref{mas_c} with variables from ring $\R=\F_2[a_0,b_0$ $,a_1,b_1,s_0,s_1,s_2,s_3,r_0,z_0,z_1,Z,A,B]$. Let us assume $f_2$ to be the unknown gate in the design which is of the form $f_2 = s_3 + P$.\\%\mathcal{F}(a_1,b_1)$.\\
\begin{figure}[ht]
	\begin{center}
	\includegraphics[scale = 0.65]{mas_c}
	\end{center}
	\vspace{-2ex}
	\caption{2-bit Mastrovito multiplier}
	\label{mas_c}
	\vspace{-1ex}
\end{figure}\\
% For the given circuit, we define \textit{cuts} across the gates based on heuristics such as dependency and levelization\cite{maciej:2016:1}. A $cut$ is defined as a set of signals that separates primary inputs from primary outputs. The prominence of these cuts is to maintain a variable order across cuts . For example at each cut $cut_m$ from the figure(\ref{tianka_ckt_c}), the following variable set has to be maintained across reductions.
% \begin{equation}
% \begin{split}
% cut_0 = \{a,b,c\} &\quad cut_3= \{z_1,z_2\}\\
% cut_1 = \{e_0,e_1,c,e_2\}  &\quad cut_4 = \{z\} \\
% cut_2 = \{e_0,d_0,e_2\}
% \end{split}
% \end{equation}
The 2x2 Mastrovito multiplier with specification $f: Z + A\cdot B$, is constructed as follows:
% \begin{lalign*}
% \begin{split}
\begin{enumerate}
    \item{Field construction: $\F_4 = \F_2[X]$ (mod $\mathcal{P}$); where $\mathcal{P} = X^2 + X + 1$ is the primitive polynomial used.}
    \item{$Z = z_0 + z_1*\al; A = a_0 + a_1*\al; B = b_0 + b_1*\al;$ are the word level polynomials, and $\al$ is the root of primitive polynomial s.t. $\mathcal{P}(\al)=0$.}
\end{enumerate}
Based on the circuit topology, RTTO$>$ with variable order:
$\{Z\}>\{A>B\}>\{z_0>z_1\}>\{r_0>s_0>s_3\}>\{s_1>s_2$ $\}>\{a_0>a_1>b_0>b_1\}$\\ 
Let $F$ be the set of all polynomials implementing the circuit which are given as:
% \begin{equation*}
% \begin{split}
% f_1:s_0 + a_0*b_0;  &  f_5:r_0 + s_1 + s_2; & f_8:A + a_0 + a_1*\al; \\
% f_2:s_3 + \mathcal{F}(a_1,b_1);  &  f_6:z_0 + s_0 + s_3; & f_9:B + b_0 + b_1*\al;\\
% f_3:s_2 + a_1*b_0;  &  f_7:z_1 + r_0 + s_3; & f_{10}:Z + z_0 + z_1*\al;\\
% f_4:s_1 + a_0*b_1;
% \end{split}
% \begin{equation}
{\small\begin{flalign*}
f_1:s_0 + a_0*b_0;  &\quad  f_5:r_0 + s_1 + s_2; & f_8:A + a_0 + a_1*\al;\\
% f_2:s_3 + \mathcal{F}(a_1,b_1);  &\quad  f_6:z_0 + s_0 + s_3; & f_9:B + b_0 + b_1*\al;\\
f_2:s_3 + P;  &\quad  f_6:z_0 + s_0 + s_3; & f_9:B + b_0 + b_1*\al;\\
f_3:s_2 + a_1*b_0;  &\quad  f_7:z_1 + r_0 + s_3; & f_{10}:Z + z_0 + z_1*\al;\\
f_4:s_1 + a_0*b_1;
\end{flalign*}}%
We shall add the ideal of vanishing polynomials $J_0$ for primary inputs, outputs, and intermediate variables.  
{\small\begin{flalign*}
f_{11}:a_0^2 + a_0; &\quad f_{15}:s_0^2 + s_0;\quad f_{19}:r_0^2 + r_0;\quad f_{23}:A^4 + A;\\
f_{12}:a_1^2 + a_1; &\quad f_{16}:s_1^2 + s_1;\quad f_{20}:z_0^2 + z_0;\quad f_{24}:B^4 + B;\\
f_{13}:b_0^2 + b_0; &\quad f_{17}:s_2^2 + s_2;\quad f_{21}:z_1^2 + z_1;\\
f_{14}:b_1^2 + b_1; &\quad f_{18}:s_3^2 + s_3;\quad f_{22}:Z^4 + Z;
\end{flalign*}}%
\begin{small}
$F = \{f_1,\dots,f_{10}\}; J = \langle F\rangle = \langle f_1,\dots,f_{10}\rangle; J_0 = \langle f_{11},\dots,f_{24}\rangle$
\end{small}
% Due to RTTO$>$, the set of polynomials ($J+J_0$) is in itself a \Grobner basis.\\

For a correct implementation, specification $f$ vanishes on circuit implementation:
% \begin{equation}
$f \in \langle f_1,f_2,f_3,\dots,f_{10}\rangle+\langle f_{11},f_{12}$ $,\dots,f_{24}\rangle$,
% \end{equation}
 where tail of $f_2$ is unknown.
% h_4f_4 \in \langle f,f_1,f_2,f_3,f_5,f_6,f_7\rangle\\
% h_4f_4 = f+h_1f_1+h_2f_2+h_3f_3+h_5f_5+h_6f_6+h_7f_7

We know that under RTTO$>$, the given set of circuit polynomials in itself form a $GB$. Hence to compute $g$, we start reducing the specification polynomial $f$ using polynomials from the set $\langle J + J_0\rangle$. We will use the following notations for reduction: '[]' to represent quotient-$h_i$'s, '()' to represent divisor-$f_i$'s, and '\{\}' to represent the partial remainder of every reduction step-$fp_i$'s.

\begin{small}
% \begin{split}
$f\xrightarrow[]{f_{10}}[1](Z + z_0 + z_1*\al)+\{ A*B+z_0+\al*z_1\}\rightarrow fp_1$

$fp_1\xrightarrow[]{f_8}[B](A+a_0+\al*a_1)+\{B*a_0+\al*B*a_1+z_0+\al*z_1\}\rightarrow fp_2$

$fp_2\xrightarrow[]{f_9}[a_0+\al*a_1](B+b_0+\al*b_1)+\{z_0+\al*z_1+a_0*b_0+\al*a_0*b_1+\al*a_1*b_0+(\al+1)*a_1*b_1\}\rightarrow fp_3$

$fp_3\xrightarrow[]{f_6}[z_0+s_0+s_3](1)+\{\al*z_1+s_0+s_3+a_0*b_0+\al*a_0*b_1+\al*a_1*b_0+(\al+1)*a_1*b_1\}\rightarrow fp_4$

$fp_4\xrightarrow[]{f_7}[z_1+r_0+s_3](\al)+\{\al*r_0+s_0+(\al+1)*s_3+a_0*b_0+\al*a_0*b_1+\al*a_1*b_0+(\al+1)*a_1*b_1\}\rightarrow fp_5$

$fp_5\xrightarrow[]{f_5}[r_0+s_1+s_2](\al)+\{s_0+(\al+1)*s_3+\al*s_1+\al*s_2+a_0*b_0+\al*a_0*b_1+\al*a_1*b_0+(\al+1)*a_1*b_1\}\rightarrow fp_6$

$fp_6\xrightarrow[]{f_1}[s_0+a_0*b_0](1)+\{(\al+1)*s_3+\al*s_1+\al*s_2+\al*a_0*b_1+\al*a_1*b_0+(\al+1)*a_1*b_1\}\rightarrow fp_7$

% $fp_6\quad\xrightarrow[]{lt(f_2)}[s_3](\underbrace{x+1}_\text{$h_2$})+\{\underbrace{(x)*s_1+(x)*s_2+(x)*a_0*b_1+(x)*a_1*b_0+(x+1)*a_1*b_1}_\text{$g$}\}$
${\scriptstyle fp_7}\xrightarrow[]{{\scriptstyle lt(f_2)}}[{\scriptstyle s_3}](\underbrace{{\scriptstyle \al+1}}_\text{$h_2$})+\{\underbrace{{\scriptstyle\al s_1+\al s_2+\al a_0b_1+\al a_1b_0+(\al+1)a_1b_1}}_\text{$g$}\}$


% \end{split}
\end{small}
Reduction order for $f:$
$f\xrightarrow[]{f_{10}}\xrightarrow[]{f_8}\xrightarrow[]{f_9}\xrightarrow[]{f_6}\xrightarrow[]{f_7}\xrightarrow[]{f_5}\xrightarrow[]{f_1}\xrightarrow[]{lt(f_2)}g$

% \begin{equation}
% \begin{split}
% h_4f_4+h_1f_1+h_2f_2+h_3f_3 = f+h_5f_5+h_6f_6+h_7f_7;\\
% h_4(d_0+P(e_1,c))+h_1f_1+h_2f_2+h_3f_3 = f+h_5f_5+h_6f_6+h_7f_7;\\
% h_4*d_0+h_4*P(e_1,c)+h_1f_1+h_2f_2+h_3f_3 = f+h_5f_5+h_6f_6+h_7f_7;\\
% h_4*P(e_1,c)+h_1f_1+h_2f_2+h_3f_3 = h_4*d_0+f+h_5f_5+h_6f_6+h_7f_7;\\
% h_4*P(e_1,c)+h_1f_1+h_2f_2+h_3f_3 = e_0*e_2+a*c+a+b*c+b+c;\\
% h_4*P(e_1,c)+h_1f_1+h_2f_2+h_3f_3 = e_0*e_2+a*c+a+b*c+b+c;
% \end{split}
% \end{equation}
Since, we know $g,h_2,f_3,f_4,J_0$, we can formulate it as ideal membership testing problem using~\eqref{member}:
\begin{center}
$g \in \langle h_2,f_3,f_4\rangle + \langle J_0\rangle$
\end{center}

This ideal membership test can be done using $lift$ procedure in SINGULAR~\cite{DGPS_410}. The procedure takes polynomial $f$, and ideal $J$ in row matrix form($[J+J_0]$) as inputs, and returns a column matrix $[U]$ as output such that:
\begin{small}
$f = [U]\cdot [J+J_0]$
\end{small}

% \begin{equation}
% \begin{split}
\begin{small}
$g = \begin{bmatrix} h_2^{'} & h_3^{'} & \dots & H \end{bmatrix} \cdot 
    \begin{bmatrix} h_2 \\ f_3 \\ \vdots \\ x_n^2 + x_n \end{bmatrix}$
\end{small}

The matrix output can be written in a linear combination as follows: 
% \end{split}
% \end{equation}
\begin{small}
$\al*s_1+\al*s_2+\al*a_0*b_1+\al*a_1*b_0+(\al+1)*a_1*b_1 = [\al*s_1+\al*s_2+\al*a_0*b_1+\al*a_1*b_0+(\al+1)*a_1*b_1](\al+1) + [0](s_2 + a_1*b_0) + [0](s_1 + a_0*b_1) + \dots + [0](x_n^2 + x_n)$;
\end{small}

The $lift$ procedure uses $GB$ to compute the linear combination. Since the generator set has a constant polynomial($\al+1$), the procedure considers it as constant one during membership computation. Hence the output in this case will be a 1x1 matrix with $g$ itself projected as the solution. To normalize the ignored constant, we need to divide(multiply the inverse) the solution $g$ by the constant$(\al+1)$ and reduce it with polynomials \{$f_3,f_4$\} in order to arrive at a implementable solution.

\begin{small}
$(\al+1)^{-1} = \al$

$h_2^{''} = \al*h_2^{'} = \al*(\al*s_1+\al*s_2+\al*a_0*b_1+\al*a_1*b_0+(\al+1)*a_1*b_1)$

$h_2^{''} = (\al+1)*s_1+(\al+1)*s_2+(\al+1)*a_0*b_1+(\al+1)*a_1*b_0+ a_1*b_1$

$h_2^{''}\xrightarrow[]{f_{3}}\xrightarrow[]{f_4}\underbrace{a_1*b_1}_\text{P}$
\end{small}

This computed $P$ is a valid solution and can be used as tail of $f_2$. If the resulting $P$ were not a solution in immediate support variables of $s_3$($a_1,b_1$), we could have arrived at this desired solution by devising a new term order. The new term order will have variables $(s_3,a_1,b_1)$ moved to the end on the variable order. We could then compute a $GB$ using the modified term order with the intermediate solution $P$ added as tail of $f_2$. This $GB$ will have one and only one polynomial which is of the form $s_3 + \mathcal{F}(a_1,b_1)$, where $\mathcal{F}$ is the function implemented by the gate for these variables.\\
Modified term order $\{Z\}>\{A>B\}>\{z_0>z_1\}>\{r_0>s_0\}>\{s_1>s_2$ $\}>\{a_0>b_0>s_3>a_1>b_1\}$.\\
To illustrate the computation of multiple solutions, let's take the current solution $a_1*b_1$ as our $P$. With RTTO$>$ as our term order, from~\eqref{quotcomp}:

$a_1*b_1 - P^{'} \in \{f_3,f_4,x_l^q-x_l\}:h_2$;\\
The result from quotient of ideal operation is given as:
{\small\begin{flalign*}
f_3:s_2 + a_1*b_0;  f_{13}:b_0^2 + b_0; f_{23}:A^4 + A;\\
f_4:s_1 + a_0*b_1;  f_{14}:b_1^2 + b_1; f_{24}:B^4 + B; \\
f_{11}:a_0^2 + a_0; f_{16}:s_1^2 + s_1; \\
f_{12}:a_1^2 + a_1; f_{17}:s_2^2 + s_2;
\end{flalign*}}%
Any polynomial from the above list when added in tail of $f_2$ along with $P$ satisfies the solution set for the given unknown component.
\end{Example}

\subsection{Circuit implementation as reference}
Consider a circuit implementation $C$, modeled as polynomials $F = \{f_1,\dots,f_s\}\in \mathbb{F}_q[in_j,x_1,\dots, x_n]$, with $J_1=\langle F \rangle$, $in_j$ as the set of all primary inputs, and $x_n$ as the word level output. Let us assume $f_i:1\le i \le s$ to be the unknown component which is of the special form:
\begin{gather*} 
f_i = x_k + P
\end{gather*}

Let us consider a different circuit $C_1$ as the golden specification which implements the same function as $C$. The reference circuit is modeled as polynomials $Q = \{q_1,\dots q_r\}\in \mathbb{F}_q[in_j,y_1,\dots, y_m]$, with $J_2=\langle Q \rangle$, $in_j$ as the set of all primary inputs, and $y_m$ as the word level output.

To formulate the problem, we will derive a new circuit structure using the above two implementations ($C,C_1$). Primary input set $in_j$ will be used as the common set of inputs for both the circuits, and the word level outputs ($x_n,y_m$) will be mitered using an XOR gate. A new specification polynomial $f$ is derived using the above setup as:
\begin{gather}
f : t*(x_n-y_m)
\end{gather}

where, $t$ is the final output of miter gate.

Now, for a correct implementation, specification $f$ should vanish on the variety of ideal generated by the circuit polynomials i.e., $f$ will be in the ideal generated by the circuit:

$f \in J_1 + J_2 + J_0$: where $J_0$ is the set of all vanishing polynomials from circuits $C$, $C_1$ and miter output $t$.

{\small $f \in \langle f_1,\dots,f_s\rangle + \langle q_1,\dots,q_r\rangle + \langle x_l^q-x_l\rangle + \langle y_u^q-y_u\rangle + \langle t^q-t\rangle$; $1\le l \le n,1\le u \le m$}

The problem formulation is now exactly same as~\eqref{member} with $f_i$ from circuit $C$ as the unknown gate. Now, we will follow the same procedure as in the first notion to realize the function of the unknown component. Once a solution has been computed, we can verify the circuit using principles from weak $\it{Nullstellensatz}$ by checking if $GB(J_1+J_2+J_0)=\{1\}$.

% \begin{algorithm}
% \caption{Resolve the unknown component for a given circuit}
% \label{algo:unknownComponent}
% \begin{algorithmic}[1]

% \Procedure{$multi\_variate\_division$}{$F,f$}

% \Procedure{$forward\_lifting$}{}

% \EndProcedure
% \end{algorithmic}
% \end{algorithm}

% Thus, $P(u_1)=P(e_1,c)=c$, which can implemented as a simple AND gate with $c$ as both inputs. 

% \begin{figure}[ht]
% 	\begin{center}
% 	\includegraphics[scale = 0.40]{mas_c}
% 	\end{center}
% 	\vspace{-1ex}
% 	\caption{correct implementation mastrovito}
% 	\label{mas_c}
% 	\vspace{-1ex}
% \end{figure}

% \begin{figure}[ht]
% 	\begin{center}
% 	\includegraphics[scale = 0.40]{mas_b}
% 	\end{center}
% 	\vspace{-4ex}
% 	\caption{buggy implementation mastrovito}
% 	\label{mas_b}
% 	\vspace{-2ex}
% \end{figure}