\section{Preliminaries}
Using computer algebra methods, the verification problem is typically formulated as a proof that the implementation satisfies the specification \cite{tim}\cite{lv}\cite{shekhar:tcad07}. The baseline approach is to check if the specification polynomial $f$ vanishes on the variety of the polynomials representing the circuit $C$ by performing a series of divisions. If it vanishes, then the circuit is deemed correct, else the remainder from reduction needs to be analyzed for bug localization and correction. Let $F=\{f_1,f_2,\dots,f_s\}$ denote a set of polynomials belonging to a finite field $\Fq$ implementing a given circuit $C$, $\Fqbar$ be its algebraic closure, and $f$ denote its specification polynomial. Let $R = \Fq[x_1,x_2,\dots,x_n]$ be a ring in variables $x_j : 1\le j\le n$, with coefficients from $\Fq$, where $q=2^k$.

Let $J\subseteq R$ be the $ideal$ generated by polynomials $f$ which is given as:
\begin{align}
\idealj
\end{align}
The set of all solutions to polynomial equations $F=0$, as in $\{f_1=0,f_2=0,\dots,f_s=0\}$ is called the variety and let $V$ denote such variety. The varieties generated are actually dependent on the $\it{ideal}$ $J$ and not on the generator polynomials themselves, hence we can represent the varieties as a function of $ideals$; $V(J)$. Since the ideals here are generated over finite fields, we can specify the variety over finite fields as $\vfqj$ and over its algebraic closure as $\vfbqj$. The verification problem is then formulated as testing if the specification polynomial $f$ vanishes on variety $\vfqj$. 

\begin{gather*}
\vspace{-2ex}
    f \textit{vanishes on } \vfqj \Leftrightarrow f \in I(\vfqj)\\
    \textit{from weak nullstellensatz : } \vfqj = \vfbqjjo \\
    \textit{from strong nullstellensatz : } I(\vfqj) = J + J_0, \text{where } J_0 = <x_i^q-x_i>
\end{gather*}

Just considering the generator set of $J$ to check if the specification $f$ vanishes on variety is not sufficient enough, and we require a canonical set of generators called \Grobner basis $GB$ to confirm the same. Recent approaches have discovered that the expensive $GB$ computation can be avoided altogether by using a specialized lexicographic term order $>$ which can be derived by analyzing the topology of the circuit. This term order\cite{lv} is derived by performing a reverse topological traversal of the circuit, and in this manuscript we refer to it as the {\it Reverse Topological Term Order}($RTTO$), wherein we start from the primary outputs, traverse the circuit to the primary inputs, and order the gates according to the their (reverse) topological levels. The RTTO renders the set of polynomials of the circuit itself a $GB$ and let $G_r$ represent such a minimal basis. Subsequently, the verification problems can be solved solely by way of GB-reduction (using multi-variate polynomial division), without any need to explicitly compute a GB.
\begin{gather*}
G_r = GB(J+J_0) \textit{ under lex RTTO \textgreater} \\
\end{gather*}

Under a given term order $>$, the specification polynomial successively reduced by the basis yields us a remainder, represented as:

\begin{gather*}
f \overset{G_r}{\longrightarrow}_+ r
\end{gather*}

If the remainder is '0', then we say that the implementation polynomial conforms to the specification polynomial and the circuit is deemed as correct. Whereas a non-zero remainder infers a buggy circuit implementation and needs to be analyzed further for a fault, localization and correction.

Given the specification polynomial $f$ and circuit polynomials $F=\{f_1,f_2, \dots, f_s\}\subseteq \R[x_1,x_2, \dots, x_n]$: $f_k\neq 0(1\leq k \leq s)$ and one unknown component $f_i$ which is of the special form $f_i = x_i + \mathcal{F}(x_j)$ where $x_i$ and $x_j$ are the known support variables with $j > i$, the algorithm returns the function implemented:$\mathcal{F}$ by this unknown gate.

Given a specification polynomial $f \in \Fq[x_1..x_n]=\R$ where $q=2^k$, and a circuit $C$ with $S$ gates. Write the gates as polynomials in $\R$ as $F=\{f_1,..,f_i,..,f_s\}: J=\langle f_1,..,f_i,..,f_s\rangle$. Let us consider $f_i$ to be the unknown component and of the special form $y_i+P(u_i)$, where $y_i$ is the leading term of the polynomial representing the unknown gate and $P$ as the function implementing the tail in terms of variables $u_i$, with $y_i$\textgreater $u_i$ as our variable order.

Let's assume that the circuit $C$ correctly implements $f$. Then $f\in I(\vfqj)=J+J_0: (J_0=\langle x_i^q-x_i\rangle)$, let's also assume $J\subset J_0$.

\begin{align}\label{eq1}
\begin{split}
    f \in J \implies & f = h_1f_1+\dots+h_if_i+\dots+h_sf_s: \\
    \text{where } h_i\in\R\\
    & h_if_i = f + h_1f_1+\dots+h_{i-1}f_{i-1}+\\
    h_{i+1}f_{i+1}+\dots+h_sf_s\\
    & h_if_i \in \langle f,f_1,\dots f_{i-1},f_{i+1}\dots f_s\rangle\\
\end{split}
\end{align}

% Let $J'$ represent this ideal $\langle f,f_1,f_2...f_{i-1},f_{i+1}...f_s\rangle$. Given the setup, can we project the variety of $J'$ on $y_i$ and $u_i$ coordinates and recover $f_i$?

\begin{align}
	\text{Is } h_if_i \in J' \cap \Fq[y_i,u_i]
\end{align}

Let $F=\{f_1,f_2,\dots,f_s\}$ denote a set of polynomials belonging to a finite field $\Fq$, and let $\overline{\Fq}$ be its algebraically closed complement. Let $R = \Fq\{x_1,x_2,\dots,x_n\}$ be a ring in variables $x_i, 1\le i\le n$ with coefficients from $\Fq$. 

\section{weak ${\it{Nullstellensatz}}$}
\subsection{Algebraically closed fields $\overline{\Fq}$}

The motive behind weak $\it{Nullstellensatz}$ is to reason about presence or absence of feasible solutions to a given $ideal$ over algebraically closed fields. Stating the theorem formally:

\begin{Theorem}
Let $\idealj$ be an ideal over the algebraically closed field $\overline{\Fq}[x_1,x_2,\dots,x_n]$ and let $\vfbqj$ be its variety. Given $J\subset\overline{\Fq}[x_1,x_2,\dots,x_n]$, $\vfbqj = \emptyset$ $\Leftrightarrow J=\overline{\Fq}[x_1,x_2,\dots,x_n] \Leftrightarrow 1 \in J \Leftrightarrow GB=\{1\}$.
\end{Theorem}

The theorem signifies that, over algebraically closed fields, if variety of a given $ideal$ is empty, then the $ideal $ encompasses the entire field. This implies that $J$ also contains the unit constant as one of the elements.If we approach the problem from a Gr\"obner Basis perspective, the reduced GB over this $J$ after finite number of iterations will end up with a unit polynomial as well, which can be used to reconstruct the entire Gr\"obner Set and in fact the whole Ring itself.
\subsection{ Not algebraically closed fields $\Fq$}
\begin{Theorem}
Let $\idealj$ be an ideal over the field $\Fq[x_1,x_2,\dots,x_n]$ and let $\overline{\Fq}$ be its algebraic closure. Let $\vfbqj$ be its variety over $\overline{\Fq}$. Given $J\subset\Fq[x_1,x_2,\dots,x_n]$, $\vfbqj=\emptyset \Leftrightarrow 1\in J \Leftrightarrow GB=\{1\}$.
\end{Theorem}

Over non closed fields, the theorem signifies that if variety of a given $ideal$ $J$ is empty, then the unit constant should be part of ideal $J$. This also infers no solution over algebraic closure $\overline{\Fq}$, hence implying no solution over $\Fq$ as well. To check if unit constant is part of the $ideal$ $J$, we need to do ideal membership testing by computing Gr¨obner basis.

Because it's tedious to compute solutions over finite fields, we need to consider the algebraically closed forms for the proof. We will see how to relate variety of $ideals$ over finite fields as compared to its algebraically closed counterpart.

Let's consider $J_0 = \langle x_1^q-x_1..x_n^q-x_n \rangle$ as the set of vanishing polynomials over algebraically closed field. These polynomials when coupled with finite field, restrict the solutions to finite fields. we also know that variety of $J_0$ doesn't change over the closure and is equal to $\vfbqjo= \Fq$.

Now to find out the existing solutions within the finite field, we need to find the intersection of varieties across algebraic closed fields and the finite field itself.

\quad \quad \quad $\vfqj = \vfbqj \cap \Fq$

\quad \quad \quad $\vfqj = \vfbqj \cap \vfqjo$

\quad \quad \quad $\vfqj = \vfbqj \cap \vfbqjo$

we know that, intersection over varieties is sum of ideals-

\quad \quad \quad $\vfqj = \vfbqjjo$

Hence to check if unit constant is a part of any finite field ideal, we can check if it is part of $J+J_0$ over its algebraically closed field.