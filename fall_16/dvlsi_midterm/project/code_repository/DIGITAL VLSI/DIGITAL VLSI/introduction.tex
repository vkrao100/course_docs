\begin{Introduction and Motivation}
	\label{section:introduction}
The bitcoin network is based on Cryptographic proof, and takes advantage of public key cryptography. When a person A is sending a bitcoin to person B, the private key of person A is used for encryption to the bitcoin which serves as a digital signature of A. However to authenticate the validity of those bitcoins the recipients of the bitcoin will require the public key of A. The public key of B also gets added to the bitcoin and it gets broadcasted throughout the bitcoin network letting everyone in the network know that particular bitcoin now belongs to B. These series of transactions are broadcasted in form of blocks throughout the network. A protection mechanism is thus used so that no illegal block can be added to these transactions by malicious attempt. Thus a proof of work function is added to each block to ensure its validity. Hashcash is such a proof of work function that is used by bitcoin network. It uses two iterations of the SHA-256 algorithm. Just like any other hash function it encrypts a data of any arbitrary size into a fixed sized data. If even one bit of the input data is modified, it generates a fully different hash value but the length of all the hashes will be the same. SHA-256 produces a 256-bit hash value and operates on 32-bit words.
Bitcoin is an upcoming and growing form of digital currency which is easy and does not require any third parties. However bitcoin theft has been documented on several occasions. There were instances when the bitcoin exchanges have shut down taking the client’s bitcoins with them. It is thus extremely important for a strong encryption algorithm as the bitcoin technology has started to appeal to masses and since many leading organizations started accepting bitcoins. We thus wanted to present a higher hash rate to enable better encryption.
\end{Introduction and Motivation}
