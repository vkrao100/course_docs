\section{Related Work}
\label{section:related}
Numerous existing ideas for cache energy and memory interface optimizations have been studied. Various factors contribute to the overall energy dissipation for both on-chip and off-chip communication.Every factor has a considerable impact on the energy consumption, thus requiring optimization for every such factor.  
 
\subsection{Energy Efficient Data Encoding Techniques}
Dynamic energy is dependent on frequency, load capacitance, voltage and activity factor. The average transitions that occur during one clock cycle, is accounted by activity factor. Every transition on the interconnects consumes energy and it is required to reduce the number of transitions in every cycle in order to reduce dynamic energy. Data encoding techniques, have been formed in the past, which reduce the activity factor, hence bringing down the dynamic energy.

Transitions occur for every change in the data bits, hence it is important to limit the Hamming weight of codewords. 3 Limited weight codes (3-LWC) is an example of M-LWC encoding scheme~\cite{limitedweight}, where the codewords have a Hamming weight of no more than M. These codewords, bring about a major reduction in the dynamic energy consumed as the Hamming weight and thus the transitions occurring due to the codewords reduces. In 3-LWC encoding scheme, one byte of data is transferred with a codeword which is 17bits long and with a maximum of three ones. A table storing all the frequently transferred data~\cite{frequentencoding} can be used. Every time data needs to be sent, it looks in to the table. When the data to be sent, is found in the table, a simple codeword with hamming distance of 1 is sent over an extra wire. This is an energy efficient scheme, when the frequency of similar data is high. In case, the data cannot be found in the look up table, the original data is sent over the wires.

When a data block needs to be transferred, an XOR operation is performed with previously transmitted data block~\cite{XORencoding}. This operation exploits the temporal locality between continuous blocks of data. An XOR operation of the present data with the previously transferred one, checks the difference in data between consecutive blocks and thus reduces transitions per clock cycle.  

Flip-N-Write~\cite{flipnwrite}, encodes the data to be written in either the original form or in its inverted form. A comparison is done with the previously stored data and accordingly picks the the form of data to be used. It is an extension of BI coding~\cite{bus-invert}, which improves the write energy in phase change memory. A write operation is modified into a read before write operation, in order to only transfer contradicting bits over the wire. 

    
\subsection{Memory Interface Optimizations}
DRAM data bus contributes a significant portion of the processor energy. Prior works to reduce the data transfer energy on the DRAM data bus have been discussed in this section. BD-encoding~\cite{bdencoding} compares the data to be sent over the DRAM data bus, with data which was sent over it recently. Based on the similarity with prior stored data set, the encoding scheme decides whether to send the xor-ed data i.e., original data xor-ed with most similar recent data word or to send the original data word. \cite{CAFO}, uses a cost calculating scheme, with respect to the number of transitions. A positive gain, leads to inversion of the bits. Data bus invert coding scheme is applied to both rows and columns, until no further reduction of bit flips is possible. More is Less~\cite{moreisless}, is another work to reduce the energy dissipated over the DDR data bus. This work exploits the inactive nature of the data bus due to timing constraints within the DRAM. A longer codeword with less ones, which utilizes the data bus for a longer time is used, in order to reduce the energy used for data transfer. All these techniques focus on reducing the number of ones/zeros to be transferred to reduce the termination energy. FADE-DDR completely removes termination energy by switching to unterminated interface and further reduces switching energy by employing efficient encoding for low swing transisitional signalling. An extension of bus invert~\cite{bus-invert}  is data bus invert~\cite{dbi}, where the number of 1s are checked for the present data rather than with the previous data.     



\subsection{Cache Energy Optimizations}
In the past, various encoding schemes have been presented for cache energy optimization. Bus invert coding~\cite{bus-invert}  is an energy efficient bus encoding technique, where Hamming distance of the data to be sent and  previously sent is calculated. If the Hamming distance exceeds N/2 for an N bit data, all the N bits need to be inverted. This encoding technique forms segments of bits, in order to use an extra wire for every segment. Bus invert is sensitive to the segment number and needs to be chosen carefully. The extra wire is used to indicate whether the data transmitted is either in original or inverted form by a bit-0 or bit-1 respectively. 
DESC~\cite{desc} employs a time dependent transfer scheme using synchronized counters. DESC is applicable only to the last-level cache and doesn't reduce the interconnect energy for upper levels. Reducing data movement energy via online data and clustering encoding, talks about reducing the number of transitions over the last level cache and the DDR interface. This is done with the help of clustering, where the data that needs to be transmitted chooses the clustering centers with least hamming distance and performs an XOR with the original data. The data transmitted includes the residual information and additional bits, to identify the cluster center at the receiver end. Every time a data has been transmitted, the cluster is updated both at the transmitter and the receiver end.
DESC~\cite{desc} is a time-dependent encoding scheme to transmit information over wires. Data to be transferred is divided into groups of 4-bits(chunks) representing a value range from 0-15. Based on the value 'N' represented by the chunks, a bit flip is introduced after 'N' clock cycles of transmission of the previous chunk. This technique includes a reset line to initiate a data transfer. In addition to the conventional encoding, this techniques includes additional schemes such as zero skipping and last value skipping, which further reduce the flips required for transmission of the data.




  
