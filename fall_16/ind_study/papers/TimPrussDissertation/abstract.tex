%\begin{abstract}
\begin{center}{\bf ABSTRACT}\end{center}


Abstraction plays an important role in digital design,
analysis, and verification, as it allows for the refinement of
functions through different levels of conceptualization. This dissertation
introduces a new method to compute a symbolic, canonical, word-level abstraction of the function implemented
by a combinational logic circuit. This abstraction provides a
representation of the function as a {\it polynomial} $Z = F (A)$
over the Galois field $\Fkk$, expressed over the $k$-bit input to the circuit, $A$.
This representation is easily utilized for formal verification (equivalence checking) of
combinational circuits.

The approach to abstraction is based upon concepts from
commutative algebra and algebraic geometry, notably the
\Grobner basis theory. It is shown that the polynomial $F(A)$ can
be derived {\it by computing a \Grobner basis of the polynomials}
corresponding to the circuit, using a specific elimination term
order based on the circuit’s topology. However, computing \Grobner bases
using elimination term orders is infeasible for large circuits.
To overcome these limitations, this work introduces {\it an efficient symbolic
computation} to derive the word-level polynomial. The presented algorithms
exploit i) the structure of the circuit, ii) the properties of \Grobner bases,
iii) characteristics of Galois fields $\Fkk$, and iv) modern algorithms from symbolic computation.

While the concept is
applicable to any arbitrary combinational logic circuit, it 
is particularly powerful in verification and equivalence checking of 
hierarchical, custom-designed and structurally dissimilar Galois field arithmetic circuits. 
%These designs are prevalent in cryptography, error correction codes, and signal processing.
In most applications, the field size and the data-path size $k$
in the circuits is very large, up to $1024$ bits.
%The U.S. National
%Institute for Standards and Technology (\emph{NIST}), recommends Galois
%fields corresponding to operand-sizes $k = 163, 233, 283, 409$ and
%$571$ bits wide. 

The proposed abstraction procedure can exploit the hierarchy of the
given Galois field arithmetic circuits.
% by computing the abstraction
%of each block in parallel before combining the representations together. 
A custom abstraction %software
tool is designed to efficiently implement the abstraction procedure.
Preliminary experiments show that, using the proposed approach, our
tool can abstract and verify Galois field arithmetic circuits
{\it up to $1024$ bits in size}. %, which is the largest NIST standard. 
Contemporary
techniques fail to verify these types of circuits beyond $163$ bits
and cannot abstract a canonical representation beyond $32$ bits.

%To efficiently implement the
%procedure, we design a custom abstraction software tool. 
%Modern abstraction techniques of
%these circuits fail beyond $32$-bits.


%Custom arithmetic circuits designed over Galois fields $\Fkk$ are
%prevalent in cryptography, error correction codes, signal processing,
%etc. In most applications, the field size and the data-path size $k$
%in the circuits can be very large. For example, the U.S. National
%Institute for Standards and Technology (NIST), recommends Galois
%fields corresponding to operand-sizes $k = 163, 233, 283, 409$ and
%$571$ bits wide. The complexity of such large architectures 
%necessitates custom and hierarchical design. Custom design can raise
%the potential for bugs in the implementation. As arithmetic bugs can
%compromise the security of crypto-systems, formal verification of
%Galois field circuits is an imperative.

%This thesis presents an approach for formal verification and
%equivalence checking of large Galois field arithmetic circuits based
%on word-level canonical polynomial abstraction from gate-level
%circuits. The word-level abstraction is based on concepts from
%commutative algebra and algebraic geometry, and relies on the
%following mathematical insights.

%A combinational circuit with $k$-inputs and $k$-outputs implements
%Boolean functions $f: \B^k \rightarrow \B^k$, where $\B = \{0, 1\}$.
%Such functions can also be construed as a mapping $f: \Fkk
%\rightarrow \Fkk$,  where $\Fkk$ denotes the Galois field of $2^k$
%elements. Every function over $\Fkk$ is a polynomial function ---
%i.e. there exists a unique, minimal, canonical polynomial
%$\Func$ that describes $f$.  This thesis derives the polynomial
%representation of the circuit as $Z = \Func(A)$ over $\Fkk$, where $A$
%and $Z$ denote, respectively, the input and output bit-vectors of the
%circuit. This can be achieved by computing a Gr\"obner basis of a set
%of polynomials derived from the circuit, using a specific elimination
%(abstraction) term order.

%Computing Gr\"obner bases using elimination orders is, however,
%practically infeasible for large circuits. To overcome this
%limitation, we devise an approach to guide the Gr\"obner basis
%computation to search for the abstraction polynomial. The approach is
%implemented in a custom tool using modern $F_4$-style polynomial
%reduction procedures. Using our approach, we are able to verify the
%correctness of up to $571$-bit Galois field multiplier circuits,
%whereas contemporary techniques prove to be infeasible beyond
%$163$-bit circuits. 
%\end{abstract}
