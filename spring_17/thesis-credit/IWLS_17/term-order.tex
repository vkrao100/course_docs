%\documentclass[pdf,final,colorBG,slideColor]{prosper}
\documentclass[xcolor=dvipsnames]{beamer}
%%\usetheme{default}
\usecolortheme[named=Maroon]{structure}
%\usetheme{Boadilla}
\usetheme{Madrid}
\useoutertheme{default}%[footline=empty]{infolines}

%\useoutertheme[footline=empty]{infolines}

% \usepackage{helvet}
% \usepackage{enumerate}
% \usepackage{amsmath}
% \usepackage{amsfonts}
% \usepackage{graphicx}
% \usepackage{ulem}
% \usepackage{multirow}
\usepackage{comment}

%%% Trial by Priyank
\usepackage{polynom}

\usepackage[absolute,overlay]{textpos}

\RequirePackage{algorithmic}
\PassOptionsToPackage{boxed}{algorithm}
\RequirePackage{algorithm}
\renewcommand{\algorithmicrequire}{\textbf{Inputs:}}
\renewcommand{\algorithmicensure}{\textbf{Outputs:}}


%\newtheorem{theorem}{Theorem}
%\newtheorem{lemma}{lemma}
%\newtheorem{corollary}{Corollary}
%\newtheorem{proposition}{Proposition}
%\newtheorem{Q}{Question}
%\newtheorem{Exa}{Example}
%\newtheorem{Definition}{Definition}

\newcommand{\B}{{\mathbb{B}}}
\newcommand{\Z}{{\mathbb{Z}}}
\newcommand{\R}{{\mathbb{R}}}
\newcommand{\Q}{{\mathbb{Q}}}
\newcommand{\C}{{\mathbb{C}}}
\newcommand{\Zn}{{\mathbb{Z}}_{n}}
\newcommand{\Zp}{{\mathbb{Z}}_{p}}
\newcommand{\F}{{\mathbb{F}}}
\newcommand{\Fbar}{{\overline{\mathbb{F}}}}
\newcommand{\Fq}{{\mathbb{F}}_{q}}
\newcommand{\Fkk}{{\mathbb{F}}_{2^k}}
\newcommand{\Zkk}{{\mathbb{Z}}_{2^k}}
\newcommand{\Fkkx}[1][x]{\ensuremath{\mathbb{F}}_{2^k}[#1]\xspace}
\newcommand{\Grobner}{Gr\"{o}bner\xspace}
\newcommand{\bi}{\begin{itemize}}
\newcommand{\ei}{\end{itemize}}

\newcommand{\blu}{\color{blue}}

\title[Division+Term Orderings]{Division Algorithms and Term
  Orderings for Symbolic Computation}

\subtitle{Motivating Gr\"obner Bases and their computation}


\author[P. Kalla]{Priyank Kalla}
%\email{rostamian@umbc.edu}
\institute[Univ. of Utah]{
\includegraphics[height=17mm]{/Users/Kalla/teaching/Comp-Algebra-Course/lectures/old_ulogo.eps}\\
\ \\
Associate Professor\\
Electrical and Computer Engineering, University of Utah\\
kalla@ece.utah.edu\\
\url{http://www.ece.utah.edu/~kalla}
}


\date{Lectures: March 9 - onwards}
%\slideCaption{}

%% Images
\pgfdeclareimage[width=.4in]{fg:logo}
{/Users/Kalla/teaching/Comp-Algebra-Course/lectures/old_ulogo.eps} 

%%%%%%%%%%%%%%%%%%%%%%%%%%%%%%%%%%%%%%%%%%%%%%%%%%
%%%%%%%%%%%%%%%%%%%%%%%%%%%%%%%%%%%%%%%%%%%%%%%%%%
%%%%%%%%%%%%%%%%%%%%%%%%%%%%%%%%%%%%%%%%%%%%%%%%%%
\begin{document}


%----------- titlepage ----------------------------------------------%
\begin{frame}[plain]
  \titlepage

\end{frame}

%\maketitle

%%%%%%%%%%%%Try section
 %% \section*{Outline}
 %% \begin{frame}
 %%   \frametitle{Outline}
 %% \tableofcontents
 %% \end{frame}

% \section{Intro}
% \subsection{What's that?}
% \subsection{Overview of something}

% \section{Galois Fields}
% \subsection{what does this give?}
% \begin{frame}
% \end{frame}

%%%%%%%%%%

%%%%%%%%%%%%%%%%%%%%%%%%%%%%%%%%%%%%%%%%%%%%%%%%%%


%%%%%%%%%%%%%%%%%%%%%%%%%%%%%%%%%%%%%%%%%%%%%%%%%%
\begin{frame}{{\large Agenda: }}

\bi
\item Wish to build a polynomial algebra model for hardware
\item Modulo arithmetic model is versatile: can represent both {\it
  bit-level} and {\it word-level} constraints 
\item To build the algebraic/modulo arithmetic model:
\bi
\item Rings, Fields, Modulo arithmetic
\item Polynomials, Polynomial functions, Polynomial Rings
\item \alert{Ideals, Varieties, Symbolic Computing and Gr\"obner Bases}
\item \alert{Decision procedures in verification}
\ei

\ei

\end{frame}


%%%%%%%%%%%%%%%%%%%%%%%%%%%%%%%%%%%%%%%%%%%%%%%%%%

%%%%%%%%%%%%%%%%%%%%%%%%%%%%%%%%%%%%%%%%%%%%%%%%%%

\begin{frame}{{\large Consolidating the results so far....}}

\bi
\item Ideal $J = \langle f_1, \dots, f_s \rangle \subseteq R =\F[x_1,
  \dots,x_d]$ generated by any set of polynomials $f_1, \dots, f_s$ 
\item $J = \langle f_1, \dots, f_s \rangle = \{\sum_{i=1}^s f_i \cdot
  h_i ~:~ h_i \in  R\}$ 
\item Many ideal generators: $J = \langle f_1, \dots, f_s\rangle =
  \dots = \langle g_1, \dots, g_t\rangle$  
\bi
\item Given: $F = \{f_1, \dots, f_s\} \in R$
\item Gr\"obner basis: $G = \{g_1, \dots, g_t\}$ a canonical
  representation of ideal $J = \langle F \rangle = \langle G \rangle$
\item Buchberger's algorithm computes a Gr\"obner basis, which we will
  study soon
\ei
\item Variety: the set of all solutions to $f_1 = \dots = f_s =0$
\item Variety depends on the ideal $J$, not just on $f_1, \dots, f_s$
\item $V(f_1, \dots, f_s) = V(g_1, \dots, g_t) = V(J)$
\ei
\end{frame}
%%%%%%%%%%%%%%%%%%%%%%%%%%%%%%%%%%%%%%%%%%%%%%%%%%

\begin{frame}{{\large Some facts about ideals and varieties}}

\bi
\item When ideal $J = \langle 1 \rangle \subseteq \F[x_1,\dots,x_d]$, 
then $J = F[x_1, \dots, x_d]$
\item  $J = \langle 1 \rangle \iff V(J) = \emptyset$; as the
  polynomial $1=0$ has no solutions
\item \alert{Variety:} Set of ALL solutions to a given system of
  polynomial equations: $V(f_1, \dots, f_s)$ 
	\begin{itemize}
	\item $V(x^2+y^2-1)=\{all\  points\  on\ circle: x^2+y^2-1=0\}$
	\item $V_{\mathbb{R}}(x^2+1)=\emptyset$; 
        \item $V_{\mathbb{C}}(x^2+1)=\{(\pm i)\}$
	\end{itemize}
\item Important to analyze variety over a specific field ($V_{\R}$
  versus $V_{\C}$)
\item Modern algebraic geometry does not \alert{explicitly solve for
  the varieties}. Rather, it reasons about the Variety by analyzing
  the Ideals! 
\bi
\item Solving for varieties is extremely hard
\item Reasoning about their presence, absence, union/intersection is
  easier
\item We need to do the same for hardware verification
\ei
\ei
\end{frame}


%%%%%%%%%%%%%%%%%%%%%%%%%%%%%%%%%%%%%%%%%%%%%%%%%%

%%%%%%%%%%%%%%%%%%%%%%%%%%%%%%%%%%%%%%%%%%%%%%%%%%
\begin{frame}{{\large Formally define a variety}}

\bi
\item Let $R = \F[x_1, \dots, x_d]$ be a ring, $f \in R$ be a
  polynomial and $\mathbf{a} = (a_1, \dots, a_d) \in \F^d$ be a point
\item We say that $f$ \alert{vanishes} on $\mathbf{a}$ when
  $f(\mathbf{a})=0$
\ei

\begin{Definition}
For any ideal $J = \langle f_1, \dots, f_s \rangle \subseteq
\mathbb{F}[x_1,\dots, x_d]$, the {\it affine variety} of $J$ over
$\mathbb{F}$ is:
$$V(J) = \{\mathbf{a} \in \mathbb{F}^d: \forall f \in
J, f(\mathbf{a}) = 0\}.$$ 
\end{Definition}

\end{frame}
%%%%%%%%%%%%%%%%%%%%%%%%%%%%%%%%%%%%%%%%%%%%%%%%%%


%%%%%%%%%%%%%%%%%%%%%%%%%%%%%%%%%%%%%%%%%%%%%%%%%%
\begin{frame}{{\large Algebraically Closed Fields (ACFs)}}

\bi
\item A field $\Fbar$ is algebraically closed, when every non-constant
  univariate polynomial $f \in \Fbar[x]$ has a root in $\Fbar$
\item Every field is either algebraically closed, or it is contained
  in an algebraically closed one  
\item Algebraically closed fields are infinite fields
\item Only over algebraically closed fields can one reason
  (unambiguously) about presence or absence of solutions (varieties)  
\bi
\item Many famous mathematical results valid (only!) over ACFs
\ei
\item Examples: $\R$ is not ACF as $V_{\mathbb{R}}(x^2+1)=\emptyset$; 
\item $\C$ is ACF; in fact $\C$ is the algebraic closure of $\R$ ($\R
  \subset \C$)
\item Finite (Galois) fields are NOT ACF! 
\bi
\item But every GF $\F_{p^k} \subset \overline{\F_{p^k}}$, where 
$\overline{\F_{p^k}}$ is the algebraic closure of $\F_{p^k}$
\ei
\item So how will we reason about $V_{\Fkk}(J)$? We will, using some
  funky Galois field results (Galois fields are awesome!)
\ei

\end{frame}
%%%%%%%%%%%%%%%%%%%%%%%%%%%%%%%%%%%%%%%%%%%%%%%%%%


%%%%%%%%%%%%%%%%%%%%%%%%%%%%%%%%%%%%%%%%%%%%%%%%%%
\begin{frame}{{\large There's a lot to study about Varieties, but...}}

\bi
\item This is a good time to first think in terms of a canonical
  representation of ideals --- i.e. a \alert{Gr\"obner Bases}
\item Recall:
\item Given polynomials $f_1, \dots, f_s \in \F[x_1,\dots, x_d]$. Let
  $F = \{f_1, \dots, f_s\}$ be the given set of polynomials 
\item Then ideal $J = \langle F \rangle \subset \F[x_1, \dots, x_d]$
\item Find another set of polynomials $G = \{g_1, \dots, g_t\} \in
  \F[x_1, \dots, x_d]$ such that: 
\bi
\item $J = \langle F \rangle = \langle G \rangle$
\item $V(J) = V(\langle F \rangle) = V(\langle G \rangle)$
\item The set $G$ has some nice properties that $F$ does not have
\item The set $G$ is called a \alert{Gr\"obner basis of ideal $J$}
\ei
\ei

\end{frame}
%%%%%%%%%%%%%%%%%%%%%%%%%%%%%%%%%%%%%%%%%%%%%%%%%%

%%%%%%%%%%%%%%%%%%%%%%%%%%%%%%%%%%%%%%%%%%%%%%%%%%
\begin{frame}{{\large The power of Gr\"obner bases}}

\bi
\item A Gr\"obner basis $G$ can help us solve (unambiguously) many
  polynomial decision questions:
\ei

\begin{block}{Ideal Membership Testing}
Given ideal $J = \langle f_1, \dots, f_s\rangle \subseteq \F[x_1,
  \dots, x_d]$, and a polynomial $f$, is $f \in J$?
\end{block}

\begin{block}{Hilbert's Nullstellensatz: The polynomial SAT/UNSAT problem}
Given ideal $J = \langle f_1, \dots, f_s\rangle \subseteq \F[x_1,
  \dots, x_d]$, is $V(J) = \emptyset$?
\end{block}

\begin{block}{The polynomial \#SAT problem}
Given ideal $J = \langle f_1, \dots, f_s\rangle \subseteq R = \F[x_1,
  \dots, x_d]$, is $V(J)$ infinite or finite? If finite, then $|V(J)|$
= ? [i.e. how many solutions to $V(J)$?]
\end{block}

\begin{block}{Elimination ideals: help in solving polynomial
    equations}
Generalize \alert{triangularization} to polynomial equations
\end{block}

\end{frame}
%%%%%%%%%%%%%%%%%%%%%%%%%%%%%%%%%%%%%%%%%%%%%%%%%%


%%%%%%%%%%%%%%%%%%%%%%%%%%%%%%%%%%%%%%%%%%%%%%%%%%
\begin{frame}{{\large A Gr\"obner basis example [From Cox/Little/O'Shea]}}


\begin{columns}

\begin{column}{5cm}
Solve the system of equations:
\begin{align*}
f_1: x^2 - y - z -1 &= 0\\
f_2: x - y^2 - z -1 &= 0\\
f_3: x -y -z^2 -1   & =0
\end{align*}
\end{column}

\begin{column}{4.5cm}

Gr\"obner basis with lex term order $x > y > z$
\begin{align*}
g_1: & x - y - z^2 -1 & =0\\
g_2: & y^2 - y - z^2 - z & =0\\
g_3: & 2yz^2 - z^4 - z^2 & = 0\\
g_4: & z^6 - 4z^4 - 4z^3 - z^2 & = 0
\end{align*}
\end{column}

\end{columns}

\bi
\item Is $V(\langle G \rangle) = \emptyset$? No, because $G \neq \{1\}$
\item $G$ tells me that $V(\langle G \rangle)$ is finite!
\item $G$ is {\it triangular}: solve $g_4$ for $z$, then $g_2, g_3$
  for $y$, and then $g_1$ for $x$
\ei

\end{frame}
%%%%%%%%%%%%%%%%%%%%%%%%%%%%%%%%%%%%%%%%%%%%%%%%%%

%%%%%%%%%%%%%%%%%%%%%%%%%%%%%%%%%%%%%%%%%%%%%%%%%%
\begin{frame}{{\large To start thinking in terms of Gr\"obner bases....}}
\begin{block}{Ideal Membership Testing}
Given ideal $J = \langle f_1, \dots, f_s\rangle \subseteq \F[x_1,
  \dots, x_d]$, and a polynomial $f$, is $f \in J$?
\end{block}

\pause 
Can you think of an approach to decide ideal membership?

\pause 
\bi [<+->]
\item Let $f = y^2x - x, ~~f_1 = yx - y, ~~f_2 = y^2 - x; ~~(y>x)$
\item Is $f \in \langle f_1, f_2 \rangle$?
\item Divide $f$ by $f_1$, obtain quotient and remainder $(q_1, r_1)$
\item Then, divide $r_1$ by $f_2$, obtain $(q_2, r_2)$
\item If $r_2 = 0$, then $f = q_1 f_1 + q_2 f_2$, so $f \in \langle f_1, f_2\rangle$.
\item But, what if we divide $f$ by $f_2$ first and then by $f_1$?
\item The culprits are: \alert{term ordering issues and the division
  algorithm}
\item Let us study these in detail
\ei

\end{frame}
%%%%%%%%%%%%%%%%%%%%%%%%%%%%%%%%%%%%%%%%%%%%%%%%%%

%%%%%%%%%%%%%%%%%%%%%%%%%%%%%%%%%%%%%%%%%%%%%%%%%%
\begin{frame}{{\large The one variable case of $\F[x]$}}

\bi
\item $f = a_n x^n + a_{n-1}x^{n-1} + \dots + a_1 x + a_0$
\item The terms of $f$ are ordered according to (descending) degrees
\item $\text{deg}(f) = n$ is the degree of $f$
\item $lt(f) = a_n x^n$ is the \alert{leading term} of $f$
\item $lm(f) = x^n$ is the \alert{leading monomial} of $f$ [often also
  called the leading power of $f (lp(f))$]
\item $lc(f) = a_n$ is the \alert{leading coefficient} of $f$
\item $lt, lm, lc$ are the main tools of the division algorithm
\ei

\end{frame}
%%%%%%%%%%%%%%%%%%%%%%%%%%%%%%%%%%%%%%%%%%%%%%%%%%

%%%%%%%%%%%%%%%%%%%%%%%%%%%%%%%%%%%%%%%%%%%%%%%%%%
\begin{frame}{{\large Polynomial Long Division (say, in $\Q[x]$)}}

Divide $f$ by $g$, get $q, r$ s.t. $f = qg + r$, with $r = 0$ or
$deg(r) < deg(g)$
\pause
Divide $f = x^3-2x^2+2x+8$ by $g =2x^2+3x + 1$\\

\pause

\polylongdiv{x^3-2x^2+2x+8}{2x^2+3x + 1}

\pause

\bi
\item Multiply $g$ by $\frac{1}{2}x$ and then compute: $r = f - \frac{1}{2}x g$
\item The key step in division: \alert{$r = f - \frac{lt(f)}{lt(g)}
  g$}
\item One-step reduction of $f$ by $g$ to $r$: $f \xrightarrow{g} r$
\item Repeatedly apply reduction: $f$ reduces to $g$ modulo $r$: $f \xrightarrow{g}_+ r$
\ei

\end{frame}
%%%%%%%%%%%%%%%%%%%%%%%%%%%%%%%%%%%%%%%%%%%%%%%%%%

%%%%%%%%%%%%%%%%%%%%%%%%%%%%%%%%%%%%%%%%%%%%%%%%%%
\begin{frame}{{\large Division Algorithm is so Simple...}}

 \begin{algorithm}[H]
 \caption{Univariate Division of $f$ by $g$}
 \label{algo:reduce}
 \begin{algorithmic}[1]

 \REQUIRE $f, g \in \F[x], g\neq 0$
 \ENSURE $q, r$ s.t. $f = qg+r$ with $r =0$ or $deg(r) < deg(g)$
 \STATE{ {$q \gets 0; ~r \gets f $} }
 \WHILE {{  ($r \neq 0$ AND $deg(g) \leq deg(r)$) }}
 \STATE{{ $q \gets q + \frac{lt(r)}{lt(g)}$}}
 \STATE{{ $r \gets r - \frac{lt(r)}{lt(g)} \cdot g $}}
 \ENDWHILE
 \STATE{{ return $q, r$;}}

 \end{algorithmic}
 \end{algorithm}

Run the algorithm on the previous example\\

Does this algorithm run on $\Z_p[x]$ as is? Say, over $\Z_{11}[x]$ for
the previous example? What about over $\Z_{8}[x]$? 

\end{frame}
%%%%%%%%%%%%%%%%%%%%%%%%%%%%%%%%%%%%%%%%%%%%%%%%%%


\begin{frame}{{\large Univariate Division }}
\bi
\item Remember: Division is modeled as cancellation of leading terms
  ($lt(f)$) by leading terms ($lt(g)$)
\item For $r = f - \frac{lt(f)}{lt(g)} g = f -
  \frac{lc(f)}{\alert{lc(g)}}\cdot\frac{lm(f)}{lm(g)}\cdot g$
\item Requires computation of inverse of $lc(g)$
\item This division algorithm works over fields $\F = \R, \Q, \C,
  \Z_p, \Fkk$, etc.
\item This division algorithm does not always work over $\Z, \Zn,
  n\neq p$. 
\ei
\end{frame}


%%%%%%%%%%%%%%%%%%%%%%%%%%%%%%%%%%%%%%%%%%%%%%%%%%
\begin{frame}{{\large Application to Ideal Membership Test}}

\bi [<+->]
\item Let $f = x; ~~f_1 = x^2; ~~f_2 = x^2 - x$ in $\Q[x]$
\item Is $f \in J = \langle f_1, f_2 \rangle$?
\item $f = f_1 - f_2$, so surely $f \in J$?
\item $f \xrightarrow{f_1} f \xrightarrow{f_2} f \neq 0$
\item $f \xrightarrow{f_2} f \xrightarrow{f_1} f \neq 0$
\item What's happening?
\item $F = \{f_1, f_2\}$ is not a Gr\"obner basis of $J$
\item Cannot decide ideal membership without Gr\"obner basis!
\ei

\end{frame}
%%%%%%%%%%%%%%%%%%%%%%%%%%%%%%%%%%%%%%%%%%%%%%%%%%


%%%%%%%%%%%%%%%%%%%%%%%%%%%%%%%%%%%%%%%%%%%%%%%%%%
\begin{frame}{{\large Gr\"obner Bases over Univariate Polynomial Rings
      $\F[x]$}}

\bi
\item When $\F$ is a field, \alert{Every ideal} $J$ of $\F[x]$ is
  generated by only one element (polynomial).
\bi
\item These rings $\F[x]$ are \alert{principal ideal domains (PID)}
\item {\it E.g.} $\Zp[x]=$ PID, but multivariate rings are not PIDs
  ({\it e.g} $\Zp[x_1, x_2] \neq$ PID)
\item Ideal of vanishing polynomials is a good example: $\langle x^p-x
  \rangle$ versus $\langle x_1^p - x_1, x_2^p - x_2 \rangle$
\ei
\item Gr\"obner Basis of $\{f_1, f_2\} = \text{GCD}(f_1, f_2)$
\item Gr\"obner Basis of $\{f_1, \dots, f_s\} = \text{GCD}(f_1,
  \text{GCD}(f_2, \dots, f_s))$
\item The Euclidean Algorithm computes the GCD of two polynomials
\item The algorithm is given in any math textbook, and can also be
  found on wikipedia (Internet)
\item Homework assignment for you..... Euclidean algorithm
\item Univariate rings are of not much use in hardware verification 
\ei

\end{frame}
%%%%%%%%%%%%%%%%%%%%%%%%%%%%%%%%%%%%%%%%%%%%%%%%%%


%%%%%%%%%%%%%%%%%%%%%%%%%%%%%%%%%%%%%%%%%%%%%%%%%%
\begin{frame}{{\large Division in Multivariate Rings $\F[x_1, \dots, x_d]$}}

\bi [<+->]
\item Divide $f = y^2x + 4yx - 3x^2$ by $g = 2y + x + 1$
\item Recall: Division is cancellation by leading terms
\item What are $lt(f), lt(g)$?
\item We need to figure out how to order the terms of $f, g$
\ei



\end{frame}
%%%%%%%%%%%%%%%%%%%%%%%%%%%%%%%%%%%%%%%%%%%%%%%%%%


%%%%%%%%%%%%%%%%%%%%%%%%%%%%%%%%%%%%%%%%%%%%%%%%%%
\begin{frame}{{\large Monomial (Term) Orderings}}

Power product: $x_1^{\alpha_1}x_2^{\alpha_2}\dots x_d^{\alpha_d},
\alpha_i \in \Z_{\geq 0}$. \\
\ \\
For simplicity: $x_1^{\alpha_1}\dots x_d^{\alpha_d} = \mathbf{x^{\alpha}, ~ \alpha} \in \Z^{d}_{\geq 0}$. 

\begin{itemize}
\item Term = $a\cdot \mathbf{x^{\alpha}}$ = coeff. times a power product
\item $ \mathbb{T}^d = \{ \mathbf{{x}^{\alpha}}: \mathbf{\alpha}
  \in \mathbb{Z}_{\geq 0}\}$ is the set of all power products 
\item A multivariate polynomial is a sum of terms
\end{itemize}


\end{frame}
%%%%%%%%%%%%%%%%%%%%%%%%%%%%%%%%%%%%%%%%%%%%%%%%%%


%%%%%%%%%%%%%%%%%%%%%%%%%
\begin{frame}{\large Impose a Monomial Ordering on $\F[x_1, \dots, x_d]$}


A total order $<$ on $\mathbb{T}^d$, and it should be a well-ordering:
\begin{itemize}
\item Total order: One and only one of the following should be true:
  $x^{\alpha} > x^{\beta}$ or   $x^{\alpha} = x^{\beta}$ or   $x^{\alpha} < x^{\beta}$.
\item $1 < x^{\alpha}, ~~~ \forall x^{\alpha} ~~~(x^{\alpha} \neq 1)$
\item $x^{\alpha} < x^{\beta} \implies x^{\alpha}\cdot x^{\gamma} <  x^{\beta}\cdot x^{\gamma}$.
\end{itemize}

\begin{Definition}[LEX]
{\bf Lexicographic order:} Let $x_1 > x_2 > \dots > x_d$
lexicographically. Also let $\alpha = (\alpha_1, \dots, \alpha_d);
~\beta = (\beta_1, \dots, \beta_d) \in \mathbb{Z}^d_{\geq 0}$. Then we
have: 
\begin{equation*}
x^{\alpha} < x^{\beta} \iff 
\begin{cases}
& \text{Starting  from the  left, the first co-ordinates of $\alpha_i, 
  \beta_i$} \\
& \text{that are different satisfy $\alpha_i < \beta_i$}

\end{cases}
\end{equation*}

\end{Definition}

For 2-variables: $1 < x_2 < x_2^2 < \dots < x_1 < \dots x_2x_1 < \dots
< x_1^2 < \dots$
\end{frame}
%%%%%%%%%%%%%%%%%%%%%%%%%%%%%%


%%%%%%%%%%%%%%%%%%%%%%%%%%%%%%%%%%%%%%%%%%%%%%%%%%
\begin{frame}{{\large DegLex and DegRevLex Orderings}}


\begin{Definition}[DEGLEX]
{\bf Degree Lexicographic order:} Let $x_1 > x_2 > \dots > x_d$
lexicographically. Also let $\alpha = (\alpha_1, \dots, \alpha_d);
~\beta = (\beta_1, \dots, \beta_d) \in \mathbb{Z}^d_{\geq 0}$. Then we
have: 
\begin{equation*}
x^{\alpha} < x^{\beta} \iff 
\begin{cases}
\sum_{i=1}^{d}\alpha_i < \sum_{i=1}^{d} \beta_i & \text{ OR }\\
\sum_{i=1}^{d}\alpha_i = \sum_{i=1}^{d} \beta_i  ~\text{AND}~
x^{\alpha} < x^{\beta} & \text{w.r.t. LEX order}
\end{cases}
\end{equation*}
\end{Definition}


\begin{Definition}[DEGREVLEX]
{\bf Degree Reverse Lexicographic order:} Let $x_1 > x_2 > \dots > x_d$
lexicographically. Also let $\alpha = (\alpha_1, \dots, \alpha_d);
~\beta = (\beta_1, \dots, \beta_d) \in \mathbb{Z}^d_{\geq 0}$. Then we
have: 
\begin{equation*}
x^{\alpha} < x^{\beta} \iff 
\begin{cases}
\sum_{i=1}^{d}\alpha_i < \sum_{i=1}^{d} \beta_i  \text{ or }\\
\sum_{i=1}^{d}\alpha_i = \sum_{i=1}^{d} \beta_i  \text{ AND the first co-ordinates}\\
\text{$\alpha_i, \beta_i$ from the RIGHT, which are different, satisfy
  $\alpha_i > \beta_i$}
\end{cases}
\end{equation*}

\end{Definition}


\end{frame}
%%%%%%%%%%%%%%%%%%%%%%%%%%%%%%%%%%%%%%%%%%%%%%%%%%


%%%%%%%%%%%%%%%%%%%%%%%%%%%%%%%%%%%%%%%%%%%%%%%%%%
\begin{frame}{{\large Term Ordering Examples}}

$f = 2x^2yz + 3xy^3 - 2x^3$\\

\bi [<+->]
\item LEX with $x > y > z$, $f$ is:
\item $f = -2x^3 + 2x^2yz + 3xy^3$
\item  DEGLEX $x>y>z$:  $f$ is:
\item $f = 2x^2yz + 3xy^3 -2x^3$
\item DEGREVLEX $x>y>z$: $f$ is: 
\item $f  = 3xy^3 + 2x^2yz - 2x^3$
\ei

\pause

Always fix a term order over a ring, and stick to it!

\ \\
\pause
$f = c_1 X_1 + c_2 X_2 + \dots + c_t X_t$ implies $X_1 > \dots > X_t$
\end{frame}
%%%%%%%%%%%%%%%%%%%%%%%%%%%%%%%%%%%%%%%%%%%%%%%%%%

%%%%%%%%%%%%%%%%%%%%%%%%%%%%%%%%%%%%%%%%%%%%%%%%%%
\begin{frame}{{\large Multi-variate Division}}

Divide $f = y^2x+4yx - 3x^2$ by $g = 2y+x+1$ with DEGLEX $y>x$ in
$\Q[x, y]$

\pause
Solved on the board in the classroom

\end{frame}
%%%%%%%%%%%%%%%%%%%%%%%%%%%%%%%%%%%%%%%%%%%%%%%%%%

%%%%%%%%%%%%%%%%%%%%%%%%%%%%%%%%%%%%%%%%%%%%%%%%%%
\begin{frame}{{\large Multivariate Division}}

 Divide $f$ by $g$: denoted $f\stackrel{g}{\textstyle\longrightarrow}
 h$, where $h = f - \frac{X}{\text{lt}(g)} g$. Here, $X$ may not be
 the leading term. 

\begin{Definition}
Let $f, f_1, \dots, f_s, h \in \F[x_1, \dots, x_n], f_i ~~ \neq 0; ~~ F
= \{f_1, \dots, f_s\}$. Then $f$ reduces to $h$ modulo $F$:
\[
f\stackrel{F}{\textstyle\longrightarrow}_+ h
\]

if and only if there exists a sequence of indices $i_1, i_2, \dots,
i_t \in \{1, \dots, s\}$ and a sequence of polynomials $h_1, \dots,
h_{t-1} \in k[x_1, \dots, x_n]$ such that

\[
f\stackrel{f_{i_1}}{\textstyle\longrightarrow}h_1
\stackrel{f_{i_2}}{\textstyle\longrightarrow}h_2 
\stackrel{f_{i_3}}{\textstyle\longrightarrow} \dots
\stackrel{f_{i_{t-1}}}{\textstyle\longrightarrow}h_{t-1}
\stackrel{f_{i_{t}}}{\textstyle\longrightarrow}h
\]
\end{Definition}

$f_1 = yx - y, f_2 = y^2 - x, ~~~f = y^2x;$ DEGLEX $y>x$ in
$\Q[x]$. Divide $f \xrightarrow{f_1, f_2}_+ h$:

\end{frame}
%%%%%%%%%%%%%%%%%%%%%%%%%%%%%%%%%%%%%%%%%%%%%%%%%%

%%%%%%%%%%%%%%%%%%%%%%%%%%%%%%%%%%%%%%%%%%%%%%%%%%
\begin{frame}{{\large Multivar Division}}
\bi
\item To divide $f$ by $F =\{f_1, \dots, f_3\}$ (say)
\item Impose term order on the ring
\item Impose the given order on polynomials of $F$: $f_1 > f_2 > f_3$
\item Divide $f$ by $f_1$ first:
\bi
\item Analyze terms of $f = c_1X_1 + c_2X_2 + \dots + c_tX_t$ in order
\item Does $lt(f_1) ~|~ c_1X_1$? If so, divide (or cancel lt(f)),
  update $f$, and check if $lt(f_1) ~|~$ the new $lt(f)$ (in updated $f$)?
\item Otherwise, does $lt(f_2) ~|~ c_1X_1$? And so on...
\ei
\item If $lt(f)$ is not divisible by any $lt(f_i)$, then move $lt(f)$
  into the remainder ($r = r + lt(f)$), and update $f$ ($f = f - lt(f)$)
\item Repeat... [See the algorithm in the next slides]
\ei
\end{frame}
%%%%%%%%%%%%%%%%%%%%%%%%%%%%%%%%%%%%%%%%%%%%%%%%%%


%%%%%%%%%%%%%%%%%%%%%%%%%%%%%%%%%%%%%%%%%%%%%%%%%%
\begin{frame}{{\large Multivar Division}}

\begin{Definition}
If $f\stackrel{F}{\textstyle\longrightarrow}_+ r$, then no term in $r$
is divisible by $\text{LT}(f_i), ~ \forall f_i \in F$. Then $r$ is
reduced w.r.t. $F$ and it is called the remainder.
\end{Definition}

\begin{Definition}
Let $f, f_1, \dots, f_s, r \in \F[x_1, \dots, x_n], f_i ~~ \neq 0; ~~ F
= \{f_1, \dots, f_s\}$. Then $f$ reduces to $r$ modulo $F$:
\[
f\stackrel{F}{\textstyle\longrightarrow}_+ r
\]
then
\[
f = u_1 f_1 + \dots + u_s f_s + r
\]
and we have that:
\begin{itemize}

\item $r$ is reduced w.r.t. $F$
\item $u_1, \dots u_s \in \F[x_1, \dots, x_n]$
\item $\text{LP}(f) = \text{MAX}(\text{LP}(f_1)\text{LP}(u_1), \dots
  \text{LP}(f_s)\text{LP}(u_s), r)$
\end{itemize}
\end{Definition}

\end{frame}
%%%%%%%%%%%%%%%%%%%%%%%%%%%%%%%%%%%%%%%%%%%%%%%%%%




%%%%%%%%%%%%%%%%%%%%%%%%%%%%%%%%%%%%%%%%%%%%%%%%%%
\begin{frame}{{\large Multvariate Division Algorithm }}

 \begin{algorithm}[H]
 \caption{Multivariate Division of $f$ by $F=\{f_1,\dots,f_s\}$}
 \label{algo:mv_reduce}
 \begin{algorithmic}[1]

 \REQUIRE $f, f_1, \dots, f_s \in \F[x_1, \dots, x_n], f_i\neq 0$
 \ENSURE $u_1,\dots, u_s, r$ s.t. $f = \sum f_i u_i+r$ where $r$ is
 reduced w.r.t. $F = \{f_1,\dots, f_s\}$ and max($lp(u_1)lp(f_1), \dots, lp(u_s)lp(f_s), lp(r)$) = $lp(f)$
 \STATE{ {$u_i \gets 0; ~r \gets 0, ~h \gets f $} }
 \WHILE {{  ($h \neq 0$ )}}
 \IF{{ $\exists i$ s.t. $lm(f_i) ~|~ lm(h)$}}
 \STATE{{choose $i$ least s.t. $lm(f_i) ~|~ lm(h)$}}
 \STATE{{ $u_i = u_i + \frac{lt(h)}{lt(f_i)}$}}
 \STATE{{ $h = h - \frac{lt(h)}{lt(f_i)} f_i$}}
 \ELSE
 \STATE {{$r = r+ lt(h)$}}
 \STATE {{$h = h - lt(h)$}}
 \ENDIF
 \ENDWHILE
 \end{algorithmic}
 \end{algorithm}
\end{frame}


%%%%%%%%%%%%%%%%%%%%%%%%%%%%%%%%%%%%%%%%%%%%%%%%%%
\begin{frame}{{\large Motivate Gr\"obner basis}}

Let $F = \{f_1, \dots, f_s\}; ~~ J = \langle f_1, \dots, f_s \rangle$
and let $f \in J$. Then we should be able to represent $f = u_1 f_1 +
\dots + u_s f_s + r$ where $r = 0$. If we were to divide $f$ by $F =
\{f_1, \dots, f_s\}$, then we will obtain an intermediate remainder
(say, $h$) after every one-step reduction. The leading term of every
such remainder (LT($h$)) should be divisible by the leading term of
at least one of the polynomials in $F$. Only then we will have $r = 0$.

\begin{Definition}
Let $F = \{f_1, \dots, f_s\}; ~G = \{g_1, \dots, g_t\};$\\
$ J = \langle f_1, \dots, f_s \rangle = \langle g_1, \dots, g_t
\rangle$.  Then $G$ is a {\bf Gr\"{o}bner Basis} of $J$
\[ ~\iff
\]
\[
\forall f \in J ~~(f\neq 0), ~~~ \exists i : \text{LM}(g_i) ~|~ \text{lm}(f)
\]

\end{Definition}


\end{frame}
%%%%%%%%%%%%%%%%%%%%%%%%%%%%%%%%%%%%%%%%%%%%%%%%%%

%%%%%%%%%%%%%%%%%%%%%%%%%%%%%%%%%%%%%%%%%%%%%%%%%%
\begin{frame}{{\large Gr\"obner Basis}}

\begin{Definition}
$G = \{g_1, \dots, g_t\} = GB(J) \iff \forall f \in J, \exists g_i
~~\text{s.t. } lm( g_i ) ~|~ lm( f )$
\end{Definition}

\ \\
\begin{Definition}
$G = GB(J) \iff \forall f \in J, f \xrightarrow{g_1, g_2, \cdots,
  g_t}_+0$ 
\end{Definition}
Implies a ``decision procedure'' for ideal membership

\end{frame}
%%%%%%%%%%%%%%%%%%%%%%%%%%%%%%%%%%%%%%%%%%%%%%%%%%


\begin{frame}{\large{Buchberger's Algorithm Computes a Gr\"obner Basis}}

%{\small
%\begin{algorithm}
%\caption {Buchberger's Algorithm}
{\bf Buchberger's Algorithm}\\
%\label{alg:gb}
%\begin{algorithmic}
% \REQUIRE : $F = \{f_1, \dots, f_s\}$
 INPUT : $F = \{f_1, \dots, f_s\}$\\
% \ENSURE : $G = \{g_1,\dots ,g_t\}$\\ %, a Gr\"{o}bner basis \\
 OUTPUT : $G = \{g_1,\dots ,g_t\}$\\ %, a Gr\"{o}bner basis \\
  $G:= F$; \\
  REPEAT\\
  \hspace{0.1in} $G' := G$\\
  \hspace{0.1in} For each pair $\{f, g\}, f \neq g$ in $G'$ DO\\
\hspace{0.2in}  $S(f, g) \stackrel{G'}{\textstyle\longrightarrow}_+
r$ \\
\hspace{0.2in}  IF $r \neq 0$ THEN $G:= G \cup \{r\}$ \\
%\hspace{0.2in}  $G(x):=G(x) / x$
UNTIL $G = G'$
%\end{algorithmic}
%\end{algorithm}

\[
S(f,g)=\frac{L}{lt(f)}\cdot f - \frac{L}{lt(g)}\cdot g
\]
$L = \text{LCM}(lm(f), lm(g))$, ~~$lm(f)$: leading monomial of $f$
%}

\end{frame}


\end{document}
