\documentclass{article}
\usepackage{algorithm}
\usepackage{algpseudocode}
\usepackage{svg}
\usepackage{xcolor}
\usepackage{amsthm}
\usepackage{float}
\usepackage{sectsty}
\usepackage{amsbsy}
\usepackage{amsthm}
\usepackage{amsmath}
\usepackage{amsfonts}
\usepackage{graphicx}
\usepackage{multirow}
\usepackage{diagbox}
\usepackage{bm}
\usepackage{hhline}
\usepackage{graphicx}
\usepackage{helvet}
\usepackage{enumerate}
\usepackage{amsmath}
\usepackage{amsfonts}
\usepackage{graphicx}
\usepackage{multirow}
\usepackage{subfig}
\usepackage{comment}
\usepackage{mathtools}
\bibliographystyle{plain}

\newtheorem{Algorithm}{Algorithm}[section]
\newtheorem{Definition}{Definition}[section]
\newtheorem{Example}{Example}[section]
\newtheorem{Proposition}{Proposition}[section]
\newtheorem{Lemma}{Lemma}[section]
\newtheorem{Theorem}{Theorem}[section]
\newtheorem{Corollary}{Corollary}[section]
\newtheorem{Proof}{Proof}

\newcommand{\B}{{\mathbb{B}}}
\newcommand{\Z}{{\mathbb{Z}}}
\newcommand{\R}{{\mathbb{R}}}
\newcommand{\Q}{{\mathbb{Q}}}
\newcommand{\N}{{\mathbb{N}}}
\newcommand{\C}{{\mathbb{C}}}
\newcommand{\Zn}{{\mathbb{Z}}_{n}}
\newcommand{\Zp}{{\mathbb{Z}}_{p}}
\newcommand{\F}{{\mathbb{F}}}
\newcommand{\Fbar}{{\overline{\mathbb{F}}}}
\newcommand{\Fq}{{\mathbb{F}}_{q}}
\newcommand{\Jo}{J_1}
\newcommand{\Jz}{J_0}
\newcommand{\Jt}{J_2}
\newcommand{\Joz}{J_1 + J_0}
\newcommand{\Jtz}{J_2 + J_0}
\newcommand{\Fqbar}{{\overline{{\mathbb{F}}_q}}}
\newcommand{\Fkk}{{\mathbb{F}}_{2^k}}
\newcommand{\Zkk}{{\mathbb{Z}}_{2^k}}
\newcommand{\Fkkx}[1][x]{\ensuremath{\mathbb{F}}_{2^k}[#1]\xspace}
\newcommand{\Grobner}{Gr\"{o}bner\xspace}
\newcommand{\bi}{\begin{itemize}}
\newcommand{\ei}{\end{itemize}}

\newcommand{\idealj}{{J = \langle f_1,f_2 \dots, f_s\rangle}}
\newcommand{\idealg}{{J = \langle g_1, \dots, g_t\rangle}}
\newcommand{\vfqj}{{V_{\Fq}(J)}}
\newcommand{\vfqjo}{{V_{\Fq}(J_0)}}
\newcommand{\vfbqj}{{V_{\overline{\Fq}}(J)}}
\newcommand{\vfbqjo}{{V_{\overline{\Fq}}(J_0)}}
\newcommand{\vfbqjjo}{{V_{\overline{\Fq}}(J+J_0)}}
\newcommand{\vfkkj}{{V_{\Fkk}(J)}}
% \newcommand{\v}{\vee}
\newcommand{\acf}{\bar{F}_q}
\newcommand{\Vacf}{V_{\bar{F}_q}}

\sectionfont{\large}

\title{Operations on Radical Ideals}
\begin{document}

\maketitle
\section{preliminaries}
Let $\Fqbar$ be the algebraic closure of $\Fq[x_1,..x_n]$. Let's consider $J_0 = \langle x_1^q-x_1..x_n^q-x_n \rangle$ as the set of all vanishing polynomials over this closure. Any arbitrary ideals $\Jo = \langle f_1,...,f_s\rangle$ \textit{and} $\Jt = \langle h_1,...,h_r \rangle$ when coupled with the vanishing polynomials, restrict the solutions to finite fields. The ideals, $\Joz$ and $\Jtz$ then become radical -
\begin{align*}
 \Joz = \sqrt{\Joz}\\
 \Jtz = \sqrt{\Jtz}
\end{align*}

\section{Intersection of Radical Ideals}
Definition 8 and proposition 9 from Chapter 4 $\cite{ideals:book}$ gives us the understanding of ideals with respect to intersection operation for finite fields.\\ 

\begin{Theorem}
\textit{If} $\Jo$ \textit{and} $\Jt$ \textit{are any given ideals in} $\Fq[x_1,...,x_n]$, \textit{ then} $\Jo \cap \Jt$ \textit{is also an ideal which is defined as the set of polynomials which belong to both} $\Jo$ \textit{and} $\Jt$.
\begin{align}
 \Jo \cap \Jt = \{ g : g \in \Jo\ and\ g \in \Jt\}  
\end{align}
\end{Theorem}

For any two Radical ideals $\Joz$ and $\Jtz$, intersection operation is defined as
\begin{align}
 (\Joz) \cap (\Jtz) = (\Jo \cap \Jt) + \Jz
\end{align}

\begin{proof}
We will prove this theorem by proving the inclusion in both the directions.
Let \textit{f} be any arbitrary polynomial such that
\begin{align}
\textit{f} \in (\Joz) \cap (\Jtz)
\end{align} 

\quad from (1),
\begin{align*}
f \in \Joz\ and\ f \in \Jtz\\
\end{align*}
Let \textit{t} be an extra variable added to the ring $\Fq[x_1,..x_n]$ -
\begin{gather*}
t.f \in t(\Joz)\ and\ (1-t)f \in (1-t)(\Jtz)\\
f = tf + (1-t)f \in t(\Joz) + (1-t)(\Jtz)\\
f = tf + (1-t)f \in t(\Jo) + t(\Jz) + \Jt + \Jz -t(\Jt)-t(\Jz)\\
\end{gather*}

\begin{align}
f = tf + (1-t)f \in t(\Jo) + (1-t)(\Jt) + \Jz
\end{align}

From chapter 4, Theorem 11$\cite{ideals:book}$, for any two given ideals $\Jo$ and $\Jt$ in $\Fq[x_1,...x_n]$ -
\begin{align}
\Jo\cap \Jt = (t\Jo+ (1-t)\Jt) \cap \Fq[x_1,...,x_n]
\end{align}

Adding field intersection to equation (4) - 
\begin{align}
f \in ((t(\Jo) + (1-t)(\Jt))\cap \Fq[x_1,...,x_n]) + (\Jz \cap \Fq[x_1,...,x_n])
\end{align}

From equations (5) and (6) -
\begin{align}
\textit{f} \in (\Jo\cap\Jt)+\Jz
\end{align}

To prove the inclusion in other direction, let's take \textit{f} to be any arbitrary polynomial such that
\begin{align}
\textit{f} \in (\Jo\cap\Jt)+\Jz
\end{align}

from equations (5) and (6)
\begin{gather*}
f \in ((t\Jo+ (1-t)\Jt) \cap \Fq[x_1,...,x_n]) + (\Jz \cap \Fq[x_1,...,x_n])\\
f \in ((t\Jo+ (1-t)\Jt) \cap \Fq[x_1,...,x_n]) + \Jz
\end{gather*}

assigning \textit{t} = 0
\begin{gather*}
f \in ((0*\Jo+ (1-0)\Jt) \cap \Fq[x_1,...,x_n]) + \Jz\\
f \in (\Jt \cap \Fq[x_1,...,x_n]) + \Jz
\end{gather*}
From chapter 4, Lemma 10$\cite{ideals:book}$ -
\begin{align}
f \in (\Jt + \Jz)
\end{align}

assigning \textit{t} = 1
\begin{gather*}
f \in ((1*\Jo+ (1-1)\Jt) \cap \Fq[x_1,...,x_n]) + \Jz\\
f \in (\Jo \cap \Fq[x_1,...,x_n]) + \Jz
\end{gather*}
\begin{align}
f \in (\Jo + \Jz)
\end{align}


From theorem 1.1, and using equations (9) and (10)
\begin{align*}
\textit{f} \in (\Joz) \cap (\Jtz)
\end{align*} 
\end{proof}

\section{Product of Radical Ideals}
\begin{Theorem}
\textit{If} $\Jo = \langle f_1,...,f_s\rangle$ \textit{and} $\Jt = \langle h_1,...,h_r \rangle$ \textit{are any given ideals in} $\Fq[x_1,...,x_n]$, \textit{ then their product denoted} $\Jo.\Jt$, \textit{is defined to be the ideal generated by all polynomials f.g where f $\in \Jo\ and\ g \in \Jt$.}
\begin{align*}
\Jo.\Jt = \{ f_1h_1+..+f_sh_r : f_i \in \Jo\ and\ h_i \in \Jt\}  
\end{align*}

The generators of which are given as 
\begin{align}
\Jo.\Jt = \langle f_ih_j : 1 \le i \le s , 1 \le j \le r\rangle
\end{align}
\end{Theorem}

For any two product ideals $\Jo$ and $\Jt$ coupled with vanishing polynomials(radical), the product intersection relation is defined as
\begin{align*}
 (\Jo.\Jt)+\Jz = (\Jo \cap \Jt) + \Jz
\end{align*}
\begin{proof}

We know that for any two arbitrary ideals $\Jo$ and $\Jt$ -
\begin{align}
\sqrt{\Jo.\Jt} = \sqrt{\Jo\cap \Jt}
\end{align}

To prove the equality, we shall prove the inclusion in both directions. Let \textit{f} be any arbitrary polynomial such that
\begin{align*}
f \in \sqrt{(\Jo.\Jt)}
\end{align*}

Given the generators of \textit{f} from (11), we can easily see that it is composed of generator sets from both $\Jo$ and $\Jt$, hence
\begin{align*}
f \in \sqrt{\Jo}\ and \ f \in \sqrt{\Jt}
\end{align*}

From the definition of radical ideal, there exists some power of \textit{q} such that
\begin{align*}
f^q \in \Jo\ and \ f^q \in \Jt\\
f^q \in \Jo \cap \Jt\\
f \in \sqrt{\Jo \cap \Jt}
\end{align*}

To prove the inclusion in other direction, let's take \textit{f} to be any arbitrary polynomial such that
\begin{align*}
f \in \Jo \cap \Jt\\
f^q \in \sqrt{\Jo\cap\Jt}\\
f^q \in \sqrt{\Jo}\  and \ f^q \in \sqrt{\Jt}\\
f^{2q} \in \Jo .\Jt\\
f \in \sqrt{\Jo .\Jt}
\end{align*}

Now over finite fields, equation (12) can be written as - 
\begin{align}
 (\Jo.\Jt)+\Jz = (\Jo \cap \Jt) + \Jz
\end{align}
Since adding the vanishing polynomials make it radical.

\end{proof}

\bibliography{vikas}

\end{document}