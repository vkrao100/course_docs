\section{Introduction}

Automated formal verification and equivalence checking of arithmetic
datapath circuits is challenging. Conventional verification
techniques, such as those based on binary  decision diagrams (BDDs)
\cite{BRYA86}, And-Invert-Graph (AIG) based reductions with SAT or
SMT-solvers \cite{alanmi:cec:iccad2006}, etc., are infeasible in
verifying complex datapath designs. Such designs often implement
algebraic computations over bit-vector operands, therefore finite
integer rings \cite{wienand:cav08} or finite fields \cite{pruss:tcad}
are considered appropriate models to devise decision
procedures for verification. For this reason, the verification
community has explored the use of algebraic geometry and symbolic
algebra algorithms for verification. In such a setting, the circuit is
modeled by way of a set of polynomials that generate an ideal, and the
verification problem is formulated using \Grobner basis (GB) reduction techniques \cite{gb_book}. 

The GB problem exhibits high computational complexity. 
%In a general
%setting, the complexity is doubly-exponential in the input data,
%whereas it is single exponential for polynomial ideals that have
%a finite number of solutions, such as those
Indeed, computing a GB (using Buchberger's
\cite{buchberger_thesis} or the $F_4$ algorithm \cite{f4}) for large
circuits is practically infeasible. Managing this complexity ought be
a major goal of any approach.

\underline{\it State-of-the-art \& Limitations:} Recent
approaches \cite{wienand:cav08} \cite{lv:tcad2013} have discovered that
particularly for circuit verification problems, the expensive GB
computation can be avoided altogether. For arbitrary combinational
\cite{wienand:cav08} \cite{lv:tcad2013} and sequential circuits
\cite{xiaojun:hldvt2016}, a specialized term order 
$>$ can be derived by analyzing the topology of the given 
circuit. This term order is derived by performing a reverse
topological traversal of the circuit, and in this manuscript we refer
to it as the {\it Reverse Topological Term Order} (RTTO). 
Imposition of RTTO $>$ on the polynomial ring
{\it renders the set of polynomials of the circuit itself a
GB.} Subsequently, the verification problems can be solved solely by
way of GB-reduction (using multi-variate polynomial division), without
any need to explicitly compute a GB. The techniques of
\cite{wienand:cav08} \cite{lv:tcad2013} have been extended and
improved further to verify integer arithmetic circuits. For instance,
\cite{ciesielski:dac2015} and \cite{rolf:date16} get more insights
from the circuit structure that  dictate specific rules on the order of
polynomials chosen in GB-reduction -- by accounting for topological
levels, reconvergent fanouts, AND-XOR gates with common inputs, etc. The authors in~\cite{cunxi:aspdac17} show that the
reduction process can be parallelized by performing reduction for each output bit independently. 

A common theme among all these relevant works is that {\it they move
  the complexity of verification from one of computing a GB to that of 
  GB-reduction (multivariate polynomial division).} These
%aforementioned approaches 
will benefit greatly by a dedicated, domain-specific implementation of
GB-reduction carried out on the given circuit under RTTO $>$. So far,
the above techniques 
\cite{wienand:cav08,pruss:tcad,lv:tcad2013,rolf:date16,ciesielski:dac2015,cunxi:aspdac17} use a
general-purpose polynomial division approach, together with explicit
set representation, for this GB-reduction. While some of these
approaches %\cite{ciesielski:dac2015,rolf:date16} 
do perform the reduction in some specific ways -- e.g., mimicking
GB-reduction under RTTO $>$ by substitution \cite{rolf:date16}, or using
TEDs to perform input-output signature comparisons \cite{ciesielski:dac2015},
or the use of $F_4$-style GB-reduction on a coefficient matrix
\cite{lv:tcad2013} -- the overall concept of polynomial division is
still utilized in its rudimentary form, involving iterative
cancellation of monomials ``1-step at a time'' on explicit
data-structures. We show in the sequel, that despite recent efforts,
such GB-reductions can still lead to {\it a worst-case size explosion
  problem.} 


\underline{\it Proposed Solution:} To make this GB-reduction on
circuits more efficient, this paper describes new techniques and
implementations, specifically targeted for
circuit verification under RTTO $>$.  In
particular, we make use of implicit characteristic set representation
of Zero-Suppressed BDDs \cite{zbdd}. By analyzing the structure of
ZBDDs for polynomial representation under RTTO $>$, we show how this
GB-reduction can be efficiently implemented using algorithms that 
specifically manipulate the ZBDD graph, by interpreting Boolean
polynomial manipulation as the algebra of unate cube sets. 

\underline{\it Rationale:} The algebraic objects
used to model the polynomial ideals derived from digital circuits are
rings of Boolean polynomials.
%generally of two types: (i) multi-variate polynomial
%rings with coefficients from the finite fields ($\Fkk$) of $2^k$
%elements, i.e. $\Fkk[x_1, \dots, x_n]$; and (ii)  quotient rings of the
%type $\Zkk[x_1, \dots, x_n] \pmod{ \langle x_i^2 - x_i\rangle}$
%corresponding to the pseudo-Boolean function models; here 
%$\Zkk = \Z \pmod{ 2^k}$. Over circuits, both models incorporate
%computations over {\it Boolean   polynomials}: (i) $\Ftwo (\equiv \B)
%\subset \Fkk$; whereas (ii) also encompasses (pseudo-) Boolean
%computations $\pmod{ x^2 = x}$.
%; \cite{wienand:cav08} refer to it as ``arithmetic bit-level''. 
When Boolean functions are represented in $\F_2$ using AND/XOR
expressions, and that too as a canonical \Grobner basis, the
representation tends to explode. 
%(depicted later in Section \ref{sec:motiv}).  
Polynomial representations employed in computer
algebra tools, such as the {\it dense-distributive 
data-structure} of the {\sc Singular} computer algebra tool
\cite{DGPS}, are inefficient for this purpose. Since addition (mod 2) and
multiplication are equivalent to XOR and AND operations,
respectively, GB-reduction can be viewed as a specialized {\it AND/XOR
  Boolean function decomposition} problem. Clearly, implicit Boolean set
representations such as decision diagrams could be employed for this
purpose. The decision diagram of choice here is the ZBDD \cite{zbdd},
because of its power to represent and manipulate sparse combinatorial
problems -- particularly ``sets of combinations'' using the unate cube
set algebra framework. 
%Monomials of a Boolean polynomial can be
%interpreted as {\it unate cubes} -- i.e. products of literals in
%positive polarity. Each cube represents one combination, and each
%literal represents an object chosen in the combination. Thus,
%GB-reduction on circuits resembles a classical logic synthesis
%problem, justifying the use of ZBDDs.


\underline{\it Technical Contributions:} %First and foremost, 
We describe when and how the GB-reduction encounters a term-explosion
(exponential blow-up) under RTTO $>$, which cannot be easily overcome
by explicit representations. We show that ZBDDs can avoid this
exponential blow-up -- thereby justifying their use. We describe how
the rudimentary polynomial division algorithms, that iteratively
cancel one monomial in every step, can be implemented on ZBDDs under
RTTO $>$. Subsequently, {\it we show that RTTO $>$ imposes a special
structure on ZBDDs that allows us to cancel multiple monomials in
every step of polynomial division}, thus improving GB-reduction in
both space and time! 
%Moreover, we also implement the specialized
%GB-reduction approach of \cite{rolf:date16} using our ZBDD
%framework. 
Finally, experiments conducted on finite field arithmetic
(crypto-circuits) benchmarks show an order of magnitude improvement
using our implementation of GB-reduction.  

\underline{\it Relationship to prior work in Boolean Gr\"obner Basis:}
The symbolic algebra community has studied properties of Boolean
GB \cite{michon:bool-ring2006} \cite{vardi-iasted07}
\cite{polybori:2009}. From among these, the work of PolyBori
\cite{polybori:2009} comes closest to ours, and is a source of
inspiration for this work. PolyBori proposed the use of ZBDDs to
compute Gr\"obner bases for Boolean polynomials. PolyBori is a {\it
  generic} Boolean GB computational engine that caters to many
permissible term orders. Its division algorithm is also based on the
conventional concept of canceling one monomial in every step of
reduction. In contrast, our algorithms are tailored for GB-reduction
under the RTTO $>$. The efficiency of our approach stems from the
observation that the RTTO $>$ imposes a special structure on the
ZBDDs, which allows for multiple monomials to be canceled in one
division-step. 


%% \underline{\it Paper Organization:} The following section describes
%% our notation and preliminary concepts. Section \ref{sec:limit}
%% describes the limitations of contemporary approaches and shows how
%% ZBDDs may overcome these. Section \ref{sec:alg} describes our
%% algorithms and implementations of GB-reduction on ZBDDs w.r.t the RTTO
%% $>$. Section \ref{sec:integer} describes how the recent methodologies
%% for integer multiplier circuits can be implemented and improved
%% computationally using ZBDDs. Experiments are described in Section
%% \ref{sec:exp}. Section \ref{sec:concl} concludes the paper. 
