\begin{abstract}

%Formal verification of arithmetic datapath circuits is a vital
%component of the hardware design flow. Conventional verification
%techniques based on SAT or SMT solvers are often infeasible for such
%applications. 
Recent developments in formal verification of arithmetic datapaths
make efficient use of symbolic computer algebra algorithms. 
The circuit is modeled as an ideal in polynomial rings, and  Gr\"obner
basis (GB) reductions are performed over these polynomials to derive a
canonical representation. As they model logic gates of the circuit,
the ideals comprise largely of Boolean (or pseudo-Boolean)
polynomials. GB reductions modulo these Boolean polynomials tend to
cause intermediate expression swell (term explosion problem) -- often
rendering the approach infeasible in a practical setting. 
These problems can be overcome by using an implicit data-structure
that represents these polynomials compactly, provides the algorithmic
flexibility to perform GB-reduction implicitly, and allows to exploit
the information derived from the topology of the circuits to improve
symbolic computation. 

This paper considers a logic synthesis analogue of GB reductions over
Boolean polynomials, by interpreting symbolic algebra  as the unate
cube set algebra over characteristic sets. By representing Boolean
polynomials as characteristic sets using Zero-suppressed BDDs (ZBDDs),
implicit algorithms are efficiently designed for GB-reduction for
datapath circuits. While polynomial manipulation algorithms have been
implemented on ZBDDs before, we show \textcolor{red}{that the imposition} of
circuit-topology based monomial orders exposes a special structure on
the ZBDD representation of the polynomials. The subexpressions
employed in the GB-reduction are readily visible as subgraphs on the
ZBDDs, which are directly used to compose the result. Our division
algorithms effectively cancel multiple monomials implicitly in
one-step, simplify the search for divisors, and avoid intermediate size
explosion. Experiments performed over various \textcolor{red}{finite
  field} arithmetic architectures demonstrate the efficiency of our
algorithms and implementations; our approach is orders of magnitude
faster as compared to conventional methods.  
\end{abstract}
