\begin{abstract}
% Automatic bug correction is a tedious and resource intensive process. 
Automatic correction of unknown components is a resource intensive process. Recent developments in resolving these unknown gates rely on incremental SAT solving. Despite using state-of-the-art SAT solvers, these approaches fail to verify finite field multipliers beyond 12-bits and hence are infeasible in a practical setting. Current formal datapath verification methods using symbolic computer algebra concepts rely heavily on the textbook structure of the circuits to realize an unknown component. These approaches
model circuit as a set of polynomials over integer rings, and use function extraction, simulation, and term rewriting using coefficient computation to arrive at a solution. The approach is not complete in the sense that the procedure cannot be extended to random logic circuits and finite field circuits due to ambiguities in coefficient calculation. The approach also fails to verify circuits with redundant gates. To overcome these limitations, this paper describes a formal approach using finite field theory to automatically realize the function implemented by an unknown component, and verify the same. The paper describes theory on resolving a single unknown component using ideal membership testing and \Grobner basis based reduction. We go onto pose the problem as a synthesis challenge and try to find multiple solutions to the unknown component using quotient of ideals concept. The paper also presents results on some preliminary experiments performed over finite field multipliers to compare efficiency of our approach against recent approaches. 
\end{abstract}