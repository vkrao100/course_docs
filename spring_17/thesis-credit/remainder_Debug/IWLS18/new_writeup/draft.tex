\section{Theory and Procedure}
Consider a specification polynomial $f$ and its circuit implementation $C$, modeled as polynomials $F = \{f_1,\dots,f_s\}\in \mathbb{F}_q[x_1,\dots, x_n]$. Generator of polynomials is given as $J=\langle F \rangle$, while $J_0$ is the set of all vanishing polynomials. Let us consider RTTO$>$%(~\ref{def:rtto})
 for the circuit. We will assume $f_i:1\le i \le s$ to be the unknown component which is of the special form:
\begin{gather} 
\label{fiform}
f_i = x_i + P
\end{gather}

where $x_i$ is the leading monomial, and $P$ is the tail representing desired solution in variables$:x_j \text{ s.t. } x_i>x_j$ in the order. 

For a correct implementation, specification $f$ should vanish on the variety of ideal generated by the circuit polynomials i.e., $f$ will be in the ideal generated by the circuit:

\begin{equation}
\label{member}
f \in J + J_0; 
f \in \langle f_1,..,f_s\rangle + \langle x_l^q-x_l\rangle;1\le l \le n
\end{equation}
% $f = h_1f_1 + h_2f_2 +\dots+h_if_i+\dots+h_sf_s+H\langle x_i^q-x_i\rangle$
Using Ideal membership testing, we can rewrite $f$ in terms of its generators as:

$f = h_sf_s + h_{s-1}f_{s-1} +\dots+h_if_i+\dots+h_1f_1+\sum\limits_{l=1}^{n}H_l\langle x_l^q-x_l\rangle$

where $H_l$ are arbitrary elements from $\Fq$.\\
From equation ~\ref{fiform}:

% $f = h_1f_1 + h_2f_2 +\dots+h_ix_i+h_iP+\dots+h_sf_s+H\langle x_i^q-x_i\rangle$
{\small$f = h_sf_s + h_{s-1}f_{s-1} +\dots+h_ix_i+h_iP+\dots+h_1f_1+\sum\limits_{l=1}^{n}H_l\langle x_l^q-x_l\rangle$}

% Given the RTTO $>$, we know the polynomials from $f_s,\dots,f_{i+1}$ and the leading term of $f_i$. Using algorithm:(\ref{algo:mv_reduce}) to reduce 
% $f - h_1f_1 - h_2f_2 -\dots-h_ix_i = h_iP+\dots+h_sf_s+H\langle x_i^q-x_i\rangle$
$f - h_sf_s -\dots-h_ix_i = h_iP+\dots+h_1f_1+\sum\limits_{l=1}^{n}H_l\langle x_l^q-x_l\rangle$

% $f - h_1f_1 - h_2f_2 -\dots-h_ix_i \in \langle h_i,f_{i+1},\dots,f_s, x_i^q-x_i\rangle$
$f - h_sf_s -\dots-h_ix_i \in \langle h_i,f_{i-1},\dots,f_1, x_l^q-x_l\rangle$\\
We shall call the intermediate remainder computed on the left hand side as $g$.
\begin{equation}
g \in \langle h_i,f_{i-1},\dots,f_1, x_l^q-x_l\rangle
\end{equation}
Given polynomials $h_i, g, f_{i-1},\dots,f_1$, we compute $h_i^{'}=P$ such that:

 $g = h_i^{'}h_i+h_{i-1}^{'}f_{i-1}+\dots+h_1^{'}f_1+\sum\limits_{l=1}^{n}H_l^{'}\langle x_l^q-x_l\rangle$

The computed $h_i^{'} = P$ is a solution to the function implemented by the unknown gate. This linear combination computation is done using $lift$
% (~\ref{lem:imt}) 
 implementation in SINGULAR\cite{DGPS_410}.

\subsection{computing all solutions space of $P$}
% This $GB$ will have one and only one polynomial which is of the form $x_i + \mathcal{F}(x_j)$, where $\mathcal{F}$ is the function implemented by the gate, and is the most desired solution. 

Since there are multiple possible solutions, it is given that $P$ is not unique. We can explore more such solutions which might satisfy the unknown component functionality. Given $P$ as one of the solutions, under RTTO$>$ we have:

$g = P*h_i+h_{i-1}^{'}f_{i-1}+\dots+h_1^{'}f_1+\sum\limits_{l=1}^{n}H_l^{'}\langle x_l^q-x_l\rangle;$\\
Since $g$ can be computed as any linear combination of polynomials, we can rewrite the equation as:\\
$P*h_i+h_{i-1}^{'}f_{i-1}+\dots+h_1^{'}f_1+\sum\limits_{l=1}^{n}H_l^{'}\langle x_l^q-x_l\rangle = P^{'}*h_i+h_{i-1}^{''}f_{i-1}+\dots+h_1^{''}f_1+\sum\limits_{l=1}^{n}H_l^{''}\langle x_l^q-x_l\rangle$;\\
Rearranging the terms:

$(P-P^{'})h_i = (h_{i-1}^{'}-h_{i-1}^{''})f_{i-1}+\dots+(h_{1}^{'}-h_{1}^{''})f_{1}+(\sum\limits_{l=1}^{n}H_l^{'}-\sum\limits_{l=1}^{n}H_l^{''})x_l^q-x_l;$

$(P-P^{'})h_i \in \langle f_{i-1},\dots,f_1,x_l^q-x_l\rangle;$\\
By definition of Quotient of Ideals:
% (~\ref{def:quo}):
\vspace{0.1in}
\begin{equation}
\label{quotcomp}
P-P^{'} \in \langle f_{i-1},\dots,f_1,x_l^q-x_l\rangle:h_i;
\end{equation}

where $C = \langle f_{i-1},\dots,f_1,x_l^q-x_l\rangle:h_i$ represents Colon ideal.

Despite being a correct solution, any computed $P$ might not guarantee the solution to be in the immediate support variables of $f_i$ due to RTTO$>$. To determine a solution in immediate support variable set $x_j$ of $f_i$, we need to use an elimination term order
% (~\ref{def:elim}) 
for the variables $x_i$ followed by $x_j$. We compute a $GB$ of the colon ideal $GB(C)$ using this elimination order with the intermediate solution $P$ added as tail of $f_i$. 
There can be many $P^{'}$ which might satisfy the above membership test. We can pick any polynomial from the quotient operation, add the previous solution $P$ and compute a new $P^{'}$. All such $P^{'}$ computed are valid solutions and will satisfy the membership test with specification polynomial $f$.

We will also have cases, when $h_i$ ends up being a constant, in which case $lift$ returns $g$ itself as a solution $h_i^{'}$. To arrive at a implementable solution, we divide $h_i^{'}$ by the constant $h_i$(multiply the inverse of $h_i$) and reduce the result by rest of the input polynomials\{$f_{i-1},\dots,f_1$\}. 

\begin{align}
h_i^{'}*h_i^{-1}\xrightarrow[]{f_{i-1}}\xrightarrow[]{f_{i-2}}\dots\xrightarrow[]{f_1}P
\end{align}