\section{Conclusion and future work}
The paper presented an approach to resolve function implemented by a unknown component in finite field arithmetic circuits. We presented a procedure to systematically use \Grobner basis based reduction and ideal membership testing to arrive at a solution, such that the resulting logic function of the circuit conforms to the reference specification. The paper also discussed quotient of ideal concept to define multiple implementable solution set. The experiments showed that our approach is better in several orders of magnitude as compared to recent SAT based approaches, hence demonstrating the effectiveness of the underlying theory. The most desired solution which is in terms of immediate support variables of the unknown component relies on expensive \Grobner basis re-computation with a different term order. To avoid this overhead, as part of our future work, we would like to explore better heuristics to do a guided computation of solution set $P$, so as to arrive at a solution with specific form in desired variables. The current experiment set deals with one unknown component or sub-circuit and needs to be extended for multiple dependent bugs in a single cone. Also, identifying the bug location, which is the primary concern in the overall scope of automated debugging needs to be addressed as well. 