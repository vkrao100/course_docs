\section{Introduction}

Past few years have seen extensive investigations into formal
verification of arithmetic circuits. Circuits that implement
polynomial computations over large bit-vector operands are hard to
verify using methods such as SAT/SMT-solvers, decision diagrams,
etc. Recent techniques have investigated the use of polynomial algebra
and algebraic geometry techniques for their verification. These
include verification of integer arithmetic circuits
\cite{Armin2017ColumnWiseVO} \cite{maciej:2015} \cite{rolf:2016} and
also finite field circuits \cite{lv:tcad2013}
\cite{blueveri:fmsd}. While these are successful in proving
correctness or detecting the presence of bugs, the problem of
debugging and correction of arithmetic circuits has only just begun to
be addressed \cite{farimah:2017:1,farimah:2016}. 

In this paper, we address the problem of rectification of buggy finite
field arithmetic circuits. A specification model ({\it Spec}) is given
either as a  polynomial description $f_{}$ over a finite field, or as a
golden model of a finite field arithmetic circuit. The finite field
considered is the field of $2^k$ elements (denoted by $\Fkk$), where
$k$ corresponds to the operand-width (bit-vector word length). An
  implementation circuit $C$ is also given. Equivalence checking is performed between the {\it Spec} and the
  circuit $C$, and the presence of a bug is detected. No restrictions
  on the number, type, or locations of the bugs are assumed.

We perform error-diagnosis and a subset   $\mathcal{N}$ of the nets of
the circuit is identified as {\it potential rectification locations.}
Given the {\it Spec}, the buggy implementation circuit $C$, and the set
$\mathcal{N}$ of potential rectifiable locations, our objective is to determine
whether or not the buggy circuit can be rectified at {\it one
  particular net $x_i\in \mathcal{N}$}. This is called {\bf
  single-fix rectification} in literature \cite{Huang:DAC2011}. If a
single-fix rectification does exist at net $x_i$ in the buggy circuit,
then our subsequent objective is to derive a polynomial function
$U(X_{PI})$ in terms of the set of primary input variables
$X_{PI}$. This polynomial can be translated (synthesized) into a logic
subcircuit such that $x_i = U(X_{PI})$ acts as the rectification
function for the buggy circuit $C$ so that $C$ matches the
specification. 

Our techniques and algorithms are based on symbolic computer algebra
and algebraic geometry -- particularly on the concepts of the Strong
Nullstellensatz and Gr\"obner bases \cite{gb_book}. We show how to
apply our techniques to rectify finite field arithmetic circuits,
where conventional SAT-solver based rectification approaches are
infeasible.

The paper is organized as follows: The following section reviews
related previous work. Section III covers preliminary concepts. The
formulation of the verification test is described in Section
IV. Section V describes conditions for rectification at a particular
net. Section VI describes how rectification function can be
synthesized once single-fix rectification is deemed possible. Section
VII describes our experimental results and Section VIII concludes the
paper. 


\section{Previous Work}
Automated diagnosis and rectification of digital circuits
has been addressed in \cite{Madre:ICCAD89,Liaw:ICCAD90}. The paper \cite{Sadowska:DAC95} presents 
algorithms for synthesizing Engineering 
Change Order (ECO) patches - an analogous problem to rectification. 
The use of interpolation for ECO has been presented in \cite{Huang:DAC2011,roland:iccad10,Ling:tcad2011}.
The single-fix rectification function approach in \cite{roland:iccad10,Ling:tcad2011} 
has been extended in \cite{Huang:DAC2011} to generate multiple partial-fix functions while guaranteeing that 
the number of different minterm becomes smaller in each step. 
% The application of 
% interpolants in finite fields, presented in Section \ref{sec:app}, is motivated by the formulation of 
% circuit rectification problem and the use of interpolation in \cite{Huang:DAC2011}. 
As these approaches are SAT based, they work well for random logic 
circuits but are not efficient for arithmetic circuits. In contrast to 
these works, our work presents a word-level formulation for single-fix rectification.
Computer algebra has been utilized for circuit debugging and rectification in
\cite{maciej:2015:2,farimah:2017:1,farimah:2016}. These approaches rely heavily on
the arithmetic structure of the circuit for coefficient calculation. Moreover,
if the arithmetic circuits contain redundancies, then we have shown
that their approach is incomplete in that it cannot resolve the
rectification question. We have uploaded an appendix at
 \url{http://eng.utah.edu/~utkarshg/cex.pdf} for detailed discussion on these
issues.


%% the approach may not identify 
%% the buggy gate due to ambiguity in coefficient values. On the other hand,
%% our approach, although more efficient for finite field arithmetic circuits, 
%% is applicable to any circuit in general.

Once rectification is deemed feasible, the problem of finding the
rectification function has been considered as a partial synthesis
problem. The most recent and relevant
approach~\cite{fujita:2015},~\cite{fujita:2012} resolves the unknown
component problem using an incremental $SAT$ formulation. The paper
models the unknown component in a given circuit($Ckt$) as a LUT by
using transformation variables($X$). The solution to these variables
implements the desired logic function so that the resulting circuit
becomes logically equivalent to a given specification $Spec()$. Let
$Ckt(X,In)$ be the formula corresponding to the given circuit with
possible transformations, where $In$ is the set of all primary inputs
to the circuit. This can be formulated naturally as a two-level QBF
with an existential quantifier followed by a universal quantifier as
shown below: 
\vspace{0.1in}
\begin{align}
\exists \textit{X}.\forall \textit{In. Ckt(X,In) = Spec(In)}:    
\end{align}

The two level QBF is then solved by repeatedly applying a SAT solver. 
  
%% \begin{enumerate}
%%   \item Let Target=($Ckt(X,In)\neq spec(In))$. Let $k$ be the num-ber of test vectors, initialized to zero. Let $TestSet$ be the set of all generated test patterns, initialized to the empty set.
%%   \item Check if Target is satisfiable.
%%   \item If SAT, $k=k+1$ and record the solution as $TestSet = TestSet \cup in_k$. Update Target = (Target($X,In))\land(Ckt(X,in_k)=Spec(in_k))$, and then go to step 2.
%%     \item If UNSAT, we have all the required test set patterns $\{in_1$ $\dots in_k\}$. Now, check if: $(Ckt(X,in_1) = Spec(in_1)) \land (Ckt(X,in_2) = Spec(in_2)) \land \dots (Ckt(X,in_k) = Spec$ $(in_k))$ is satisfiable.
%%     \item If SAT, then any solution $X$ is a correct set of transformation, while an UNSAT result proves that there does not exist a correct set of transformation.
%% \end{enumerate}

% By experiment, the approach shows that if the circuit is correct under these input patterns($in_k$), it is guaranteed to be correct for all of $2^{In}$ input patterns.
The work in~\cite{maciej:2017} poses the unknown component formulation
as a camouflaged circuit model and tries to de-obfuscate several types
of camouflaging techniques using incremental SAT solving. The approach
used in~\cite{andreas:2005} inserts logic corrector MUXs on the
unknown sub-circuits and relies on SAT solvers to realize the
functionality. 

Despite using state-of-the-art SAT solvers, all the above approaches
fail to verify large and complex finite field arithmetic circuits. The
solvers still model the problem as decision procedures and, as
demonstrated by our experimental results, are shown to be inefficient
in solving verification problems on multiplier circuits beyond
12-bits.  

%% The technique from Farahmandi et al.~\cite{farimah:2016} deals with
%% automatic debugging and correction using computer algebra
%% concepts. The authors use function extraction~\cite{maciej:2015:1}
%% with a specific term order~\cite{lv} to do equivalence checking,
%% subsequently generating a remainder in case of failure. The approach
%% then finds all possible assignments to variables of the remainder such
%% that it generates a non-zero value. This test set helps arrive at a
%% pruned gate list for bug localization. The procedure then takes every
%% gate in the pruned list, starting from primary inputs, and tries to
%% match the appeared remainders pattern. It does so by computing the
%% difference between the polynomial computed at the output of the
%% suspicious gate against the polynomial computed by a probable set of
%% gate corrections. The coefficient computation~\cite{maciej:2015:2}
%% during pattern matching relies heavily on the half-adder based circuit
%% structure. The paper doesn't discuss the ambiguities in weight
%% calculations when the gate structure differs from the given
%% topology. The approach is not complete in the case when there are
%% redundant gates in the circuit as we found through our
%% experiments.

%% The authors of paper \cite{maciej:2018} present an approach to 
%% diagnose and rectify finite field multipliers using a forward topological 
%% order. The approach is topologically constrained for the finite field 
%% multipliers. The approach presented in this paper makes no such assumption 
%% about the circuit topology.

The authors in~\cite{scholl:2} present a QBF 
formulation for answering whether a partial implementation can be extended 
to a complete design that models a given specification. In contrast, the approach 
presented in this paper finds a single-fix rectification solution,
modelled as an unknown component in the circuit, after it is found
that an appropriate function for unknown component exists. 

%% We presented an approach to resolve the unknown component using \Grobner basis and ideal membership test. It also described a method to find multiple solutions for the unknown component. In this paper,
%% we have improved the previous approach and extended it by finding locations in the circuit where rectifications are possible. The authors in \cite {} describe a procedure to ascertain that the circuit can be rectified at a certain location, and find a rectification function for the circuit using interpolants in finite fields. The part of our approach for finding potential rectification location can be supplemental to their work.

% These drawbacks are resolved in our paper in addition to providng a complete solution to the unknown component problem.

 
