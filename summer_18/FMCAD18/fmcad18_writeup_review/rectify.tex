\section{Identification of the Rectification Target}

After the presence of a bug is detected, we address the problem of
single-fix rectification of $C$. In this section, we present an
approach that ascertains whether or not a single-fix rectification can
be applied at a given (target) net $x_i$ in $C$. In principle, our
approach can be applied at every net $x_i$ in $C$ to see if $C$ at all
admits single-fix rectification. However, it is possible to 
heuristically prune the search for these target nets $x_i$ by
analyzing the non-zero remainder obtained by the \Grobner basis
reduction $f\xrightarrow{F,F_{0}^{PI}}_+r$. We first briefly describe
an approach that we have utilized in this paper to construct a subset
$\mathcal{N}\subseteq \{x_1,\dots,x_n\}$ as possible rectification
targets. This rectification target pruning approach is inspired from
\cite{farimah:2016}. 
Then we address the rectification theorem and the search for
the rectification function --- the main contributions of this paper. 

\subsubsection{Potential rectification target nets} The
circuit $C$ has $k$-bit operands, and the output is expressed as $Z =
\sum_{i=0}^{k-1} z_i \alpha^i$. Then the non-zero remainder $r$ can 
be partitioned based on the coefficients of the monomials in $r$ and
re-expressed as: %according to: % powers of $\alpha$: 

\vspace{-0.1in}
\begin{align}
r = \alpha^0 (r_0) + \alpha^1 (r_1) + \dots + \alpha^{k-1} (r_{k-1})
\end{align}

Non-zero  terms $r_i$ (with coefficient $\alpha^i$) imply that the
effect of the bug is observable at the bit-level output $z_i$. We
consider the transitive fanin cones of logic of the output bits
$z_i$. When a bug affects multiple outputs, a single-fix rectification
might exist only at the nets that lie in the intersection of the
respective fanin-cones of the affected outputs. In our experiments, we
include these nets in $\mathcal{N}$ to check if any one of them admits
a single-fix rectification.

\begin{Example}
As shown in Ex. \ref{ex:1}, $f\xrightarrow{F,F_{0}^{PI}}_+r =
(\alpha+1)a_0a_1b_1b_0+(\alpha+1)a_0a_1b_1+(\alpha+1)a_1b_1b_0+(\alpha)a_1b_0$. We
re-write the remainder $r = \alpha^0 r_0 + \alpha^1 r_1 = 
\al\cdot (a_0a_1b_1b_0+a_0a_1b_1+a_1b_1b_0+a_1b_0) +
1\cdot (a_0a_1b_1b_0+a_0a_1b_1+a_1b_1b_0)$. Since both $r_0$ and $r_1$ are
non-zero, the bug affects both primary outputs $z_0, z_1$. By
identifying the nets that lie in the intersection of the fanin cones
of $z_0, z_1$, we construct $\mathcal{N} =
\{s_4,s_3,s_2,s_1,e_3,e_2,e_0\}$ as potential rectifiable locations.
\end{Example}

\subsubsection{Confirming a rectification target} 
After post-verification debugging is performed to identify a set of
nets $\mathcal{N}\subseteq \{x_1,\dots,x_n\}$ that are potential
rectification target nets, we now present an approach that {\it confirms}
whether or not the circuit can indeed be single-fix-rectified at net
$x_i$. {\it Single-fix-rectification at target net $x_i$ means that
  \underline{there exists} a polynomial function $U(X_{PI})$ which,
  when implemented at net $x_i$, ensures that the circuit $C$ would
  correctly implement the specification $f$.} Note that
$x_i=U(X_{PI})$ is a polynomial function of the type
$\F_2^{|X_{PI}|}\rightarrow\F_2$ as it implements a subcircuit at net $x_i$.

In the set of polynomials $F$, we replace $f_i = x_i+U(X_{PI})$ as the
polynomial for the rectification function at $x_i$, where $U(X_{PI})$
is a hitherto unknown/unresolved polynomial function component. In
other words, $F$ is updated to $F = \{f_1,\dots, f_{i-1}, f_i=
x_i+U(X_{PI}),f_{i+1},\dots,f_s\}$.  We state and prove the 
{\it rectification theorem that checks for the existence of
  $U(X_{PI})$ as a single-fix rectification function at $x_i$.}

\begin{Theorem}[Rectification Theorem]
Given the specification polynomial $f$, and the implementation circuit
$C$, derive RTTO $>$ to represent the polynomials. Using RTTO $>$,
construct two ideals:  
\bi
\item $J_L = \langle F_L\rangle$, where
  $F_L=\{f_1,\dots,f_{i-1},f_i=x_i+1,f_{i+1},\dots,f_s\}$;
\item $J_H = \langle F_H\rangle$, where
  $F_H=\{f_1,\dots,f_{i-1},f_i=x_i,f_{i+1},\dots,f_s\}$;
\ei
where the polynomials $f_1,\dots,f_{i-1},f_{i+1},\dots,f_s$ are the
same as in the generators of ideal $J$ (representing the circuit), and
$f_i$ is replaced with $f_i = x_i+1$ in $J_L$ and $f_i = x_i$ in $J_H$,
respectively. Perform the reductions: 
\bi
\item $f\xrightarrow{F_L, F_{0}^{PI}}_+r_L $
\item $f\xrightarrow{F_H,F_{0}^{PI}}_+r_H $
\ei

Let $V_{\Fq}(r_L), V_{\Fq}(r_H)$ denote the varieties of $r_L$ and
$r_H$, respectively, over the given field $\Fq$. Then the buggy
circuit $C$ admits a single-fix rectification at the net (gate 
output) $x_i$ if and only if $V_{\Fq}(r_L) \cup V_{\Fq}(r_H) =
\Fq^{|X_{PI}|} = V(J_0^{PI})$.
\end{Theorem}

\begin{proof}
As rectification at net $x_i$ makes the circuit $C$ match the
specification $f$, $f$ should vanish on $V(J)$. Thus, the
rectification condition can be equivalently stated as:
``$f$ vanishes on $V_{\Fq}(J) \iff V_{\Fq}(r_L) \cup V_{\Fq}(r_H) =
  \Fq^{|X_{PI}|}$.''

(i) To prove $\Rightarrow$: Let $x_{PI} \in \Fq^{|X_{PI}|}$ be an
assignment to the primary input variables of $C$. For every point
$x_{PI}$, there exists a corresponding assignment $x_{int}$ to the
rest of the variables of the circuit. For each primary input assignment,
the target net $x_i$ evaluates to either $x_i = 0$ or $x_i = 1$. When
$x_i = 0$, then $J_H$ vanishes on the point $(x_{PI},
x_{int})$. Likewise, when $x_i = 1$, $J_L$ vanishes on $(x_{PI},
x_{int})$. Since $f\xrightarrow{J_H,J_0}_+r_H$ and
$f\xrightarrow{J_L,J_0}_+r_L$, and $f$ vanishes on the point
$(x_{PI},x_{int})$, we obtain that either $r_H(x_{PI}) = 0$ or
$r_L(x_{PI}) = 0$. In other words, for every 
primary input assignment $x_{PI}$, either $r_L$ or $r_H$ vanish. This
implies that $V(r_L) \cup V(r_H) = \Fq^{|X_{PI}|} = V(J_0^{PI})$.

(ii) To prove ``$\Leftarrow$'': Say there exists an assignment to the
primary inputs $x_{PI} \in \Fq^{|X_{PI}|}$ such that $r_H$ vanishes on
$x_{PI}$, i.e. $r_H(x_{PI})=0$. Corresponding to $x_{PI}$, there exists
an assignment to the rest of the variables of the circuit $x_{int}$. 
As $f\xrightarrow{J_H, J_0}_+ r_H$, we have that $f$ is a member of the
ideal $J_H + J_0 + \langle r_H \rangle$. Therefore, when
$r_H(x_{PI})=0$, the ideal $J_H$ also vanishes on $(x_{PI},x_{int})$,
and $J_0$ by definition vanishes everywhere. This implies that
$f(x_{PI},x_{int})=0$. Similarly, the argument also holds that when
$r_L(x_{PI})=0$, then $f(x_{PI},x_{int})=0$. This proves that for all
primary inputs if $r_L$ and $r_H$ vanish, then $f$ vanishes too; and
that completes the proof.


  
\end{proof}

Note that the check ``Is $V_{\Fq}(r_L) \cup V_{\Fq}(r_H) =
\Fq^{|X_{PI}|} = V(J_0^{PI})?$'' can be performed as shown below,
where the union of varieties corresponds to the product of ideals.
\begin{align}
 V_{\Fq}(r_L) \cup V_{\Fq}(r_H)  =&V_{\Fq}(r_L\cdot r_H)\\
  V_{\Fqbar}(\langle r_L\cdot r_H \rangle+ J_0^{PI}) =&V_{\Fq}(\langle r_L\cdot r_H \rangle+ J_0^{PI})
\end{align}

Thus, to check for single-fix rectification at the net $x_i$, we need
to compute the \Grobner basis $G = GB(\{r_L\cdot r_H\}\cup F_{0}^{PI})$
and see if $G$ exactly equals $F_{0}^{PI}$.  

\begin{Example}
\label{ex:3}
Continuing with our running example, we demonstrate the rectification
checks at nets $e_3, s_1$. As the bug was introduced at $e_3$, it is
obvious that the circuit is rectifiable at $e_3$. For the
rectification check at $e_3$, we mark the polynomial $f_{10}$ for
modification:

\bi
\item $J_L = \langle F_L\rangle$, where $F_L=\{f_1,\dots, f_{10}=e_3+1,\dots, f_{16}\}$,
\item $J_H = \langle F_H\rangle$, where $F_H = \{f_1,\dots, f_{10}=e_3,\dots, f_{16}\}$.
\ei

Reducing the specification $f: Z+A\cdot B$ modulo these ideals, we get:
\bi
\item $r_L = f \xrightarrow[]{F_L, F_{0}^{PI}}_+{(\alpha+1)a_1b_1b_0+(\alpha+1)a_1b_1}$
\item $r_H = f \xrightarrow[]{F_H,F_{0}^{PI}}_+{(\alpha+1)a_1b_1b_0+(\alpha)a_1b_0}$
\ei
  
When we compute the \Grobner basis $G = GB(r_L\cdot r_H, F_{0}^{PI})$,
we obtain $G = \{a_0^2-a_0, a_1^2-a_1,b_0^2-b_0,b_1^2-b_1\}$,
corresponding to the ideal of all vanishing polynomials in primary
inputs. This implies the existence of a rectification function at
$e_3$. 

In fact, the rectification test also passes for the net $s_4$;
implying that the bug at $e_3$ can indeed be rectified at a different
gate which does not lie in the fanin cone of $e_3$. However, the
rectification test  fails at net $s_1$. When the problem is formulated by 
modifying the polynomial $f_{12}$ at net $s_1$, the corresponding
computation for $G = GB(r_L\cdot r_H, F_{0}^{PI})$ results in
$G=\{a_0^2-a_0,b_0^2-b_0,a_1^2-a_1,b_1^2-b_1, ~~a_1b_0, ~a_0a_1b_1 +
(\alpha)a_0a_1b_0\}$. Due to the presence of the last 2 polynomials, 
$G \neq F_0^{PI}$, and rectification is not possible at net $s_1$. 
\end{Example}
