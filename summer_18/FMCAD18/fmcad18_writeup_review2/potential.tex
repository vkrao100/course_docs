\section{The Search for Rectification Target Nets}

In principle, once a bug is detected by computing
$f\xrightarrow{f_1,\dots,f_s}_+r_{\neq 0}$, a correction based on $r$
can always be added at the primary outputs of the circuit. This is,
however, not very desirable as this may not help in identifying the
cause of the bug, and the added correction logic may unacceptably
increase the size of the circuit. Our objective is to identify a subset
$\mathcal{N} \subseteq \{x_1,\dots,x_n\}$ of nets closer to the
primary inputs of the circuit such that the circuit {\it may or may
  not be} rectifiable at the nets in $\mathcal{N}$. We present a
heuristic approach that analyzes the non-zero remainder of the
\Grobner basis reduction $f\xrightarrow{F,F_{0}^{PI}}_+r$, and
constructs the set $\mathcal{N}$. Should the circuit admit a
single-fix rectification (at any number of nets), our heurustic will
ensure that at least one of these rectifiable locations will be
present in $\mathcal{N}$. Once the set $\mathcal{N}$ is computed, we
describe techniques in the next section to unequivocally confirm
whether or not a net $x_i \in \mathcal{N}$ admits a single-fix
rectification so that $C$ matches the specification $f$. 

Note that as the circuit has $k$-bit operands, the output $Z =
\sum_{i=0}^{k-1} z_i \alpha^i$. Then the non-zero remainder $r$ can
also be re-expressed according to powers of $\alpha$:

\begin{align}
r = \alpha^0 (r_0) + \alpha^1 (r_1) + \dots + \alpha^{k-1} (r_{k-1})\\
\end{align}

Here $r_0,\dots, r_{k-1}$ are polynomials
