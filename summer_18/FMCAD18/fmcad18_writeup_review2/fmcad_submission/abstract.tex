{\bf Abstract:} {\it Formal verification of arithmetic circuits checks
  whether or not a 
gate-level circuit correctly implements a given specification
model. In cases where this equivalence check fails -- 
the presence of a bug is detected -- it is required to:
i) debug the circuit, ii) identify a set of nets (signals) where the
circuit might be rectified, and iii) compute the corresponding
rectification functions at those locations. This paper addresses the
problem of post-verification debugging and correction (rectification)
of finite field arithmetic circuits. The specification model and the
circuit implementation may differ at any number of inputs. We present
techniques that determine whether the circuit can be rectified at one
particular net (gate output) -- i.e. we address single-fix rectification.

Starting from an equivalence checking setup modeled as a polynomial
ideal membership test, we analyze the ideal membership residue to
identify potential single-fix rectification locations. Subsequently,
we use Nullstellensatz principles to ascertain if indeed a single-fix
rectification can be applied at any of these locations. If a
single-fix rectification exists, we derive a rectification function by
modeling it as the synthesis of an unknown component problem. Our
approach is based upon the \Grobner basis algorithm, which we use both
as a decision procedure (for rectification test) as well as a
quantification procedure (for computing a rectification
function). Experiments are performed over various finite field
arithmetic circuits that demonstrate the efficacy of our approach,
whereas SAT-based approaches are infeasible. 
}
