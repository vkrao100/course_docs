\section{Introduction}

Craig interpolation is a method to construct and refine abstractions
of functions. It finds application in formal verification of hardware
designs and software programs, in logic synthesis of Boolean
functions, and also as a tool in proof complexity theory. It is a
logical tool to extract concise explanations for the infeasibility of
a mutually inconsistent set of statements. Craig
\cite{craig-interpolate} showed that for a valid implication $A
\implies B$, where $A, B$ are first order formulae containing no free
variables, there is a formula $I$ such that $A \implies I$, $I
\implies B$ and the non-logical symbols of $I$ appear in both $A$ and
$B$. The formula $I$ is called the {\it Craig interpolant}, or
interpolant for short. As  
propositional logic also admits Craig interpolation, the formal
verification community has extensively investigated interpolants and
their computation from resolution proofs of CNF-SAT problems. In the
propositional logic domain, the concept is stated with a slight
modification.  

\begin{Definition}\label{def:ci}
Let $(A, B)$ be a pair of CNF formulae (sets of clauses) such that $A
\w B$ is unsatisfiable. Then there exists a formula $I$ such that: (i)
$A\implies I$; (ii) $I \w B$ is unsatisfiable; and (iii) $I$ refers
only to the common variables of $A$ and $B$, i.e. $Var(I) \subseteq
Var(A) \cap Var(B)$. The formula $I$ is called the {\bf interpolant}
of $(A,B)$. 
\end{Definition}

Given the pair $(A, B)$ and their refutation proof, a procedure called
the {\it interpolation system} constructs the interpolant in linear
time and space in the size of the proof \cite{McMillan:CAV03}. As the
abilities of SAT solvers for proof refutation have improved,
interpolants have been exploited as abstractions in various problems
that can be formulated as unsatisfiable instances, e.g. model checking
\cite{McMillan:CAV03}, logic synthesis \cite{roland:bidecomp},
etc. Their use as abstractions have also been replicated in other
(combinations of) theories \cite{McMillan:TACAS04}
\cite{Kapur:SIGSOFT06} \cite{Cimatti:TACAS08} \cite{Griggio:FMCAD11},
etc.  
%These concepts have been applied to various problems in
%automated reasoning. 
%; {\it e.g.} for the theory of linear
%inequality \cite{McMillan:TACAS04}, data-type theories
%\cite{Kapur:SIGSOFT06}, Linear arithmetic and difference logic
%\cite{Cimatti:TACAS08}, Bit-vector theories \cite{Griggio:FMCAD11},
%among others.  


In this paper, we introduce the notion of {\it Craig interpolants in
polynomial algebra over finite fields} ($\Fq$) of $q$ elements, where
$q = p^k$ is a prime power. Given a mutually inconsistent pair of sets
of polynomials with coefficients from $\Fq$ that have no common zeros,
we show that Nullstellensatz over finite fields admits
interpolation. We represent the sets $A, B$ (from Def. \ref{def:ci}) as
{varieties of corresponding ideals}, and prove the existence of
an interpolant for the pair $(A,B)$. In this setting, {\it interpolants
correspond to varieties} -- subsets of the $n$-dimensional affine
space $\Fq^n$ -- and are represented by polynomial ideals, more
precisely, by a {\it Gr\"obner basis of corresponding ideals.}

Intuitively, it should be apparent that polynomial algebra over finite
fields would admit Craig interpolation (a first order theory over
$\Fq$ definitely admits quantifier elimination \cite{gao:qe-gf-gb}).
However, our literature search for interpolants and their computation
with polynomials in arbitrary finite fields did not reveal much prior
work in this area. %turned out to be unsuccessful. 
%There is a need for the theory and algorithms for
%interpolation in this domain. 
Recent years have witnessed
investigations in formal verification, abstraction and synthesis of
datapath circuits with $k$-bit operands, where the problems have been
modeled 
%using algebraic geometry 
over finite fields ($\Fkk$) \cite{pruss:tcad} \cite{xiaojun:hldvt2016}
or over finite integer rings ($\Zkk$) \cite{sivaram:todaes}. 
%Analogous to Boolean
%function decomposition, 
%there is also a need for polynomial (word-level) datapath synthesis
Interpolants can be exploited as abstractions of functions
($f:\Fkk\rightarrow\Fkk$) in this domain, and can make these
approaches practical. Motivated by the above needs, this paper
presents the theory of Craig interpolation in finite fields, and
describes algorithms to compute them.  


{\it Contributions:} Using the extensive machinery of algebraic
geometry in finite fields,
% -- including Nullstellensatz, projections of
%varieties, elimination and extension theory, set operations on ideals
%and varieties, etc. -- 
this paper makes the following contributions: 
1) Formally define the notion of interpolants in polynomial algebra
  over finite fields $\Fq$, and prove their existence in this domain.
2) Derive the relationship of interpolants with elimination ideals,
  and show how to compute them using Gr\"obner bases. 
3) Compute the {\it smallest} interpolant, i.e. the one
  contained in every other interpolant. Analogously, compute the
  {\it largest} interpolant, i.e. the one containing all
  other interpolants. 
4) Count the total number of all possible interpolants.
5) We show how all interpolants can be enumerated in
$\mathbb{F}_2$. However, as it is impractical to explore all possible
interpolants, we present an algorithm to heuristically enumerate a few
interpolants (explore the interpolant lattice): beginning with the
smallest, progressively visiting larger ones,   and terminating at the
largest interpolant.  

{\it Paper Organization:} The following section briefly reviews prior
work in Craig interpolation in various theories, and contrasts it
against the concepts presented in this paper. Section \ref{sec:prelim} 
describes the preliminary concepts of algebraic geometry and Gr\"obner
bases in finite fields. Section \ref{sec:theory} describes the theory
of interpolation in finite fields and shows how they can be computed
using the Gr\"obner basis algorithm. Section \ref{sec:alg} describes
techniques and an algorithm to enumerate the interpolants. 
%All the concepts are also demonstrated by means of
%examples. \debug{Section \ref{sec:dis} compares our results against
%interpolant-classification in propositional logic. --  we will see
%about this} 
Section \ref{sec:exp} describes some of our experiments with unsat
instances to generate the interpolants. Section
\ref{sec:conc} concludes the paper.  Some of the proofs of the
theorems and lemmas are omitted from the main body of the manuscript
and are included in an appendix. 

%The omitted proofs of
%the theorems and lemmas are contained in an appendix.

\vspace{-0.1in}
\section{Review of Previous Work}
In the past decade or so, there has been an explosion in the study,
classification and application of interpolants. In abstraction-based
model checking, interpolants are used as over-approximate image
operators \cite{McMillan:CAV03}. In Boolean function decomposition,
given a function $F(A, B, C)$ with support variables partitioned into
disjoint subsets $A, B, C$, it is required to decompose $F = G(A, C)
\odot H(B,C)$, where $\odot$ denotes the Boolean $\vee, \wedge,
\oplus$ operations. The existence of such a 
decomposition with the given variable partition is formulated as a
unsatisfiability checking problem. Craig interpolants can then be used
to compute $G, H$ \cite{roland:bidecomp}
\cite{roland:ashenhurst}. 
%Conceptually, these problems 
%have quantifiers and interpolants can be used in lieu of the more
%expensive quantifier elimination. 
In proof complexity, interpolants have been used as a tool to derive 
lower bounds; {\it e.g.} by reasoning that if $A\implies B$ does not
have a simple interpolant, then it cannot have a simple proof
\cite{pudlak:ci}. The authors in \cite{PudlakPCFA1998} present
an interpolation theorem for Nullstellensatz refutations and the
polynomial calculus \cite{CEI:stoc-96} which can then be used for
proving lower bounds. 

The use of interpolants as abstractions has also been replicated in
other combinations of theories. For example, the theory of linear
inequality \cite{McMillan:TACAS04}, data-type theories
\cite{Kapur:SIGSOFT06}, linear arithmetic and difference logic
\cite{Cimatti:TACAS08}, bit-vector SMT theories
\cite{Griggio:FMCAD11}, etc., are just a few of the many instances
of the usage of interpolation in various domains outside of purely
propositional logic. The aforementioned works derive interpolants from
resolution proofs obtained from SAT/SMT-solvers
(\cite{Cimatti:TACAS08}), or generate them by solving constrains
in the theories of linear arithmetic with uninterpreted functions
(\cite{Rybalchenko:VMCAI-2007}),  or exploit their connection to
quantifier elimination (\cite{Kapur:SIGSOFT06}), etc. As an 
alternative to interpolation, \cite{Kovacs2009} suggests the use of
local proofs and symbol eliminating inferences for invariant
generation.  
%%  to interpolation based on symbol
%% elimination inferences is presented in  which can be 
%% applied even for theories not having the interpolation property. 
However, the problem has been insufficiently investigated over
polynomial ideals in finite fields from an algebraic geometry
perspective. 


The works that come closest to ours are by Gao {\it et al.}
\cite{gao:qe-gf-gb} and \cite{gao:gf-gb-ms}. While they do not
address the interpolation problem per se, they do describe important
results of Nullstellensatz, projections of varieties and quantifier
elimination over finite fields that we extensively utilize in this
paper.  

The work of \cite{dsilva:vmcai2010} classifies (orders) the
interpolants according to their logical strength for model 
checking. They present a labeled interpolation system built on the
resolution proof where each vertex of the proof is annotated with
partially ordered labels.  Interpolants generated from different sets
of labels have the same  order of strength as the order of the labels.
This way a (sub-)lattice of interpolants is generated with the
smallest interpolant being the same as obtained from the McMillan's
system ($L_M$) \cite{McMillan:CAV03} and the largest being the
complement of inverse of $L_M$. In contrast, we present a method for
polynomials in $\F_2$ that can generate the complete lattice of
interpolants with the absolute smallest and   absolute largest
interpolants. The labeled interpolation system of
\cite{dsilva:vmcai2010} is generalized  to support  propositional
hyper-resolution proofs \cite{Weissenbacher2012}. 
% They show how to transform a refutation proof to generate
% interpolants of various strengths. 
More recently, \cite{rummer:fmcad2013} presents the notion of
interpolation abstraction, and describes a semantic framework for
exploring interpolant lattices. In contrast to these works that
qualitatively order the interpolants w.r.t. a given application
(e.g. model checking),  we describe a method to explore interpolants
based on the cardinality of the zero-sets of polynomial ideals, which
in turn corresponds to the size of the abstraction.  

