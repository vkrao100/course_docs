\section{Conclusion}
\label{sec:conc}
This paper has presented a detailed theory and algorithm describing
the notion of Craig interpolants for a pair of polynomial ideals in
finite fields with no common zeros. The approach utilizes concepts
from computational algebraic geometry. Interpolants always exist in
this setting, and they correspond to the variety of an 
elimination ideal. In addition to defining the smallest and the largest
interpolants, techniques are described to compute them using Gr\"obner
basis concepts.  The total number of interpolants is also determined
by counting the number of points in the variety of (set) difference of
the largest and the smallest interpolants. Over the field $\F_2$, 
a technique is presented that can enumerate all possible interpolants. 
Given an interpolant, a heuristic algorithm is provided that 
returns a list of progressively larger interpolants, terminating in
the largest one. Experiments conducted demonstrate the validity of our
results. As part of future work, we are pursuing heuristic based 
methods to compute interpolants in a more controlled fashion and
classify them according to  their capability of abstraction. 
