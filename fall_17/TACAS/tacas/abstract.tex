\begin{abstract}
%Craig interpolation is an important result in logic that analyzes
%%the relationship between a mutually inconsistent pair of formulas. The
%interpolant is an abstraction, and finds applications in many areas in
%synthesis and verification. 
This paper considers Craig interpolation for a mutually inconsistent
pair of polynomial constraints over finite fields $\F_q$, for
$q$ any prime power. Using techniques from algebraic geometry, we show
that Nullstellensatz over finite fields admits Craig
interpolation. The constraints are represented as polynomial ideals with
inconsistent varieties, and it is shown how various interpolants,
including the smallest and the largest one, can be computed using the
Gr\"obner basis (GB) algorithm. The number of all possible
interpolants can also be easily identified. We describe techniques to explore and traverse
the interpolant lattice: starting with the Gr\"obner basis of the smallest
interpolant, we generate progressively larger ones, terminating in the
largest interpolant.  

\end{abstract}
