\vspace{-0.1in}
\section{Notation and Preliminary Concepts}
\label{sec:prelim}
Let $\Fq$ denote the finite field of $q$ elements where $q=p^k$ is a
prime power, $\Fqbar$ be its algebraic closure, and $R = \fqring$ the
polynomial ring in $n$ variables $x_1,\dots,x_n$, with coefficients
from $\Fq$. A monomial is a power product of the form  $X =
x_1^{e_{1}}\cdot x_2^{e_{2}}\cdots x_n^{e_{n}}$, where  
$e_i \in \Z_{\geq 0}, i\in \{1, \dots,n\}$. A {\it polynomial} $f \in
R$ is written as a finite sum of terms   
$f = c_1 X_1 + c_2 X_2 + \dots + c_t X_t$, where $c_1, \dots, c_t$ are 
coefficients and $X_1, \dots, X_t$ are monomials. Impose a monomial
order $>$ (a term order) on the ring -- i.e. a total order and a
well-order on all the monomials of $R$ s.t. multiplication with
another monomial preserves the order. Then the monomials of all
polynomials $f = c_1 X_1 + c_2 X_2 + \dots + c_t X_t$ 
are ordered w.r.t. to $>$, such that  $X_1 > X_2 > \dots >  X_t$.
Subject to $>$, $lt(f) = c_1 X_1, ~lm(f) = X_1, ~lc(f) = c_1$,
are the {\it leading term}, {\it leading monomial} and {\it   leading
  coefficient} of $f$, respectively. In this work, 
we consider mostly with lexicographic (lex) term orders.

\subsubsection{Ideals, Varieties and Gr\"obner Bases:} 
Given a set of polynomials $F = \{f_1, \dots, f_s\}$ in $R$, the {\it
  ideal} $J \subseteq R$ generated by them is: %\vspace{-0.1in} 
$J = \langle f_1, \dots, f_s \rangle = \{\sum_{i=1}^{s} h_i\cdot f_i:
~h_i \in R\}.$ The polynomials $f_1, \dots, f_s$
form the {\it basis} or the {\it   generators} of $J$.    


Let $\bm{a} = (a_1,\dots,a_n) \in \Fq^n$ be a point in the affine
space, and $f$ a polynomial in $R$. If $f(\bm{a}) = 0$, we say
that $f$ {\it vanishes} on $\bm{a}$. We have to
analyze the {\it set of all common zeros} of the polynomials of $F$
that lie %$\{f_1, f_2,\dots, f_s\}$ 
within the field $\Fq$. This zero set is called the {\it variety}. It
depends not just on the given set of polynomials but rather on the
ideal generated by them. We denote it by $V_{\Fq}(J) =
V_{\Fq}(f_1,\dots,f_s)$, where: 
$$V_{\Fq}(J) = V_{\Fq}(f_1, \dots, f_s) = \{\bm{a} \in \Fq^n: \forall
f \in J, f(\bm{a}) = 0\}.$$

Varieties can be different when restricted to the given field $\Fq$
or considered over its algebraic closure $\Fqbar$. We will generally
drop the subscript when considering varieties over $\Fq$ and
denote $V(J)$ to imply $V_{\Fq}(J)$. The subscripts will be used,
however, to avoid any ambiguities, e.g. when comparing $V_{\Fq}(J)$
against the one over the closure $V_{\Fqbar}(J)$. 

Given two ideals $J_1 = \langle f_1,\dots,f_s\rangle, J_2=\langle
h_1,\dots,h_r\rangle$, the sum $J_1 + J_2 = \langle
f_1,\dots,f_s,h_1\dots,h_r\rangle$, and their product $J_1\cdot J_2 =
\langle f_i\cdot h_j: 1\leq i\leq s, 1\leq j\leq r\rangle$. Ideals and
varieties are dual concepts: $V(J_1 + J_2) = V(J_1) \cap V(J_2)$, and
$V(J_1\cdot J_2) = V(J_1) \cup V(J_2)$. Moreover, if $J_1 \subseteq
J_2$ then $V(J_1)\supseteq V(J_2)$.

%is an ideal, and so is their
%intersection $J_1\cap J_2$. 
%The union of ideals is, in general, not an
%ideal; however, $J_1 + J_2$ is the smallest ideal containing $J_1
%\cup J_2$. 



\underline{\it Gr\"obner Basis:} An ideal may have many different sets
of generators:  $J = \langle f_1,\dots,f_s\rangle = \dots = \langle
g_1,\dots,g_t\rangle$. Given a 
non-zero ideal $J$, a {\it Gr\"obner 
  basis} (GB) for $J$ is a finite set of polynomials $G = \{g_1,\dots,
g_t\}$ satisfying $\langle \{lm(f) ~|~ f \in J\} \rangle = \langle
lm(g_1),\dots,lm(g_t)\rangle$. Then $J = \langle G \rangle$ holds and
so $G=GB(J)$ forms a basis for $J$. A GB $G$ possesses important
properties that allow to solve many polynomial computation and
decision problems. The famous Buchberger's algorithm (see Alg. 1.7.1 
in \cite{gb_book}) takes as input the set of polynomials $F =
\{f_1,\dots,f_s\}$ and computes the GB
$G=\{g_1,\dots,g_t\}$. A GB can be {\it reduced} to eliminate
redundant polynomials from the basis. A reduced GB is a canonical
representation of the ideal. In this work, the set $G$ will denote a
reduced GB, and any reference to computation of an ideal can be 
construed as constructing its GB.  
%Also, when we reason about properties of ideals
%(interpolants), the reader may assume that a $G = GB(J)$ has been

\subsubsection{Varieties over finite fields and the structure of
  Gr\"obner bases:} When the variety of an ideal is finite, then the
ideal is said to be {\it zero-dimensional}. As $V_{\Fq}(J)$ is a
finite set, $J$ is zero-dimensional. 
%% As we operate over finite fields $\Fq$, which
%% are a finite set of points, we are concerned only with
%% zero-dimensional ideals.  
A GB for a zero dimensional ideal exhibits a 
special structure that we exploit in this work. 

For all elements $\alpha \in \Fq, \alpha^q = \alpha$. Therefore, the
polynomial $x^q-x$ vanishes everywhere in $\Fq$, and is called the
vanishing polynomial of the field, sometimes also referred to as the
field polynomial. Denote by $J_0 = \langle
x_1^q-x_1,\dots,x_n^q-x_n\rangle$ the ideal of all vanishing
polynomials in the ring $R$. Then $V_{\Fq}(J_0) = V_{\Fqbar}(J_0) =
\Fq^n$. Therefore, given any ideal $J$, $V_{\Fq}(J) = V_{\Fqbar}(J)
\cap\Fq^n = V_{\Fqbar}(J) \cap V_{\Fqbar}(J_0) = V_{\Fqbar}(J+J_0) =
V_{\Fq}(J+J_0)$. 



\begin{Theorem}[{\it The Weak Nullstellensatz over finite fields (from
      Theorem 3.3 in \cite{gao:gf-gb-ms})}]
\label{thm:weak-ns-ff}
{\it For a finite field $\Fq$ and the ring $R = \Fq[x_1, \dots, x_n]$, let
$J = \langle f_1, \dots, f_s\rangle \subseteq R$, and let $J_0 = \langle
x_1^q-x_1, \dots, x_n^q -  x_n\rangle$ be the ideal of vanishing
polynomials. Then $V_{\Fq}(J) = \emptyset \iff 1 \in J + J_0 \iff G =
reducedGB(J+J_0) = \{1\}$. }
\end{Theorem}

To find whether a set of polynomials $f_1,\dots,f_s$ have no common
zeros in $\Fq$, we can compute the reduced GB $G$ of
$\{f_1,\dots,f_s,x_1^q-x_1,\dots,x_n^q-x_n\}$ and see if $G = \{1\}$. If
$G\neq\{1\}$, then $f_1,\dots,f_s$ do have common zeros in $\Fq$, and
$G$ consists of the finite set of polynomials $\{g_1,\dots,g_t\}$ with the
following properties. 

\begin{Theorem}[{\it Gr\"obner bases in finite fields (application of
      Theorem 2.2.7 from \cite{gb_book} over $\Fq$)}]
\label{thm:gb-finite}
{\it For $G = GB(J+J_0) = \{g_1,\dots,g_t\}$, the following statements
  are equivalent:
\begin{enumerate}
\item The variety $V_{\Fq}(J)$ is finite.
\item For each $i = 1,\dots, n$, there exists some
$j\in\{1,\dots,t\}$ such that $lm(g_j) = x_i^l$ for some $l\in
\mathbb{N}$. 
\item The quotient ring ${\Fq[x_1\dots,x_n]}\over{\langle G\rangle}$ forms a
  finite dimensional vector space.
\end{enumerate}
}
\end{Theorem}

In other words, the ideal $J+J_0$ is zero-dimensional, and for each
variable $x_i$, there exists an element in the GB whose leading term
is a pure power of $x_i$. When that happens, we can also count the
number of solutions. For a GB $G$, let $LM(G)$ denote the set of 
 leading monomials of all elements of $G$: $LM(G) =
 \{lm(g_1),\dots,lm(g_t)\}$.  

\begin{Definition}[{\it Standard Monomials}]
Let $\bm{X^e} = x_1^{e_1}\cdots x_n^{e_n}$ denote a monomial. The set
of standard monomials of $G$ is defined as 
$ SM(G) = \{\bm{X^e} : \bm{X^e} \notin \langle LM(G) \rangle\}.$
\end{Definition}

\begin{Theorem}[{\it Counting the number of solutions (Theorem 3.7 in
      \cite{gao:gf-gb-ms})}] 
\label{thm:count}
{\it
Let $G = GB(J+J_0)$, and $|SM(G)| = m$, then the ideal $J$ vanishes on
$m$ distinct points in $\Fq^n$. In other words, $|V(J)| = |SM(G)|.$
}
\end{Theorem}

%% We demonstrate the application of these results using an example.

%% \begin{Example}
%% Consider the ideal $J_A = \langle ab, bd, bc + c, cd, bd + b + d + 1
%% \rangle \subset \F_2[a,b,c,d]$ and $J_0 = \langle a^2 - a,b^2 - b,c^2
%% - c,d^2 -d \rangle$. Using a lex term order with $a > e > b > c > d$,
%% compute $G = GB(J+J_0) = \{cd, b+d+1 \}

%% \end{Example}
%\include{exm1}

\subsection{Radical ideals and the Strong Nullstellensatz} 
\begin{Definition}
Given an ideal $J\subset R$ and $V(J) \subseteq \Fq^n$, the {\it ideal
of polynomials that vanish on} $V(J)$ is $I(V(J)) = \{ f \in R :
\forall \bm{a} \in V(J), f(\bm{a}) = 0\}$.
\end{Definition}

If $I_1 \subset I_2$ are ideals then $V(I_1) \supset V(I_2)$, and
similarly if $V_1 \subset V_2$ are varieties, then $I(V_1) \supset
I(V_2)$. 

\begin{Definition}
For any ideal $J\subset R$, the {\bf radical} of $J$ is defined
as $\sqrt{J} = \{f \in R: \exists m \in \mathbb{N} s.t. f^m \in J\}.$
\end{Definition}

When $J = \sqrt{J}$, $J$ is called a radical ideal. Over algebraically
closed fields, the {\it Strong Nullstellensatz} establishes the
correspondence between radical ideals and varieties. Over finite
fields, it has a special form. 


\begin{Lemma}
\label{lemma:radical-ff}
(From \cite{gao:qe-gf-gb}) For an arbitrary ideal $J\subset
\Fq[x_1,\dots,x_n]$, and  $J_0 = \langle
x_1^q-x_1,\dots,x_n^q-x_n\rangle$, the ideal $J + J_0$ is radical; 
i.e. $\sqrt{J+J_0} = J+J_0$. 
\end{Lemma}


\begin{Theorem}[{\it The Strong Nullstellensatz over finite fields
   (Theorem 3.2 in \cite{gao:qe-gf-gb})}] \label{thm:strong-ns}  
For any ideal $J \subset \Fq[x_1,\dots,x_n], ~I(V_{\Fq}(J)) = J + J_0$.
\end{Theorem}

%% \begin{proof}
%% $I(V(J)) = I(V_{\Fq}(J))  = I(V_{\Fqbar}(J + J_0) = \sqrt{J+J_0} = J + J_0$.
%% \end{proof}

\subsection{Projection of varieties and elimination ideals in finite
  fields} 

\begin{Definition}
Given an ideal $J = \langle f_1,\dots, f_s \rangle \subset R$ and its
variety $V(J) \subset \Fq^n$,  
the $l$-th projection of $V(J)$ denoted as $Pr_l(V(J))$ is the mapping
\begin{center}
$Pr_l(V(J)):\Fq^n \rightarrow \Fq^{n-l}, ~Pr_l(a_1,\dots,a_n) = (a_{l+1},\dots,a_n) $
\end{center}
for every $\bm{a} = (a_1,\dots,a_n) \in V(J)$.
\end{Definition}
% The projection of variety of $J_A$ from Example \ref{example:ja} on
% the variable set $C$ is $Pr_A(\Vac(J_A))$ and is equal to $(bcd):\{100,110,001\}$.

\begin{Definition}
Given an ideal $J \subset \Fq[x_1,\dots,x_n]$, the $l$-th elimination
ideal $J_l$ is an ideal in $R$ defined as $J_l = J \cap \Fq[x_{l+1},\dots,x_n]$.
\end{Definition}

The next theorem shows how we can obtain the generators of the $l$-th
elimination ideal using Gr\"obner bases.

\begin{Theorem}[{\it Elimination Theorem \cite{ideals:book}}]
Given an ideal $J \subset R$ and its GB $G$ $w.r.t.$ the
lexicographical (lex) order on the variables 
where $x_1 > x_2 > \cdots > x_n$, then for every $0 \leq l \leq n$ we
denote by $G_l$ the GB of $l$-th elimination ideal of $J$ and compute it as:
\begin{center}
$G_l = G \cap \Fq[x_{l+1},\dots,x_n]$
\end{center}
\end{Theorem}
% The elimination ideal corresponding to $J_A$ from Example \ref{example:ja}
% that eliminates the variables from the set $A$ is 
% $\langle cd,b+d+1 \rangle$ and its variety $\{001,100,110\}$.

In a general setting, the projection of a variety is a subset of the
variety of an elimination ideal: $Pr_l(V(J)) \subseteq V(J_l)$. However,
operating over finite fields, when the ideals contain the vanishing
polynomials, then the above set inclusion turns into an equality.


\begin{Lemma}[Lemma 3.4 in \cite{gao:qe-gf-gb}]
\label{lemma:project}
Given an ideal $J \subset R$ that contains the vanishing polynomials of 
the field, then $Pr_l(V(J)) = V(J_l)$, 
i.e. the $l$-th projection of the variety of ideal $J$ is equal to 
the variety of its $l$-th elimination ideal.

\end{Lemma}

%% \begin{Theorem}[Extension Theorem \cite{coxbook}] 
%% Given the ideal $J \in R$ and its first elimination ideal $J_1$,
%% write each generator $f_i (1\leq i\leq s)$ of $J$ in the form,
%% \begin{center}
%% $f_i = h_i(x_2,\dots,x_n)\cdot x_1^{N_i} + \text{monomials with degree of $x_1 < N_i$}$,
%% \end{center}
%% where $N_i \geq 0$ and $h_i \in \Fq[x_2,\dots,x_n]$ is non-zero. Let's say that there is
%% point $(a_2,\dots,a_n)$ in $V(I_1)$. If $(a_2,\dots,a_n) \not \in V(h_1,\dots,h_s)$,
%% then there exists $a_1 \in \Fq$ such that 
%% $(a_1,a_2,\dots,a_n) \in V(J)$.   
%% \end{Theorem}

%% In other words, if the condition $(a_2,\dots,a_n) \not \in V(h_1,\dots,h_s)$ is satisfied,
%% then the point $(a_2,\dots,a_n) \in V(I_1)$ can be extended to a point
%% $(a_1,a_2,\dots,a_n) \in V(J)$.
%% \par For an ideal $J$ that contains the vanishing polynomials, its GB
%% $G = \{g_1,\dots,g_t\}$ has the property that for each variable in $\{x_1,\dots,x_n\}$
%% there must be some polynomial $g_i$ such that $lm(g_i) = x_i^l$ for $l \in \mathbb{N}$.
%% Therefore, every point in $V(G_l)$ can be extended to a point in $V(G)$.  

We will utilize all of the above concepts to derive the results in
this paper. 
