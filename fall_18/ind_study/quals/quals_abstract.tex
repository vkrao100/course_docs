\documentclass[2pt]{article}
\usepackage{algorithm}
\usepackage{algpseudocode}
\usepackage{svg}
\usepackage{color}
\usepackage{soul}
\usepackage{xcolor}
\usepackage{amsthm}
\usepackage{float}
\usepackage{sectsty}
\usepackage{amsbsy}
\usepackage{amsthm}
\usepackage{amsmath}
\usepackage{amsfonts}
\usepackage{graphicx}
\usepackage{multirow}
\usepackage{diagbox}
\usepackage{bm}
\usepackage{hhline}
\usepackage{graphicx}
\usepackage{helvet}
\usepackage{enumerate}
\usepackage{amsmath}
\usepackage{amsfonts}
\usepackage{graphicx}
\usepackage{multirow}
\usepackage{subfig}
\usepackage{comment}
\usepackage{mathtools}

\newtheorem{Algorithm}{Algorithm}[section]
\newtheorem{Definition}{Definition}[section]
\newtheorem{Example}{Example}[section]
\newtheorem{Proposition}{Proposition}[section]
\newtheorem{Lemma}{Lemma}[section]
\newtheorem{Theorem}{Theorem}[section]
\newtheorem{Corollary}{Corollary}[section]
\newtheorem{Proof}{Proof}

\newcommand{\B}{{\mathbb{B}}}
\newcommand{\Z}{{\mathbb{Z}}}
\newcommand{\R}{{\mathbb{R}}}
\newcommand{\Q}{{\mathbb{Q}}}
\newcommand{\N}{{\mathbb{N}}}
\newcommand{\C}{{\mathbb{C}}}
\newcommand{\Zn}{{\mathbb{Z}}_{n}}
\newcommand{\Zp}{{\mathbb{Z}}_{p}}
\newcommand{\F}{{\mathbb{F}}}
\newcommand{\Fbar}{{\overline{\mathbb{F}}}}
\newcommand{\Fq}{{\mathbb{F}}_{q}}
\newcommand{\Jc}{J_c}
\newcommand{\Jz}{J_0}
\newcommand{\Jb}{J_b}
\newcommand{\Jt}{J_2}
\newcommand{\al}{\alpha}
\newcommand{\Joz}{J_1 + J_0}
\newcommand{\Jtz}{J_2 + J_0}
\newcommand{\Fqbar}{{\overline{{\mathbb{F}}_q}}}
\newcommand{\Fkk}{{\mathbb{F}}_{2^k}}
\newcommand{\Zkk}{{\mathbb{Z}}_{2^k}}
\newcommand{\Fkkx}[1][x]{\ensuremath{\mathbb{F}}_{2^k}[#1]\xspace}
\newcommand{\Grobner}{Gr\"{o}bner\xspace}
\newcommand{\bi}{\begin{itemize}}
\newcommand{\ei}{\end{itemize}}

\newcommand{\idealj}{{J = \langle f_1,f_2 \dots, f_s\rangle}}
\newcommand{\idealg}{{J = \langle g_1, \dots, g_t\rangle}}
\newcommand{\vfqj}{{V_{\Fq}(J)}}
\newcommand{\vfqjo}{{V_{\Fq}(J_0)}}
\newcommand{\vfbqj}{{V_{\overline{\Fq}}(J)}}
\newcommand{\vfbqjo}{{V_{\overline{\Fq}}(J_0)}}
\newcommand{\vfbqjjo}{{V_{\overline{\Fq}}(J+J_0)}}
\newcommand{\vfkkj}{{V_{\Fkk}(J)}}
% \newcommand{\v}{\vee}
\newcommand{\acf}{\bar{F}_q}
\newcommand{\Vacf}{V_{\bar{F}_q}}

\sectionfont{\small}

\title{{R}ectification of {A}rithmetic {C}ircuits using {C}omputer {A}lgebra {T}echniques}
\author{Vikas Rao\\
Department of  Electrical and Computer Eng.\\
University of Utah\\Vikas.k.rao@utah.edu }
\date{}
\begin{document}

\maketitle
Formal verification of Arithmetic circuits checks whether 
or not a gate-level circuit correctly implements a
given specification model. In cases where this equivalence
check fails – the presence of a bug is detected. Subsequently, it is required
to: i) debug the circuit, ii) identify a set of nets (signals)
where the circuit might be rectified, and iii) compute the
corresponding rectification functions at those locations. As a preliminary work, 
I addressed the problem of debugging
and correction (rectification) in finite field arithmetic circuits\cite{vkrao:fmcad18}.
I presented techniques that
determine whether the circuit can be rectified at one particular
net (gate output) – i.e. a single-fix rectification issue is addressed.
Starting from an equivalence checking setup modeled as
a polynomial ideal membership test, I analyzed the ideal
membership residue to identify potential single-fix rectification
locations. Subsequently, Nullstellensatz principles were used to
ascertain if indeed a single-fix rectification can be applied at
any of these locations. If a single-fix rectification exists, a
rectification function is derived by modeling it as the synthesis
of an unknown component problem\cite{vkrao:iwls18}.  

% It takes considerable amount of time for verification and
% rectification check when the bug is close to the output. My eventual goal is
% to work on further improving the experiments by employing
% better data structures, and devising better
% heuristics to perform rectification check. 
As part of my future work, I intend to deep-dive into the theory of
permissible functions to explore the synthesis side of the 
problem in finding a more deterministic solution from the given solution space.
There is also a need for further investigation on 
how the current procedure can be extended to cover integer 
arithmetic circuits. Further research also needs to address 
exploring the current approach for the case of multi-fix rectification. 

% Since the techniques from computer algebra such as
% Grobner basis are considered both decision and quantification procedure, 
% there is a need for investigation on how to integrate 
% these concepts with the modus operandi of synthesis.

\bibliographystyle{ieeetr}
\bibliography{vikas}

\end{document}
